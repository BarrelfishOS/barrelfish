% BEGIN LICENSE BLOCK
% Version: CMPL 1.1
%
% The contents of this file are subject to the Cisco-style Mozilla Public
% License Version 1.1 (the "License"); you may not use this file except
% in compliance with the License.  You may obtain a copy of the License
% at www.eclipse-clp.org/license.
%
% Software distributed under the License is distributed on an "AS IS"
% basis, WITHOUT WARRANTY OF ANY KIND, either express or implied.  See
% the License for the specific language governing rights and limitations
% under the License.
%
% The Original Code is  The ECLiPSe Constraint Logic Programming System.
% The Initial Developer of the Original Code is  Cisco Systems, Inc.
% Portions created by the Initial Developer are
% Copyright (C) 1996 - 2006 Cisco Systems, Inc.  All Rights Reserved.
%
% Contributor(s):
%
% END LICENSE BLOCK
%
% @(#)umsarrays.tex     1.4
%

\chapter{Non-logical Storage and References}
\label{chaparrays}
%HEVEA\cutdef[1]{section}

%----------------------------------------------------------------------
\section{Introduction}
%----------------------------------------------------------------------

This chapter describes primitives that allow one to break the normal logic
programming rules in two ways:
\begin{itemize}
\item information can be \emph{saved across logical failures} and backtracking;
\item information can be accessed by \emph{naming} rather than by argument
  passing.
\end{itemize}
Obviously, these facilities must be used with care and should always
be encapsulated in an interface that provides logical semantics.

ECLiPSe provides several facilities to store information across
backtracking.  The following table gives an overview.  If at all
possible, the handle-based facilities (bags, shelves and stores) should be
preferred because they lead to cleaner, reentrant code (without global
state) and reduce the risk of memory leaks.

\vspace{2mm}
\begin{center}
\begin{tabular}{l|l l l}
Facility & Type & Access & See \\
\hline
shelves&                array&          by handle&      shelf_create/2,3 \\
bags&                   unordered bag&  by handle&      bag_create/1 \\
stores&                 hash table&     by handle&      store_create/1 \\
anonymous records&      ordered list&   by handle&      record_create/1 \\
named shelves&          array&          by name&        shelf/2 \\
named stores&           hash table&     by name&        store/1 \\
non-logical variables&  single cell&    by name&        variable/1 \\
non-logical arrays&     array&          by name&        array/1,2 \\
records&                ordered list&   by name&        record/1,2 \\
dynamic predicates&     ordered list&   by name&        dynamic/1, assert/1 \\
\end{tabular}
\end{center}
\vspace{2mm}

\index{reference}
\index{global reference}
The other facility described here, \emph{Global References}, does not store
information across failure, but provides a means to give a name to
an otherwise logical data structure. See section \ref{globrefs}.


%----------------------------------------------------------------------
\section{Bags}
%----------------------------------------------------------------------

A bag is an anonymous object which can be used to store
information across failures.
A bag is unordered and untyped. Any {\eclipse} term can be stored in a bag.
Bags are referred to by a handle.
An empty bag is created using
\bipref{bag_create/1}{../bips/kernel/storage/bag_create-1.html},
data is stored in the bag by invoking
\bipref{bag_enter/2}{../bips/kernel/storage/bag_enter-2.html},
and the stored data can be retrieved as a list with
\bipref{bag_retrieve/2}{../bips/kernel/storage/bag_retrieve-2.html} or
\bipref{bag_dissolve/2}{../bips/kernel/storage/bag_dissolve-2.html}.

A typical application is the
implementation of the \predspec{findall/3} predicate or similar functionality.
As opposed to the use of \predspec{record/2} or \predspec{assert/1}, the
solution using
bags is more efficient, more robust, and trivially reentrant.
\begin{verbatim}
    simple_findall(Goal, Solutions) :-
        bag_create(Bag),
        (
            call(Goal),
            bag_enter(Bag, Goal),
            fail
        ;
            true
        ),
        bag_dissolve(Bag, Solutions).
\end{verbatim}


%----------------------------------------------------------------------
\section{Shelves}
%----------------------------------------------------------------------

A \defnotion{shelf} is an anonymous object which can be used to store information
across failures.  A typical application is counting of solutions,
keeping track of the best solution, aggregating information across
multiple solutions, etc.

A shelf is an object with multiple slots whose contents survive
backtracking.  The content of each slot can be set and retrieved
individually, or the whole shelf can be retrieved as a term.

Shelves are referred to by handle, not by name, so they make it easy
to write robust, reentrant code.  A shelf disappears when the system
backtracks over its creation, when the shelf handle gets garbage
collected, or when it is explicitly destroyed.

A shelf is initialized using
\bipref{shelf_create/2}{../bips/kernel/storage/shelf_create-2.html} or
\bipref{shelf_create/3}{../bips/kernel/storage/shelf_create-3.html}.
Data is stored in the slots (or the shelf as a whole) with
\bipref{shelf_set/3}{../bips/kernel/storage/shelf_set-3.html}
and retrieved with
\bipref{shelf_get/3}{../bips/kernel/storage/shelf_get-3.html}.

Example: Counting how many solutions a goal has:
\begin{verbatim}
    count_solutions(Goal, Total) :-
        shelf_create(count(0), Shelf),
        (
            call(Goal),
            shelf_get(Shelf, 1, Old),
            New is Old + 1,
            shelf_set(Shelf, 1, New),
            fail
        ;
            shelf_get(Shelf, 1, Total)
        ).
\end{verbatim}
In this particular example, we could have used
\bipref{shelf_inc/2}{../bips/kernel/storage/shelf_inc-2.html} to
increment the counter.


%----------------------------------------------------------------------
\section{Stores}
%----------------------------------------------------------------------


\index{hash table}
A \defnotion{store} is an anonymous object which can be used to store information
across failures.  A typical application is aggregating information across
multiple solutions.  Note that, if it is not necessary to save information
across backtracking, the use of the
\bipref{library(hash)}{../bips/lib/hash/index.html}
is more appropriate and efficient than the facilities described here.

A store is a hash table which can store arbitrary terms under arbitrary
ground keys. Modifications of a store, as well as the entries within it,
survive backtracking.  The basic operations on stores are entering and
looking up information under a key, or retrieving the store contents as
a whole.

Stores come in two flavours:  anonymous stores are created with
\bipref{store_create/1}{../bips/kernel/storage/store_create-1.html}
and referred to by handle, while named stores are created with a
\bipref{store/ 1}{../bips/kernel/storage/store-1.html}
declaration and referred to by their name within a module.
If possible, anonymous stores should be preferred because they make it
easier to write robust, reentrant code.  For example, an anonymous store
automatically disappears when the system backtracks over its creation,
or when the store handle gets garbage collected.  Named stores, on the
other hand, must be explicitly destroyed in order to free the
associated memory.

Data is entered into a store using
\bipref{store_set/3}{../bips/kernel/storage/store_set-3.html}
and retrieved using
\bipref{store_get/3}{../bips/kernel/storage/store_get-3.html}.
It is also possible to retrieve information: either all keys with
\bipref{stored_keys/2}{../bips/kernel/storage/stored_keys-2.html},
or the full contents of the table with
\bipref{stored_keys_and_values/2}{../bips/kernel/storage/stored_keys_and_values-2.html}.
Entries can be deleted via
\bipref{store_delete/2}{../bips/kernel/storage/store_delete-2.html} or
\bipref{store_erase/1}{../bips/kernel/storage/store_erase-1.html}.

\index{memoization}
\index{Fibonacci}
\indextt{fib/2}
A typical use of stores is for the implementation of memoization.
The following is an implementation of the Fibonacci function, which
uses a store to remember previously computed results.
It consists of the declaration of a named store, a wrapper predicate
\predspec{fib/2} that handles storage and lookup of results, and the standard
recursive definition \predspec{fib_naive/2}:
\begin{verbatim}
    :- local store(fib).

    fib(N, F) :-
        ( store_get(fib, N, FN) ->
            F = FN                    % use a stored result
        ;
            fib_naive(N, F),
            store_set(fib, N, F)      % store computed result
        ).

    fib_naive(0, 0).
    fib_naive(1, 1).
    fib_naive(N, F) :-
        N1 is N-1, N2 is N-2,
        fib(N1, F1), fib(N2, F2),
        F is F1 + F2.
\end{verbatim}
Using this definition, large function values can be efficiently computed:
\begin{verbatim}
    ?- fib(300, F).
    F = 222232244629420445529739893461909967206666939096499764990979600
    Yes (0.00s cpu)
\end{verbatim}



The next example shows the use of an anonymous store, for computing
how many solutions of each kind a goal has.
The store is used to maintain counter values, using the
solution term as the key to distinguish the different counters:
\begin{verbatim}
    solutions_profile(Sol, Goal, Profile) :-
        store_create(Store),
        (
            call(Goal),
            store_inc(Store, Sol),
            fail
        ;
            stored_keys_and_values(Store, Profile)
        ).
\end{verbatim}
Running this code produces for example:
\begin{verbatim}
    ?- solutions_profile(X, member(X, [a, b, c, b, a, b]), R).
    X = X
    R = [a - 2, b - 3, c - 1]
    Yes (0.00s cpu)
\end{verbatim}


%----------------------------------------------------------------------
\section{Non-logical Variables}
%----------------------------------------------------------------------
\index{non-logical variable}\index{variable!non-logical}
Non-logical variables in {\eclipse} are a means of storing a copy
of a Prolog term under a name (an atom).
The atom is the \defnotionni{name} and the associated
term is the \defnotionni{value} of the non-logical variable.
This term may be of any form, whether an integer or a huge compound
structure.
Note that the associated term is being copied and so if it is not ground,
the retrieved term is not strictly identical to the stored one
but is a \emph{variant} of it\footnote{%
  Though this feature could be used to make a copy of a term with new variables,
  it is cleaner and more efficient to use
  \bipref{copy_term/2}{../bips/kernel/termmanip/copy_term-2.html} for that
  purpose}.
There are two fundamental operations that can be performed on a non-logical
variable:
setting the variable (giving it a value), and referencing the variable
(finding the value currently associated with it).

The value of a non-logical variable is set using the
\bipref{setval/2}{../bips/kernel/storage/setval-2.html}
predicate.
This has the format
\begin{quote}
\preddef{setval(\pattern{Name},~\pattern{Value})}\indextt{setval/2}
\end{quote}
For instance, the goal
\begin{quote}
\begin{verbatim}
setval(firm, 3)
\end{verbatim}
\end{quote}
gives the non-logical variable \about{firm} the value 3.
The value of a non-logical variable is retrieved using the
\bipref{getval/2}{../bips/kernel/storage/getval-2.html}
predicate.

The goal
\begin{quote}
\begin{verbatim}
getval(firm, X)
\end{verbatim}
\end{quote}
will unify \about{X} to
the value of the non-logical variable \about{firm}, which has been previously
set by
\bipref{setval/2}{../bips/kernel/storage/setval-2.html}.
If no value has been previously set, the call raises an exception.
If the value of a non-logical variable is an integer, the predicates
\bipref{incval/1}{../bips/kernel/storage/incval-1.html}
and \bipref{decval/1}{../bips/kernel/storage/decval-1.html}
may be used toincrement and decrement the value of the
variable, respectively.
The predicates \bipref{incval/1}{../bips/kernel/storage/incval-1.html} and
\bipref{decval/1}{../bips/kernel/storage/decval-1.html} may be used e.g., in a
failure-driven loop to provide
an incremental count across failures as in the example:
\begin{quote}
\begin{verbatim}
count_solutions(Goal, _) :-
        setval(count, 0),
        call(Goal),
        incval(count),
        fail.
count_solutions(_, N) :-
        getval(count, N).
\end{verbatim}
\end{quote}
However, code like this should be used carefully.
Apart from being a non-logical feature, it also causes the code to be
not reentrant. That is, if \predspec{count_solutions/2} were called recursively
from
inside \about{Goal}, this would smash the counter and yield incorrect
results.\footnote{%
  A similar problem can occur when the counter is used by an interrupt handler.}

The visibility of a non-logical variable is local to the module
where it is first set. It is good style to declare them using
\bipref{local/1}{../bips/kernel/modules/local-1.html}
\bipref{variable/1}{../bips/kernel/storage/variable-1.html}
declarations. For example, in the above example one should use
\begin{quote}
\begin{verbatim}
:- local variable(count).
\end{verbatim}
\end{quote}
If it is necessary to access the value of a variable in other modules,
exported access predicates should be provided.


%----------------------------------------------------------------------
\section{Non-logical Arrays}
%----------------------------------------------------------------------
\index{non-logical array}\index{array!non-logical}
Non-logical arrays are a generalisation of the non-logical variable, capable of
storing
multiple values.
An array has to be declared in advance.
It has a fixed number of dimensions and a fixed size in each dimension.
Arrays in {\eclipse} are managed solely by special predicates.
In these predicates, arrays are represented by compound terms, e.g.,
\notation{matrix(5, 8)}
where \about{matrix} is the name of the array, the arity of 2 specifies
the number of dimensions, and the integers 5 and 8 specify the size
in each dimension. The number of elements this array can hold is
thus 5*8 = 40.
The elements of this array can be addressed from \notation{matrix(0, 0)}
up to \notation{matrix(4, 7)}.

An array must be explicitly created using a
\bipref{local/1}{../bips/kernel/modules/local-1.html}
\bipref{array/1}{../bips/kernel/storage/array-1.html}
declaration, e.g.,
\begin{quote}
\begin{verbatim}
:- local array(matrix(5, 8)).
\end{verbatim}
\end{quote}
The array is only accessible from within the module where it was declared.
The declaration will create a two-dimensional, 5-by-8 array with 40 elements
\notation{matrix(0, 0)} to \notation{matrix(4, 7)}.
Arrays can be  erased using the predicate
\bipref{erase_array/1}{../bips/kernel/storage/erase_array-1.html},
e.g.,
\begin{quote}
\begin{verbatim}
erase_array(matrix/2).
\end{verbatim}
\end{quote}

The value of an element of the array is set using the
\bipref{setval/2}{../bips/kernel/storage/setval-2.html}
predicate. The first argument of
\bipref{setval/2}{../bips/kernel/storage/setval-2.html} specifies the element
which is
to be set, the second specifies the value to assign to it.
The goal
\begin{quote}
\begin{verbatim}
setval(matrix(3, 2), plato)
\end{verbatim}
\end{quote}
sets the value
of element (3, 2) of array \about{matrix} to the atom \about{plato}.
Similarly, values of array elements are retrieved by use of the
\bipref{getval/2}{../bips/kernel/storage/getval-2.html}
predicate. The first argument of
\bipref{getval/2}{../bips/kernel/storage/getval-2.html} specifies the element to
be
referenced, the second is unified with the value of that element.
Thus if the value of \notation{matrix(3, 2)} had been set as above, the goal
\begin{quote}
\begin{verbatim}
getval(matrix(3, 2), Val)
\end{verbatim}
\end{quote}
would unify \about{Val} with the atom \about{plato}.
Similarly to non-logical variables, the value of integer array elements
can be updated using \bipref{incval/1}{../bips/kernel/storage/incval-1.html} and
\bipref{decval/1}{../bips/kernel/storage/decval-1.html}.

It is possible to declare arrays whose elements are
constrained to belong to certain types. This allows {\eclipse} to increase
time and space efficiency of array element manipulation.
Such an array is created for instance by the predicate
\begin{quote}
\begin{verbatim}
:- local array(primes(100),integer).
\end{verbatim}
\end{quote}
The second argument specifies the type of the elements of the array.
It takes as value an atom from the
list {\tt float} (for floating point numbers),
{\tt integer} (for integers), {\tt byte} (an integer modulo 256),
or {\tt prolog} (any Prolog term---the resulting array is the
same as if no type was specified).
When a typed array is created, the value of each element is initialized to zero
in the case of {\tt byte}, {\tt integer} and {\tt float}, and to
an uninstantiated variable in the case of {\tt prolog}.
Whenever a typed array element is set, type checking is carried out.

As an example of the use of a typed array, consider the following goal, which
creates a 3-by-3 matrix describing a 90 degree rotation about the x-axis of
a Cartesian coordinate system.
\begin{quote}
\begin{verbatim}
:- local array(rotate(3, 3), integer).
:- setval(rotate(0, 0), 1),
   setval(rotate(1, 2), -1),
   setval(rotate(2, 1), 1).
\end{verbatim}
\end{quote}
(The other elements of the above array are automatically initialized to zero).

The predicate
\bipref{current_array/2}{../bips/kernel/storage/current_array-2.html}
is provided to find the size, type and visibility of defined arrays.
of the array and its type to be found:
\begin{quote}
\preddef{current_array(\pattern{Array},~\pattern{Props})}
\end{quote}
where \about{Array} is the array specification as in the declaration (but it
may be uninstantiated or partially instantiated), and \about{Props} is
a list indicating the array's type and visibility.
Non-logical variables are also returned: \about{Array} is then an atom, and
the type is \notation{prolog}.
\begin{quote}
\begin{verbatim}
[eclipse 1]: local(array(pair(2))),
        setval(count, 3),
        local(array(count(3,4,5), integer)).

yes.
[eclipse 2]: current_array(Array, Props).

Array = pair(2)
Props = [prolog, local]     More? (;)

Array = count
Props = [prolog, local]     More? (;)

Array = count(3, 4, 5)
Props = [integer, local]     More? (;)

no (more) solution.
[eclipse 3]: current_array(count(X,Y,Z), _).

X = 3
Y = 4
Z = 5
yes.
\end{verbatim}
\end{quote}


%----------------------------------------------------------------------
\section{Global References}
%----------------------------------------------------------------------
\label{globrefs}
Terms stored in non-logical variables and arrays are copies of the
\bipref{setval/2}{../bips/kernel/storage/setval-2.html} arguments,
and the terms obtained by
\bipref{getval/2}{../bips/kernel/storage/getval-2.html} are thus not identical
to the original terms, in particular their variables are different.
Sometimes it is necessary to be able
to access the original term with its variables, i.e., to have
\emph{global variables} in the meaning of conventional programming
languages.
A typical example is global state that a set of predicates wants to
share without having to pass an argument pair through all the
predicate invocations.

{\eclipse} offers the possibility to store references to general terms
and to access them even inside predicates that have no common variables
with the predicate that has stored them.
They are stored in so-called \defnotioni{references}{reference}.%
\index{global reference}
For example,
\begin{quote}
\begin{verbatim}
:- local reference(p).
\end{verbatim}
\end{quote}
or
\begin{quote}
\begin{verbatim}
:- local reference(p, 0).
\end{verbatim}
\end{quote}
creates a named reference \about{p} (with an initial value of 0)
which can be used to store references to terms.
This reference is accessed and modified in the same way as non-logical
variables,
with \bipref{setval/2}{../bips/kernel/storage/setval-2.html}
and \bipref{getval/2}{../bips/kernel/storage/getval-2.html},
but the following points are different for references:
\begin{itemize}
\item the accessed term is identical to the stored term (with its current
substitutions):
\begin{quote}
\begin{verbatim}
[eclipse 1]: local reference(a), variable(b).

yes.
[eclipse 2]: Term = p(X), setval(a, Term), getval(a, Y), Y == Term.
X = X
Y = p(X)
Term = p(X)
yes.
[eclipse 3]: Term = p(X), setval(b, Term), getval(b, Y), Y == Term.

no (more) solution.
\end{verbatim}
\end{quote}

\item the modifications are backtrackable, when the execution fails
over the \bipref{setval/2}{../bips/kernel/storage/setval-2.html} call, the
previous value of the global reference is restored

\begin{quote}
\begin{verbatim}
[eclipse 4]: setval(a, 1), (setval(a, 2), getval(a, X); getval(a, Y)).
X = 2
Y = Y     More? (;)

X = X
Y = 1
\end{verbatim}
\end{quote}

\item there are no arrays of references, but the same effect can be
achieved by storing a structure in a reference and using the structure's
arguments. The arguments can then be accessed and modified using
\bipref{arg/3}{../bips/kernel/termmanip/arg-3.html} and
\bipref{setarg/3}{../bips/kernel/termmanip/setarg-3.html} respectively.
\end{itemize}

%There is only a limited number of references available and their
%use should be considered very carefully.
The use of references should be considered carefully.
Their overuse can lead to programs which are
difficult to understand and difficult to optimize.
Typical applications use at most a single reference per module,
for example to hold state that would otherwise have to be passed
via additional arguments through many predicate invocations.

%HEVEA\cutend
