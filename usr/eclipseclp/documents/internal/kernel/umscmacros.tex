% BEGIN LICENSE BLOCK
% Version: CMPL 1.1
%
% The contents of this file are subject to the Cisco-style Mozilla Public
% License Version 1.1 (the "License"); you may not use this file except
% in compliance with the License.  You may obtain a copy of the License
% at www.eclipse-clp.org/license.
% 
% Software distributed under the License is distributed on an "AS IS"
% basis, WITHOUT WARRANTY OF ANY KIND, either express or implied.  See
% the License for the specific language governing rights and limitations
% under the License. 
% 
% The Original Code is  The ECLiPSe Constraint Logic Programming System. 
% The Initial Developer of the Original Code is  Cisco Systems, Inc. 
% Portions created by the Initial Developer are
% Copyright (C) 1988 - 2006 Cisco Systems, Inc.  All Rights Reserved.
% 
% Contributor(s): David Miller / Joachim Schimpf, ECRC 
% 
% END LICENSE BLOCK

%

%%%%%%%%%%%%%%%%%%%%%%%%%%%%%%%%%%%%%%%%%%%%%%%%%%%%%%%%%%%%%%%%%%%%%%%%%%
% \eclipse Documentation
%
% umsmacros.tex
%
% REL	DATE	AUTHOR		DESCRIPTION
% 2.10	090489	David Miller	Update for Latex
% 3.0	140590	Joachim Schimpf	Update for 3.0
%
%%%%%%%%%%%%%%%%%%%%%%%%%%%%%%%%%%%%%%%%%%%%%%%%%%%%%%%%%%%%%%%%%%%%%%%%%%
% \comment{@(\#)appendices.mss	20.3 9/19/88}
% \part{appendices, root = `manual.mss'}
% \pageheading{Odd, Right `\title{Chapter}'}
% \pageheading{Even, Left `Appendix \ref{Chapter}'}

\chapter{'C' Macros used in External Built-In Predicates}
\label{chapcmacros}

The following is a summary of the macros used in the C code implementing
most of the {\eclipse} built-in predicates.
This set of macros was historically the first "external predicate
programming interface", but was later replaced by the safer "embedding
interface" described in the Embedding Manual.
The built-in predicates implemented in C still use this old, lower
level interface.

TAt least the following set of include files
must be included at the start of any file containing external predicate
definitions:
\begin{quote}
\begin{verbatim}
#include  "config.h"
#include  "sepia.h"
#include  "types.h"
#include  "error.h"
#include  "dict.h"
\end{verbatim}
\end{quote}

\newpage
\section{Returns}
The following macros contain a {\tt return} statement and so they
cannot be used inside nested functions called by the external procedure.

\noindent
 \\
\begin{tabular}{|p{5cm}p{10cm}|} 
\hline
{\tt Succeed_} & return from procedure successfully.\\  

{\tt Fail_} & return from procedure as a failure.\\ 

{\tt Succeed_If(expr)} &
return successfully iff {\tt expr} is true, fail otherwise.\\

{\tt Succeed_Last} &
returns successfully from a backtrackable predicate.\\

{\tt Bip_Error(code)} &
abnormal return from procedure. {\tt code} is one of the
error codes defined in `error.h'. The most commonly used will be:
\begin{itemize}
\item SYS_ERROR --- operating system error.
\item INSTANTIATION_FAULT --- unexpected variable.
\item TYPE_ERROR --- wrong type.
\item RANGE_ERROR --- out of range.
\item ARITH_EXCEPTION --- e.g.\ division by 0.
\end{itemize}
\\

{\tt Set_Errno} &
used when the system variable {\tt errno} contains the error code
of a system call, before the predicate returns using
{\tt Error(SYS_ERROR)}.\\
\hline
\end{tabular}

%\newpage

\section{Dereferencing}
A variable should be dereferenced before being used. The following
macro makes sure that a pointer points to the end of a reference
chain, if there is any.

\noindent
 \\
\begin{tabular}{|p{5cm}p{10cm}|} 
\hline
{\tt Dereference_(ref)} & 
{\tt ref} is a pointer to a {\tt pword} structure.
If this {\tt pword} is not a reference,
nothing is done. If the pointed {\tt pword} is a reference, 
this macro follows 
the whole reference chain and changes the value of {\tt ref}
so that it points to the chain end.
If {\tt ref} or the end of the reference chain is a free variable, 
{\tt ref} will point to the variable itself. The arguments of
external predicates are dereferenced before being passed to the 
predicate, so there is no need to dereference them again.
The compound arguments, on the other hand, may contain
subarguments that might have to be dereferenced.
Since the value of {\tt ref} may be changed by this macro, it might
be necessary to save its previous value. \\
\hline
\end{tabular}

\newpage
\section{Type Checking}
When checking the type, do the following :
\begin{itemize}
\item dereference the value; this is not necessary for the arguments
of the external predicate. 

%\item check for a reference, using {\tt IsRef()}. 

\item check for the type. 
\end{itemize}
When the tag is been compared directly to a tag value (in the 
{\tt switch} statement), the macro {\tt Ptag(tag.kernel)} must be used.
The following macros return true if their argument (a tag) is of the correct 
type.

\noindent
 \\
\begin{tabular}{|p{5cm}p{10cm}|} 
\hline
{\tt IsRef(tag)} &	a reference or variable (self reference). \\
{\tt IsVar(tag)} &	a simple variable. \\
{\tt IsName(tag)} &	a named variable. \\
{\tt IsMeta(tag)} &	a metaterm (attributed variable). \\

{\tt IsList(tag)} & 	a pair, i.e.\ a nonempty list. \\

{\tt IsStructure(tag)} & a compound term (not a list). \\ 

{\tt IsString(tag)} & a string. \\

{\tt IsNil(tag)} &	{\it nil}. \\

{\tt IsInteger(tag)} & an integer. \\

{\tt IsFloat(tag)} &	single precision float number. \\

{\tt IsDouble(tag)} &	double precision float number. \\
{\tt IsBignum(tag)} &	large integer number. \\
{\tt IsRational(tag)} &	rational number. \\
{\tt IsInterval(tag)} &	 breal number. \\

{\tt IsAtom(tag)} & an atom. \\

{\tt IsSimple(tag)} &	
a term which is uniquely defined by its tag and value and so a comparison with
another term has just to check these two. Other terms, namely compound
terms and strings need to compare more than just the tag and value.\\

{\tt IsCompound(tag)} & a compound term (i.e.\ a structure or a list). \\

{\tt IsNumber(tag)} &	any nof the numeric types. \\

{\tt SameType(t1, t2)} & true if the arguments have the same type. \\
\hline
\end{tabular}

%\newpage

\section{Type requirements}
The following macros test the type of Prolog words and return
with an error if the type is not correct.
They must not be used inside nested functions, unless their
return code is correctly passed back to the system.

\noindent
 \\
\begin{tabular}{|ll|} 
\hline
{\tt Error_If_Ref(tag)} & 
return with an instantiation fault if argument is a variable. \\

{\tt Check_Ref(tag)} & 
return with an instantiation fault if argument is not a \\
& \ \ variable. \\

{\tt Check_List(tag)} & 
return with an instantiation fault if argument is a variable, \\
& \ \ otherwise with a type error if argument is not a list. \\

{\tt Check_Pair(tag)} & 
return with an instantiation fault if argument is a variable, \\
& \ \ otherwise with a type error if it is not a non NIL list. \\

{\tt Check_Structure(tag)} & 
return with an instantiation fault if argument is a variable, \\
& \ \ otherwise with a type error if it is not a structure. \\

{\tt Check_String(tag)} & 
return with an instantiation fault if argument is a variable, \\
& \ \ otherwise with a type error if it is not a string. \\

{\tt Check_Nil(tag)} & 
return with an instantiation fault if argument is a variable, \\
& \ \ otherwise with a type error if it is not a nil list. \\

{\tt Check_Integer(tag)} & 
return with an instantiation fault if argument is a variable, \\
& \ \ otherwise with a type error if it is not an integer. \\

{\tt Check_Float(tag)} & 
return with an instantiation fault if argument is a variable, \\
& \ \ otherwise with a type error if it is not a float. \\

{\tt Check_Atom(tag)} & 
return with an instantiation fault if argument is a variable, \\
& \ \ otherwise with a type error if it is not an atom. \\

{\tt Check_Atom_Or_Nil(val, tag)} & 
return with an instantiation fault if the prolog term \\
& \ \ represented by the two arguments is a variable, \\
& \ \ otherwise with a type error if it is not an atom or the \\
& \ \ null list. \\

{\tt Check_Output_List(tag)} & 
return with a type error if argument is not a list or a free \\
& \ \ variable. \\

{\tt Check_Output_Pair(tag)} & 
return with a type error if argument is not a non NIL list \\
& \ \ or a free variable. \\

{\tt Check_Output_Structure(tag)} & 
return with a type error if argument is not a structure or \\
& \ \ a free variable. \\

{\tt Check_Output_String(tag)} & 
return with a type error if argument is not a string or a \\
& \ \ free variable. \\

{\tt Check_Output_Nil(tag)} & 
return with a type error if argument is not a nil list or a \\
& \ \ free variable. \\

{\tt Check_Output_Integer(tag)} & 
return with a type error if argument is not an integer or \\
& \ \ a free variable. \\

{\tt Check_Output_Float(tag)} & 
return with a type error if argument is not a float or a \\
& \ \ free variable. \\

{\tt Check_Output_Atom(tag)} & 
return with a type error if argument is not an atom or a \\
& \ \ free variable. \\
\hline
\end{tabular}

%\newpage

\section{Unification}
The macros {\tt Return_Unify_} are used to unify two terms.
They cause an exit from the external predicate and so they must
not be used inside nested function calls.
If the unification of the two terms succeeds, so does the
external predicate, otherwise it fails.

\noindent
 \\
\begin{tabular}{|lp{7.3cm}|} 
\hline
{\tt Return_Unify_Integer(val, tag, int)} 
& unifies a general prolog term with an integer number. \\

{\tt Return_Unify_Float(val, tag, float)} 
& unifies a general prolog term with an float. \\

{\tt Return_Unify_Atom(val, tag, atom)} 
& unifies a general prolog term with an atom. \\

{\tt Return_Unify_Nil(val, tag)} 
& unifies a general prolog term with the null list. \\

{\tt Return_Unify_String(val, tag, ptr)} 
& unifies a general prolog term with a pointer to an \eclipse string. \\

{\tt Return_Unify_List(val, tag, listptr)} 
& unifies a general prolog term with a list. \\

{\tt Return_Unify_Structure(val, tag, strptr)} 
& unifies a general prolog term with a struc\-tu\-re. \\ 

{\tt Return_Unify_Pw(val1, tag1, val2, tag2)} 
& unify two general Prolog terms. To be used if
the type of neither of the two terms is
known. \\
\hline
\end{tabular}

\vspace*{0.5cm}

The macros {\tt Request_Unify_} can be used to unify any number
of term pairs.
To be able to use it, the macro
{\tt Prepare_Requests} must appear inside the declaration
part of the external procedure,
and the exit from the external procedure must be made
through the use of the macro {\tt Return_Unify}.
After the return to the system, all the specified term pairs
are unified.
If all these unifications succeed, so does the external predicate,
otherwise it fails.

\noindent
 \\
\begin{tabular}{|p{8.3cm}p{7.0cm}|} 
\hline
{\tt Prepare_Requests} & must be used before any {\tt Request_Unify} macro. \\

{\tt Return_Unify} & return from the external and perform the
unifications requested by {\tt Request_Unify}. \\

{\tt Request_Unify_Integer(val, tag, int)} & request unification of a general prolog term with an integer. \\ 

{\tt Request_Unify_Float(val, tag, float)} & request unification of a general prolog term with a float. \\ 

{\tt Request_Unify_Atom(val, tag, atom)} & request unification of a general prolog term with an atom. \\

{\tt Request_Unify_Nil(val, tag)} & request unification of a general prolog term with a nil list. \\

{\tt Request_Unify_String(val, tag, string)} & request unification of a general prolog term with a string. \\

{\tt Request_Unify_List(val, tag, listptr)} & request unification of a general prolog term with a list. \\ 

{\tt Request_Unify_Structure(val, tag, strptr)} & request unification of a general prolog term with a structure. \\

{\tt Request_Unify_Pw(val1, tag1, val2, tag2)} & unify two general Prolog terms.  To be used if
the type of neither of the two terms is
known. \\ 
\hline
\end{tabular}

\vspace{0.5cm}

%Note that no space may be allocated from the global stack
%(i.e.\ no structures, lists or strings created) between two
%calls to a {\tt Request} macro.

\section{Prolog Term Construction}
Macros to create tagged Prolog words at a specified address, and to
allocate list and structure frames.

\noindent
 \\
\begin{tabular}{|p{7.3cm}p{8.0cm}|} 
\hline
{\tt Make_Atom(pw, did)} & create atom or functor at address pw\\
{\tt Make_Float(pw, float)} & create float at address pw\\
{\tt Make_Integer(pw, int)} & create integer at address pw\\
{\tt Make_Nil(pw)} & create nil at address pw\\
{\tt Make_List(pw, plist)} & create list reference at address pw\\
{\tt Make_Ref(pw, pdest)} & create reference at address pw\\
{\tt Make_String(pw, cstring)} & create string at address pw\\
{\tt Make_Struct(pw, pstruct)} & create structure reference at address pw\\
{\tt Make_Var(pw)} & create free variable at address pw\\
{\tt Push_List_Frame()} & allocate a list cell at address TG\\
{\tt Push_Struct_Frame(did)} & allocate a structure at address TG and 
initialise the functor field\\
{\tt Make_Stack_String(len, val, s)} &
allocates space for a string of length {\tt len} and sets {\tt val}
to the string value and {\tt s} to the first string character.
The {\it len+1} bytes of memory starting at {\tt s} are to be filled
with the string and a zero terminator.\\
{\tt Cstring_To_Prolog(cstring, val)} &
convert a C string to an equivalent Prolog string {\tt val}.\\
word32 {\tt Did(string, arity)} &
returns the DID for the functor with name {\tt string} and arity {\tt arity}.\\
\hline
\end{tabular}

\index{Did}
\index{Cstring_To_Prolog}
\index{Make_Stack_String}
\index{Push_Struct_Frame}
\index{Push_List_Frame}
\index{Make_Var}
\index{Make_Struct}
\index{Make_String}
\index{Make_Ref}
\index{Make_Nil}
\index{Make_Nil}
\index{Make_Integer}
\index{Make_Float}
\index{Make_Atom}

\vspace*{0.5cm}

\noindent
TG is the top of the global stack, on which strings, lists and structures
are pushed.
Note that \eclipse strings are not directly compatible with strings in C.
The value part of a string pword points to a pword which holds
the string length and which is followed by  a tag and the string proper.
Despite the explicit length field, the string is zero terminated
to have a certain degree of C compatibility. However, since Prolog
strings may contain the NUL character, Prolog strings might be truncated
when treated as C strings.

%\newpage

\section{Prolog Term Decomposition}
These are the macros to convert Prolog terms into C data types.

\noindent
 \\
\begin{tabular}{|p{7.7cm}p{7.5cm}|}
\hline
char * {\tt DidName(did)} &
gives a pointer to the name of the functor {\tt did} as a C string.\\

long {\tt DidArity(did)} & gives the arity of {\tt did}. \\

pword * {\tt DidString(did)} &
gives a pointer to the name of the functor {\tt did} as a Prolog string.\\

{\tt Get_Name(val, tag, cstring)} &
sets {\tt cstring} to the name of an atom or string. \\
\hline
\end{tabular}
\index{Get_Name}
\index{DidString}
\index{DidArity}
\index{DidName}

\noindent
\begin{tabular}{|p{7.7cm}p{7.5cm}|}
\hline
{\tt Get_Proc_Did(val, tag, did)} & 
if {\tt val} and {\tt tag} specify a compound term of the form
{\bf Name/Arity}, {\tt did} is assigned the DID of the
corresponding functor otherwise the corresponding
error type is raised. Arities greater than MAX_ARITY
yield a RANGE_ERROR.\\

{\tt Get_Functor_Did(val, tag, did)} & 
like {\tt Get_Proc_Did}, however there is no arity
arity restriction and apart from a term {\it Name/Arity}
it accepts as well an atom and returns its DID.\\

long {\tt StringLength(val)} & given the value of a string pword, it returns the
length of the string, excluding the terminating zero.\\

char * {\tt StringStart(val)} & given the value of a string pword, it returns a
pointer to the first character of the string.\\

double {\tt Dbl(val)} & given the value double
precision Prolog float, return a C double.\\
\hline
\end{tabular}
\index{Dbl}
\index{StringStart}
\index{StringLength}
\index{Get_Functor_Did}
\index{Get_Proc_Did}

%\section{Arrays}
%\begin{tabular}{|p{7.3cm}p{7.9cm}|}
%\hline
%{\tt \mbox{Get_Visible_Array_Address(adid, vmod,} tmod, address)} &
%{\tt adid} is the DID of the array functor, {\tt address} must be
%a {\tt pword} pointer, {\tt vmod} and {\tt tmod} are value and tag
%of the module from where the array is looked up.
%{\tt address} is set to point to the
%beginning of the array or global variable. If
%no visible array exists, an exception is raised.\\
%
%{\tt Get_Array_Address(adid, address)} &
%The same as above, but we look for a global array rather than the visible one. 
%This is mainly provided for backward compatibility.\\
%
%{\tt \mbox{Get_Visible_Array_Header(adid, vmod, } tmod, address)} & 
%sets {\tt address} to point to the header of the visible array.
%An error is raised if no such array is visible.\\
%{\tt Get_Array_Header(adid, address)} & 
%The same as above, but we look for a global array rather than the visible one. 
%This is mainly provided for backward compatibility.\\
%\hline
%\end{tabular}
%\index{Get_Array_Header}
%\index{Get_Visible_Array_Header}
%\index{Get_Array_Address}
%\index{Get_Visible_Array_Address}
%
%\vspace*{0.3cm}
%
%The array layout is as follows:
%The header is a pword whose tag specifies the type of the array:
%\begin{itemize}
%\item TCOMP - Prolog type
%
%\item TINT - integer array
%
%\item TFLOAT - float array
%
%\item TSTRG - byte array
%\end{itemize}
%The value of the pword points to a block of (arity + 1) {\tt word32}'s,
%the first one is the DID of the array functor
%and the following ones are its dimensions.
%The following memory locations are occupied by the array itself.
%It is stored row-wise (as in C), i.e.\ array elements that differ by
%one in the last subscript are adjacent in memory.
%
%\section{Input / Output}
%\begin{tabular}{|p{6cm}p{9cm}|}
%\hline
%{\tt Fprintf(strm, format, args)} &
%prints out on the stream {\tt strm} the expression specified by
%{\tt format} and {\tt args}. {\tt format} is the control string and
%{\tt args} are the arguments, (as in {\tt printf}). \\
%
%{\tt Write(val, tag, strm)} &
%prints out on the stream {\tt strm} the prolog term specified by
%{\tt val} and {\tt tag}. \\
%
%{\tt Writeq(val, tag, strm)} &
%prints on the stream {\tt strm} the prolog term specified by 
%{\tt val} and {\tt tag}. Strings and atoms are quoted if necessary.\\ 
%\hline
%\end{tabular}

\section{Backtracking}
\begin{tabular}{|p{6cm}p{9cm}|}
\hline
{\tt Remember(n, val, tag)} & 
remembers the value which will be given to the {\tt n}-th argument of the
external predicate when an attempt is made to resatisfy the procedure. \\
{\tt Cut_External} &
discards all following alternative solutions. Without the use of this macro
the predicate never fails.  \\
\hline
\end{tabular}

%\section{Suspension}
%\begin{tabular}{|p{6.7cm}p{8.3cm}|}
%\hline
%{\tt Mark_Suspending_Variable_Inst(var)} & 
%{\tt var} is a pointer to the variable which is responsible for delaying
%a call, it becomes its {\it suspending variable}.This macro is to be used
%both in external predicates that occur as conditions in delay clauses
%and in external predicates which delay themselves.
%The suspended goal will be woken as soon as the variable is instantiated.\\ 
%
%{\tt Mark_Suspending_Variable(var)} & 
%{\tt var} is a pointer to the variable which is responsible for delaying
%a call, it becomes its {\it suspending variable}.This macro is to be used
%both in external predicates that occur as conditions in delay clauses
%and in external predicates which delay themselves.
%The suspended goal will be woken as soon as the variable is instantiated
%or bound to another suspending variable.\\ 
%
%{\tt Delay }& 
%returns to the Prolog system and suspends the external predicate.
%Some variables must have been marked by
%{\tt Mark_Suspending_Variable}, otherwise the effect is unpredictable. \\
%\hline
%\end{tabular}
%\index{Delay}
%\index{Mark_Suspending_Variable}
%\index{Mark_Suspending_Variable_Inst}
%
%\newpage
%
%\section{Calling Prolog Procedures}
%\begin{tabular}{|p{6.8cm}p{8cm}|}
%\hline
%{\tt \mbox{Prolog_Call(vgoal, tgoal, vmodule,} tmodule)} &
%    call the goal in the module and keep all bindings, global stack and trail
%    state.\\
%{\tt Prolog_Call_Nobind(vgoal,\hfill tgoal,  vmodule, tmodule)} &
%    call the goal in the module, but report only success or failure.
%    In this case no bindings or compound terms created by this
%    goal are kept.\\
%\hline
%\end{tabular}

