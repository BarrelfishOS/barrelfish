% BEGIN LICENSE BLOCK
% Version: CMPL 1.1
%
% The contents of this file are subject to the Cisco-style Mozilla Public
% License Version 1.1 (the "License"); you may not use this file except
% in compliance with the License.  You may obtain a copy of the License
% at www.eclipse-clp.org/license.
% 
% Software distributed under the License is distributed on an "AS IS"
% basis, WITHOUT WARRANTY OF ANY KIND, either express or implied.  See
% the License for the specific language governing rights and limitations
% under the License. 
% 
% The Original Code is  The ECLiPSe Constraint Logic Programming System. 
% The Initial Developer of the Original Code is  Cisco Systems, Inc. 
% Portions created by the Initial Developer are
% Copyright (C) 2006 Cisco Systems, Inc.  All Rights Reserved.
% 
% Contributor(s): Joachim Schimpf, IC-Parc
% 
% END LICENSE BLOCK

\documentstyle[epsf,11pt]{article}
%\epsfbox{file.eps}
\def\eclipse{ECLiPSe\ }
\def\eclipsetwo{ECLiPSe\ II\ }
\def\cplex{CPLEX\ }
\def\lp{linear programming\ }

%------------------------------------------------------------------------------
\title{\eclipse Message Passing Library Specification}
%------------------------------------------------------------------------------
\author{Joachim Schimpf}

\begin{document}
\maketitle

%------------------------------------------------------------------------------
\section{Introduction}
%------------------------------------------------------------------------------

A number of users to date have implemented message-passing-based
applications on top of \eclipse's socket primitives (e.g.
\cite{pitt}, \cite{vass}, Venice Lagoon Application). 
The aim of this paper is to suggest a higher-level library
that can in the future be used to implement this kind of applications
more conveniently.

Another point worth noting is that the parallel \eclipse system
contains a higher-level message passing C-library, which implements
most desirable features. It is used as an implementation layer of
the parallel system, but is not accessible from the \eclipse
language level.

%------------------------------------------------------------------------------
\section{Architecture}
%------------------------------------------------------------------------------

We envisage an architecture of communicating {\em agents}, not enforcing
a distinction between {\em clients} and {\em servers}.
Agents providing particular services can be found by name.
A {\em naming service} is provided that maps names to agent addresses.

A higher level concept of service {\em brokers} a la CORBA \cite{corba}
can be implemented on top of that. Preferable the full specification
of this functionality layer is left to \eclipsetwo.

Messages are sent to {\em logical ports}.
Logical ports can be created dynamically.
Communication is basically asynchronous,
synchronous facilities (e.g. remote procedure call style) can be built on top.

Messages are general \eclipse terms. Its is essential that they can contain
port identifiers in order to be able to built higher level communication
protocols..

%------------------------------------------------------------------------------
\section{Communication Protocols}
%------------------------------------------------------------------------------

An agent that wants to provide a service does the following:
\begin{itemize}
\item Create a port that will receive service requests.
\item Register the port under a specific name with the name server.
\item Attach a notification procedure to the port, or alternatively
enter a busy loop for polling the port.
\end{itemize}
A client wanting a service
\begin{itemize}
\item Looks up an appropriate port in the name server.
\item Sends a service request message to that port, including the
identifier of the port to reply to.
\item Waits for the reply or continues working until notified of the reply.
\end{itemize}

If a longer conversation between client and server is necessary (more than
a single message pair exchange), both client and server are expected to
create a port dedicated to this particular conversation.
This has the advantage that there are no ambiguities about origin, order,
priority and purpose of incoming messages on those ports.
In particular, a priority scheme is unnecessary, since dedicated ports
can be provided for different priority (e.g.\ urgent) messages.


%------------------------------------------------------------------------------
\section{Language Level Interface}
%------------------------------------------------------------------------------

\subsection{Creating Communication Ports}
A port is a communication endpoint.
Its main components are a queue for incoming messages,
a triggering mechanism for the notification of the application program.
\begin{description}
\item [port\_allocate(-PortId)] Create a local port on which
messages can be received.
\item [port\_deallocate(+PortId)] Remove a local port.
\item [port\_set\_option(+PortId, +Option)] Configure a local port, e.g.\ 
\begin{itemize}
\item attach/detach notification procedure
\item enable/disable notification
\item attach application data to the port
\end{itemize}
\item [port\_get\_option(+PortId, +Option)]
Retrieve properties of a local port.
\end{description}

\subsection{Naming Service}
The naming service is a registry that maps names to port identifiers.
\begin{description}
\item [nsrv\_register(+Name, +PortId)] Register PortId under Name.
\item [nsrv\_lookup(+Name, -PortId)] Find an entry.
\item [nsrv\_deregister(+Name)] Deregister an entry.
\end{description}

\subsection{Messages}
Messages are general \eclipse terms and are sent to ports, normally
remote ports.
\begin{description}
\item [msg\_send(+PortId, +Message)] Send a message to a (remote) port.
\item [msg\_receive(+PortId, -Message)] Receive a message on a local port.
\item [msg\_pending(+PortId)] Check for presence of received messages.
\end{description}

\subsection{Notification}
User-defined notification procedures are called when new messages
arrive on a port:
\begin{description}
\item [port\_handler(+PortId)] The port PortId has one or more new messages
\item [...] There may be other events that need to notify the application.
\end{description}
The system makes sure that there is no nested notification on the same port,
i.e.\ no notification while a notification procedure is executing.


%------------------------------------------------------------------------------
\section{Implementation}
%------------------------------------------------------------------------------

Two implementation approaches seem reasonable.
One is to extend the approach followed in \cite{pitt}, \cite{vass} and
implement higher-level notions like brokers, name servers and logical ports
on top of \eclipse socket communication primitives.
This has the advantage that most of the code can be written in
\eclipse rather than C. This means more compact code, but it may
complicate it subsequent transfer into \eclipsetwo.

The other way is to build on top of the PDS library (\cite{kees},
about 23000 lines of C).
This library already provides the notions of nameserver, logical ports
and typed messages.
Identical functionality is implemented for communication across
wide area networks, local area networks and shared memory domains,
presenting the application writer with a unified abstract view of
communication.
The library is thoroughly tested since it supplies all the
communication services used in the parallel \eclipse system.
What is needed is lifting the application programmer interface to the \eclipse
language layer.

For the transfer of general \eclipse terms across a heterogeneous
network of machines, a machine-independent data format is necessary.
\eclipse includes conversion facilities that can be used for this purpose.
They are currently used when \eclipse terms are stored and retrieved from
databases.

%------------------------------------------------------------------------------
\section{Stateful Servers}
%------------------------------------------------------------------------------

In many real-life applications, it is not possible, or simply not
efficient enough to rely completely on a model where every service
request from a client to a server is independent and self-contained.
Often it may be necessary to set up a significant amount of state
before a server can satisfy requests. For example, a mathematical
programming server would set up its simplex matrix once, but would
repeatedly be asked to re-solve the problem with minor modifications.

This notion of stateful servers can be implemented by either spawning a server
copy with its own (physical) communication ports.
Otherwise by creating just a new logical port, associated with some
conversation-specific data, which is served
by the original server in an interrupt-driven  or multi-threaded way.
The message passing library need not address this problem in more
detail, apart from providing logical ports that can be dynamically created
and destroyed.

%------------------------------------------------------------------------------
\section{Conclusion}
%------------------------------------------------------------------------------

For \eclipsetwo, the PDS library \cite{kees} provides the more thorough and
complete implementation basis. Integration into the system should be easy since \eclipsetwo
will allow smoother integration of C libraries than the current \eclipse does.

For exploring the appropriateness of the application programmer interface,
the current \eclipse system with a slightly extended library \cite{vass}
is a good enough vehicle.

%------------------------------------------------------------------------------
\newpage
\begin{thebibliography}{123456789ABCDEF}

\bibitem{corba}
OMG\\
{\em The Common Object Request Broker: Architecture and Specification}\\
OMG Document Number 91.12.1, Wiley \& Sons, 1992.

\bibitem{vass}
V. Liatsos\\
{\em Message Passing in \eclipse}\\
IC-Parc Technical Note, November 1995

\bibitem{pitt}
J. Pitt and J. Cunningham\\
{\em Blending the Balkans: One Approach to Integrating and Combining
Inference Engines}\\
ESPRIT Project MEDLAR II, Deliverable DIV.2.2-6P

\bibitem{kees}
K. Schuerman\\
{\em AMSG Programmer's Manual, Version 0.1},\\
{\em BMSG Programmer's Manual, Version 0.1},\\
{\em NSRV Programmer's Manual, Version 0.1},\\
ECRC Internal Reports, 1995

\end{thebibliography}

\end{document}
