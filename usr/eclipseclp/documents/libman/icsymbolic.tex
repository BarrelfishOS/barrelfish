% BEGIN LICENSE BLOCK
% Version: CMPL 1.1
%
% The contents of this file are subject to the Cisco-style Mozilla Public
% License Version 1.1 (the "License"); you may not use this file except
% in compliance with the License.  You may obtain a copy of the License
% at www.eclipse-clp.org/license.
% 
% Software distributed under the License is distributed on an "AS IS"
% basis, WITHOUT WARRANTY OF ANY KIND, either express or implied.  See
% the License for the specific language governing rights and limitations
% under the License. 
% 
% The Original Code is  The ECLiPSe Constraint Logic Programming System. 
% The Initial Developer of the Original Code is  Cisco Systems, Inc. 
% Portions created by the Initial Developer are
% Copyright (C) 2006 Cisco Systems, Inc.  All Rights Reserved.
% 
% Contributor(s): 
% 
% END LICENSE BLOCK

%----------------------------------------------------------------------
\chapter{The Symbolic Domain Library}
%HEVEA\cutdef[1]{section}
\label{chapicsymbolic}
\index{library!ic_symbolic|(}
%----------------------------------------------------------------------


%----------------------------------------------------------------------
%\section{Introduction}
%----------------------------------------------------------------------

The {\em ic_symbolic} library is a solver for constraints over ordered
symbolic domains.
It is implemented on top of library(ic) (see \ref{chapic}),
by mapping symbolic domains to finite integer domains.
There are also several mixed-domain constraints, which have both
symbolic and integer arguments.

%----------------------------------------------------------------------
\section{Domains and Domain Variables}
%----------------------------------------------------------------------

This library uses the
\biptxtref{domain}{domain/1}{../bips/kernel/termcomp/domain-1.html}
feature provided by the ECLiPSe kernel. 
This means that domains need to be declared.
The declaration specifies the domain values and their order. For example:
\begin{quote}\begin{verbatim}
?- local domain(weekday(mo,tu,we,th,fr,sa,su)).
\end{verbatim}\end{quote}
declares a domain with name 'weekday' and values 'mo', 'tu' etc.  The
domain values are implicitly ordered, with 'mo' corresponding to 1,
until 'su' corresponding to 7.  Domain values must be unique within
one ECLiPSe module, i.e. a symbolic value can belong to at most one
domain.

A variable of a declared domain can then be created using
\begin{quote}\begin{verbatim}
?- X &:: weekday.
X = X{[mo, tu, we, th, fr, sa, su]}
Yes (0.00s cpu)
\end{verbatim}\end{quote}
or multiple variables using
\begin{quote}\begin{verbatim}
?- [X,Y,Z] &:: weekday.
X = X{[mo, tu, we, th, fr, sa, su]}
Y = Y{[mo, tu, we, th, fr, sa, su]}
Z = Z{[mo, tu, we, th, fr, sa, su]}
Yes (0.00s cpu)
\end{verbatim}\end{quote}


%----------------------------------------------------------------------
\section{Basic Constraints}
%----------------------------------------------------------------------
The following constraints implement the basic relationships between
two domain values.  The constraints require their arguments to come
from identical domains, otherwise an error is raised.
\begin{description}
\item[\biptxtrefni{?X \&= ?Y}{\&=/2!ic_symbolic}{../bips/lib/ic_symbolic/YE-2.html}]
\index{\&=/2@\texttt{\&=/2}!ic_symbolic}
    X is the same as Y
\item[\biprefnoidx{?X \&\bsl= ?Y}{../bips/lib/ic_symbolic/YRE-2.html}]
    \index{\&\bsl=/2!ic_symbolic}
    X is different from Y
\item[\biptxtrefni{?X \&$<$ ?Y}{\&</2!ic_symbolic}{../bips/lib/ic_symbolic/YL-2.html}]
\index{\&</2@\texttt{\&</2}!ic_symbolic}
    X is strictly before Y in the domain order
\item[\biptxtrefni{?X \&$>$ ?Y}{\&>/2!ic_symbolic}{../bips/lib/ic_symbolic/YG-2.html}]
    \index{\&>/2@\texttt{\&>/2}!ic_symbolic} X is strictly after Y in the domain order
\item[\biptxtrefni{?X \&=$<$ ?Y}{\&=</2!ic_symbolic}{../bips/lib/ic_symbolic/YEL-2.html}]
\index{\&=</2@\texttt{\&=</2}!ic_symbolic}
    X is the same as Y, or before Y in the domain order
\item[\biptxtrefni{?X \&$>$= ?Y}{\&>=/2!ic_symbolic}{../bips/lib/ic_symbolic/YGE-2.html}]
\index{\&>=/2@\texttt{\&>=/2}!ic_symbolic}
    X is the same as Y, or after Y in the domain order
\item[\biptxtrefni{shift(?X,?C,?Y)}{shift/1!ic_symbolic}{../bips/lib/ic_symbolic/shift-3.html}]
\index{shift/2@\texttt{shift/2}!ic_symbolic}
    Y is C places above X in the domain order.
    X and Y have symbolic domains, C has an integer domain.
\item[\biptxtrefni{rotate(?X,?C,?Y)}{rotate/3!ic_symbolic}{../bips/lib/ic_symbolic/rotate-3.html}]
\index{rotate/3@\texttt{rotate/3}!ic_symbolic}
    like shift/3 but wraps at domain boundary.
\item[\biptxtrefni{element(?Index,++List,?Value)}{element/3!ic_symbolic}{../bips/lib/ic_symbolic/element-3.html}]
\index{element/3@\texttt{element/3}!ic_symbolic}
    Value occurs in List at position Index.
    Value has a symbolic domain, Index has an integer domain.
    List is a number of symbolic domain values.
\end{description}
For example
\begin{quote}\begin{verbatim}
?- [X, Y] &:: weekday, X &< Y.
X = X{[mo, tu, we, th, fr, sa]}
Y = Y{[tu, we, th, fr, sa, su]}
Yes (0.00s cpu)

?- X &:: weekday, X &=< we.
X = X{[mo, tu, we]}
Yes (0.00s cpu)
\end{verbatim}\end{quote}
    

%----------------------------------------------------------------------
\section{Global Constraints}
%----------------------------------------------------------------------
A number of global constraints are available which directly correspond
(and are in fact implemented via) their counterparts in
lib(ic_global):
\begin{description}
\item[\biptxtrefni{alldifferent(+List)}{alldifferent/1!ic_symbolic}{../bips/lib/ic_symbolic/alldifferent-1.html}]
\index{alldifferent@\texttt{alldifferent}!ic_symbolic}
    All list elements are different
\item[\biptxtrefni{occurrences(++Value,+List,?N)}{occurrences/3!ic_symbolic}{../bips/lib/ic_symbolic/occurrences-3.html}]
    Value occurs N times in List
\item[\biptxtrefni{atmost(++N,+List,++Value)}{atmost/3!ic_symbolic}{../bips/lib/ic_symbolic/atmost-3.html}]
\index{atmost/3@\texttt{atmost/3}!ic_symbolic}
    Value occurs at most N times in List
\end{description}


%----------------------------------------------------------------------
\section{Internals}
%----------------------------------------------------------------------

Internally, symbolic domains are mapped to integer ranges from 1 up to
the number of domain elements.  The first value in the domain
declaration corresponds to 1, the second to 2 and so on.  Similarly,
symbolic domain variables can be mapped to a corresponding IC integer
variable.  This mapping is accessible through the predicate
\bipref{symbol_domain_index/3}{../bips/lib/ic_symbolic/symbol_domain_index-3.html}:
\begin{quote}\begin{verbatim}
?- symbol_domain_index(fr, D, I).
D = weekday
I = 5
Yes (0.00s cpu)

?- X &:: weekday, symbol_domain_index(X, D, I).
X = X{[mo, tu, we, th, fr, sa, su]}
D = weekday
I = I{1 .. 7}
Yes (0.00s cpu)

?- X &:: weekday, X &\= we, symbol_domain_index(X, D, I).
X = X{[mo, tu, th, fr, sa, su]}
D = weekday
I = I{[1, 2, 4 .. 7]}
Yes (0.00s cpu)
\end{verbatim}\end{quote}
    
The integer variable I mirrors the domain of the symbolic variable X and vice versa.

%----------------------------------------------------------------------
\section{Extending and Interfacing this Library}
%----------------------------------------------------------------------

Because of the mapping of symbols to integers, new constraints over
symbolic variables can be implemented simply by placing numeric (IC)
constraints on the corresponding integer variables.

Similarly, the facilities of the ic_search library can be exploited
when working with symbolic variables.  Instead of labeling the
symbolic variables, one can use the various facilities of ic_search to
label the corresponding integer variables instead.


%----------------------------------------------------------------------
\index{library!ic_symbolic|)}
%----------------------------------------------------------------------

%HEVEA\cutend
