% BEGIN LICENSE BLOCK
% Version: CMPL 1.1
%
% The contents of this file are subject to the Cisco-style Mozilla Public
% License Version 1.1 (the "License"); you may not use this file except
% in compliance with the License.  You may obtain a copy of the License
% at www.eclipse-clp.org/license.
% 
% Software distributed under the License is distributed on an "AS IS"
% basis, WITHOUT WARRANTY OF ANY KIND, either express or implied.  See
% the License for the specific language governing rights and limitations
% under the License. 
% 
% The Original Code is  The ECLiPSe Constraint Logic Programming System. 
% The Initial Developer of the Original Code is  Cisco Systems, Inc. 
% Portions created by the Initial Developer are
% Copyright (C) 2006 Cisco Systems, Inc.  All Rights Reserved.
% 
% Contributor(s): 
% 
% END LICENSE BLOCK

\chapter{Introduction}
%HEVEA\cutdef[1]{section}

This manual documents the major \eclipse\ libraries.
They are enabling tools for the development and delivery
of planning and scheduling applications.
Since this is an area of active research and new developments,
these libraries are subject to technical improvements, addition
of new features and redesign as part of our ongoing work.

In this section we shall briefly summarize the constraint solvers that
are available as \eclipse\ libraries.
No examples are given here - each solver has its own documentation
with examples for the interested reader.

\section{Suspended Goals: {\em suspend}}
The constraint solvers of \eclipse\ are all implemented using suspended
goals.
In fact the simplest implementation of any constraint is to suspend it
until all its variables are sufficiently instantiated, and then test it.

The library {\em suspend} contains versions of 
the mathematical constraints \verb0>=0, \verb0>0,
\verb0=:=0, \verb0=\=0, \verb0=<0, \verb0<0
which behave like this\footnote{
Note that the global flag {\em coroutine} has a similar effect:
it causes the arithmetic comparisons as well as many other
built-in predicates to delay until they are sufficiently instantiated}.

\section{Finite Domains: {\em ic}}
\subsection{{\em Integer Domain}}
The standard constraint solver offered by most constraint programming
systems is the {\em finite domain} solver, which applies constraint
propagation techniques developed in the AI community
\cite{VanHentenryck}.  
\eclipse\  supports finite domain constraints via the {\em ic}
library\footnote{There is also an older implementation, the {\em fd} library,
whose use is deprecated}.
This library implements finite domains of integers, and the usual
functions and constraints on variables over these domains.

\subsection{Symbolic Domain: {\em ic_symbolic}}
In addition to integer domains, \eclipse\ offers finite domains of
ordered non-numeric values, for example ${red, green, blue}$.
These are implemented by the {\em ic_symbolic} library.

Whilst there is a standard set of constraints supported by the 
{\em ic} library in \eclipse\  and in 
most constraint programming systems, many more finite domain
constraints have been introduced which have uses in specific
applications and do not belong in a generic constraint programming
library.
The behaviour of these constraints is to prune the finite domains of
their variables, in just the same way as the standard
constraints.
Therefore \eclipse\  offers several further libraries which implement more
constraints using the {\em ic} library. 

\subsection{Global Constraints: {\em ic\_global}}
One such library is {\em ic\_global}.
It supports a variety of constraints, each of which takes as an argument
a list of finite domain variables, of unspecified length.
Such constraints are called ``global'' constraints  \cite{beldiceanu}.
Examples of such constraints, available from the {ic\_global} library
are
\verb0alldifferent/10, \verb0maxlist/20, \verb0occurrences/30 and
\verb0sorted/20.

\subsection{Scheduling Constraints}
There are several \eclipse\  libraries implementing global constraints for
scheduling applications.  The constraints have the same semantics,
but different propagation.  The constraints take a list
of tasks (start times, durations and resource needs), and a maximum
resource level. They reduce the finite domains of the task start times
by reasoning on resource bottlenecks \cite{lepape}.  Three \eclipse\  libraries
implementing scheduling constraints are
{\em cumulative}, {\em edge\_finder} and {\em edge\_finder3}.

\section{Sets}
\eclipse\  offers constraint solving over the domain of finite sets of
integers. The {\em ic\_sets} library works together with the {\em ic} library
to reason about sets and set cardinality \cite{gervet}\footnote{
There is also an older implementation, the {\em conjunto} library, which
is generally less efficient, but implements sets of symbolic elements as
well as integer sets}.

\section{Intervals}
Besides finite domains, \eclipse\  also offers continuous domains in the
form of numeric intervals.
These are also implemented by the {\em ic} library, which is an integration
of an 
integer finite domain solver and interval reasoning over continuous
intervals\footnote{
The {\em ic} library replaces the old {\em ria} interval solver, and
covers most of the functionality of the finite domain solver {\em fd}}.
It solves equations and inequations between 
general arithmetic expressions over continuous or integral variables.
The expressions can include non-linear functions such as $sin$, built-in
constants such as $pi$. Piecewise linear unary functions are also available.

In addition to constraints, {\em ic} offers search techniques 
({\em splitting} \cite{VanHentenryck:95} and {\em squashing} 
\cite{lhomme96boosting})
for solving problems involving continuous numeric variables.


\section{User-Defined Constraints}
\subsection{Generalised Propagation: {\em propia}}
The predicate {\em infers} takes as one argument
any user-defined predicate, and as a second argument a form of
propagation to be applied to that predicate.

This functionality enables the user to turn any predicate into a
constraint \cite{LeProvost93b}. The forms of propagation include finite
domains and intervals.

\subsection{Constraint Handling Rules}
The user can also specify predicates using rules with guards
\cite{Fruehwirth}.  
They delay until the guard is entailed or disentailed, and then
execute or terminate accordingly. 

This functionality enables the user to implement constraints in a way
that is clearer than directly using the underlying {\em suspend}
library.

\section{Repair}
The {\em repair} library allows a {\em tentative} value to be
associated with any variable \cite{cp99wkshoptalk}.
This tentative value may violate constraints on the variable, in which
case the constraint is recorded in a list of violated constraints.
The repair library also supports propagation {\em invariants}
\cite{Localizer}.
Using invariants,  if a variable's tentative
value is changed, the consequences of this change can be propagated to
any variables whose tentative values depend on the changed one.
The use of tentative values in search is illustrated in the \eclipse\ 
``Tutorial on Search Methods''.
 
\section{Linear Constraints}
There are two libraries supporting linear constraint solving.  The
first {\em eplex} provides an interface to external linear
programming packages.  
It offers flexibility and scalability, but may
require a license for the external software.
The second {\em clpqr} can support infinite precision, but is less
efficient and scalable and offers fewer facilities.

\subsection{External Linear Solvers: {\em eplex}}
{\em eplex} supports a tight integration \cite{Bockmayr} between
external linear solvers (CPLEX \cite{ILOG} and XPRESS \cite{Dash})
and \eclipse. 
Constraints as well as variables can appear in both the external
linear solver and other \eclipse\  solvers.
Variable bounds are automatically passed from the \eclipse\  {\em range}
solver to the external solver.
Optimal solutions and other solutions can be returned to \eclipse\  as
required.
Search can be carried out either in \eclipse\  or in the external solver.

\subsection{{\em clpqr}}
The {\em clpqr} library offers two implementations of the Simplex
method for solving linear constraints \cite{Holzbauer}.  
One version uses rationals and
is exact.  The other version uses floats.
This library employs public domain software, and can be used for small
problems (with less than 100 variables).

\subsection{Piecewise Linear: {\em eplex\_relax}}
This library handles any user-defined piecewise linear function as a
constraint closely integrated with {\em eplex}.  It offers better
pruning than the standard handling of piecewise linear constraints
in the external solvers \cite{Ajili}.

%\section{Combining Linear and Finite Domain Propagation}
%\subsection{{\em fdplex}}
%A simple way to achieve maximum propagation is to send all numeric
%constraints both to {\em fd} and to {\em eplex} \cite{RWH99}.
%This requirement is automatically supported by the {\em fdplex}
%library.

\subsection{Probing for Scheduling}
For scheduling applications where the cost is dependent on each start
time, a combination of solvers can be very powerful.
For example, we can use finite domain
propagation to reason on 
resources and linear constraint solving to reason on cost \cite{HaniProbe}.
 
The {\em probing\_for\_scheduling} library supports such a combination,
via a similar user interface to the {\em cumulative} constraint mentioned
above.


\section{Other Libraries}
The solvers described above are just a few of the many libraries
available in ECLiPSe and listed in the \eclipse\  library directory.

Libraries are not only for constraint solvers -- for example, the
{\em \eclipse\ SQL Database Interface} library provides an interface to
external Database Management Systems, allowing users to add and retrieve data
from the database within an \eclipse\ program.

Any \eclipse\  user who has implemented a constraint solver is welcome to
send the code to the \eclipse\  team so that it can be added to
the available libraries.
Comments and suggestions on the existing libraries are also welcome!

%HEVEA\cutend
