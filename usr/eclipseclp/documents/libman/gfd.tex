% BEGIN LICENSE BLOCK
% Version: CMPL 1.1
%
% The contents of this file are subject to the Cisco-style Mozilla Public
% License Version 1.1 (the "License"); you may not use this file except
% in compliance with the License.  You may obtain a copy of the License
% at www.eclipse-clp.org/license.
% 
% Software distributed under the License is distributed on an "AS IS"
% basis, WITHOUT WARRANTY OF ANY KIND, either express or implied.  See
% the License for the specific language governing rights and limitations
% under the License. 
% 
% The Original Code is  The ECLiPSe Constraint Logic Programming System. 
% The Initial Developer of the Original Code is  Cisco Systems, Inc. 
% Portions created by the Initial Developer are
% Copyright (C) 2012 Cisco Systems, Inc.  All Rights Reserved.
% 
% Contributor(s): Kish Shen
% 
% END LICENSE BLOCK

%HEVEA\cutdef[1]{section}

\section{Introduction}

   The GFD library is an interface for {\eclipse} to Gecode's finite domain constraint
   solver. Gecode ({\tt www.gecode.org}) is an open-source toolkit for 
   developing
   constraint-based systems in C++, and includes an  integer 
   finite domain constraint solver.

   This interface provides a high degree of code compatibility with the finite 
   domain portion of the IC library, and to a lesser extent, with the FD
   library as well. This means that programs originally written for the
   IC library should run with GFD with little modifications, beyond 
   renaming any explicit reference to the ic family of modules. For example,
   here is a GFD program for N-Queens:

\begin{quote}
\begin{verbatim}
:- lib(gfd).

queens_list(N, Board) :-
    length(Board, N),
    Board :: 1..N,
    (fromto(Board, [Q1|Cols], Cols, []) do
        ( foreach(Q2, Cols), param(Q1), count(Dist,1,_) do
            Q2 #\= Q1,
            Q2 - Q1 #\= Dist,
            Q1 - Q2 #\= Dist
        )
    ),
    labeling(Board).  
\end{verbatim}
\end{quote}

This version of the program is from an example IC version of N-Queens,
with just \verb':- lib(ic)' replaced by \verb':- lib(gfd)'. The
search is done in \eclipse, using GFD's \verb'labeling/1', which essentially
employs no heuristics in selecting the variable (input order) and choice of 
value to label the selected variable to (from minimum).



\section{Problem Modelling}

GFD provides facilities to model and solve problems over the 
finite integer domain, with Gecode as the solver. It supports the constraints
provided by Gecode -- and Gecode supports a large set of constraints. The 
search to solve the problem can be done at the 
{\eclipse} level (with support from Gecode for variable and value selections), 
or the whole search can be performed by Gecode itself using one of its 
search engines.   

Implementation-level differences (like Gecode's 
{\it re-computation based\/} model vs. \eclipse's {\it backtracking\/}
model) are largely invisible to the user,
as GFD automatically maintains the Gecode computational state 
to match \eclipse's.

\subsection{Usage}

To load the GFD library into your program, simply add the following directive
at an appropriate point in your code.

\begin{quote}
\begin{verbatim}
:- lib(gfd).
\end{verbatim}
\end{quote}

\subsection{Integer domain variables}

An (integer) domain variable is a variable which can be instantiated only to a
value from a given finite set of integer values. 

A variable becomes a domain variable when it first appears in a (GFD) 
constraint. If the constraint is a domain constraint, then the variable will
be given the domain specified by the constraint. Otherwise, the variable will
be given a default domain, which should be large enough for
most problem instances. 

The default domain is an interval, and the maximum and minimum values of this
interval can be changed using \bipref{gfd_set_default/2}{../bips/lib/gfd/gfd_set_default-2.html} (with the options 
{\tt interval_max} and {\tt interval_min} for the maximum and minimum values,
respectively).  You can also obtain the current values
of interval_max and interval_min using \bipref{gfd_get_default/2}{../bips/lib/gfd/gfd_get_default-2.html}.

The Gecode documentation suggests that domain variables should be given as
small a domain as possible, and requires the user to explicitly specify a domain for 
all domain variables. While this is not required by GFD, following Gecode's
convention is still a good idea, since overly large domains can negatively
affect performance.  It is therefore recommended to make use of domain
constraints, and specify the domain for variables before using them in other
constraints.

A domain variable is mapped into a Gecode \verb'IntVar'.

In GFD, as in IC, booleans (such as the boolean in reified constraints) are
normal domain variables with the domain \verb'[0,1]'. However, in Gecode, 
boolean domain variables are a separate class \verb'BoolVar'. Boolean
domain variables are implemented in GFD by linking the integer
domain variable representing the boolean to an internal 
\verb'BoolVar' in the Gecode solver state, and the user always access the 
boolean via the integer domain variable.


\subsection{Constraints}
GFD supports the (integer finite domain) constraints implemented by Gecode.
Some of these constraints correspond to those in \eclipse's native finite 
domain solvers (IC and FD), but many do not. For those that are 
implemented in IC and/or FD, the same name and syntax is used by GFD 
(although details may differ, such as the types allowed for arguments).  
See 
\begin{htmlonly}
\ahref{../bips/finite-domain_constraints.html}{Table of Finite domain Constraints}
\end{htmlonly}
\begin{latexonly}
the Table of Finite domain Constraints (included with the documentation
in \verb|<ECLiPSeDir>/doc/bips/finite-domain_constraints.html|) 
\end{latexonly}
for the constraints supported by \eclipse's finite domain solvers.
%One difference is that all these constraints are defined in the GFD
%library itself, and can thus be called without module qualification or
%loading additional libraries. 

\label{conlev}
In addition to propagating constraints at a unspecified or default level,
Gecode also support three propagaion consistency levels. GFD map
these consistency levels to four modules:
\begin{description}
\item[{\tt gfd_gac}] Domain consistent (Generalised Arc-Consistent), maps to {\tt ICL_DOM} in Gecode.
\item[{\tt gfd_bc}] Bound consistent, maps to {\tt ICL_BND} in Gecode.
\item[{\tt gfd_vc}] Value consistent, maps to {\tt ICL_VAL} in Gecode.
\item[{\tt gfd}] Default consistency level, maps to {\tt ICL_DEF} in Gecode.
\end{description}
Constraints can be posted unqalified for the default consistency level, or
explicitly qualified with the module name for the desired 
consistency level, e.g.
\begin{quote}
\begin{verbatim}
gfd_gac:alldifferent(Xs)
\end{verbatim}
\end{quote}
Alternatively, constraints can also be imported from one of the modules and 
then posted unqualified at the imported consistency level.

The default consistency level is supported for all constraints, but
posting a constraint at the other three consistency levels is supported only
if that consistency level is implemented for that constraint
in Gecode -- see the individual documentation for the constraints for details.
Note that the three consistency modules are implicitly created when GFD
is loaded, and do not need to be loaded explicitly. 

Even constraints that involve expressions may be posted at specific 
consistency levels. In fact, a consistency level can be specified for
any sub-expression within the expression. However, it is possible that some of
the sub-constraints and sub-expressions inside the expressions are not 
supported at the given consistency level. In such cases, these sub-expressions 
will be posted at the default consistency level. Note that any 
sub-expressions with its own specified consistency level will also be 
factored out and posted separately. 

For example, the N-Queens example
can post domain consistent versions of the \verb'#\=/2' constraints:
\begin{quote}
\begin{verbatim}
:- lib(gfd).

queens_list(N, Board) :-
    length(Board, N),
    Board :: 1..N,
    (fromto(Board, [Q1|Cols], Cols, []) do
        ( foreach(Q2, Cols), param(Q1), count(Dist,1,_) do
            gfd_gac: (Q2 #\= Q1),
            gfd_gac: (Q2 - Q1 #\= Dist),
            gfd_gac: (Q1 - Q2 #\= Dist)
        )
    ),
    labeling(Board).
\end{verbatim}
\end{quote}
In this particular example, using the stronger propagation actually results in
a reduction in performance, as there is no reduction in search space from the
stronger propagation, but an increase in the cost of doing the propagation, 

Gecode maintains a solver state representing the problem, with the constraints
and problem variables, and performs constraint propagation within this solver
state. GFD adds the constraints and problem variables to this state
as they are posted in the user program. Unlike
the problem variables, there are no \eclipse level representation of the
constraints. One consequence is that floundering -- active constraints 
remaining at the end of computation -- can only be determined from the solver
state. GFD provides \bipref{solver_constraints_number/1}
{../bips/lib/gfd/solver_constraints_number-1.html},  
and \bipref{solver_vars_number/1}
{../bips/lib/gfd/solver_vars_number-1.html} to obtain the current number of
active constraints and problem variables, respectiviely, in the current solver
state.

Unlike \eclipse's native solvers, constraint propagation in Gecode must be 
triggered explicitly. In GFD, triggering propagation in the Gecode solver state
is implemented as a demon goal {\tt gfd_do_propagate/1} at scheduling 
priority 9.
When a constraint is posted, it is added to the solver state,
and the propagate demon goal is
scheduled for execution. It is thus possible to post multiple
 constraints without propagation by posting the constraints at a more
urgent (i.e. numerically smaller) priority than 9 (see
\bipref{call_priority/2}{../bips/kernel/suspensions/call_priority-2.html}).
This could reduce the cost of performing the propagation.

\paragraph{Constraint argument conventions}

GFD's constraints (and other predicates) follows several conventions 
for the format of its arguments. These are outlined in this section.

For arguments in constraints that are domain variables, the argument 
can be supplied as an
array element using array notion (e.g. \verb|Array[1]|).
Non-domain variables will be turned into domain variables with the default 
interval, and an integer can be given in place of a domain variable, 
representing a domain variable with a singleton domain. 

For arguments that are a group of variables or values, these are supplied
as a collection  
(as used in \bipref{collection_to_list/2}{../bips/lib/lists/collection_to_list-2.html}). In some cases, the elements in the collection are required to be 
ordered, and for such cases, the ordering is 
with respect to the flattened form of the collection, with indexing starting
at 1 for the first element. Array notation can again be used to supply a 
part of the array as a collection.

Indexing always start at 1 for \eclipse, 
which is different from Gecode, where index starts from 0.
GFD maps between the two index conventions in various ways,
depending on the constraint, with the aim of 
minimising the overhead. GFD also supports versions of these constraints that 
uses Gecode's native indices, i.e. starting from 0, and these have an 
additional {\tt _g} in their name (e.g. {\bf bin_packing_g/3} is the Gecode native
index version of {\bf bin_packing/3}). These versions of the constraint do not
have the overhead of converting the index value, but may be
incompatible with the rest of \eclipse.

In general, any collection argument can be supplied as a nested collection
(e.g. nested listed or multi-dimensional arrays). In some
cases, nested collections are required and supply information required
by the constraint, e.g.\ 
the dimension in multi-dimensional bin-packing predicates. Also,
some predicates' argument expect a two dimensional matrix, in which
case the argument must be supplied as a two dimensional collection 
(list of lists, two dimensional array). A two dimensional collection is
organised in the standard \eclipse way, i.e.\ row-major order, as a 
collection of rows, where each row having the same number of elements, and
each of these
element representing the column elements in that row. In array notation,
the row is addressed first, followed by the column, i.e.\  
\verb'[Row,Column]'. 

\subsubsection{Domain constraints}

The following domain constraints are supported by GFD:

\begin{description}
\item[\biptxtrefni{?Vars \#:: ++Domain}{\#::/2!gfd}{../bips/lib/gfd/HNN-2.html}]
Constrains Vars to have the domain Domain. A \biptxtrefni{reified}{\#::/3!gfd}{../bips/lib/gfd/HNN-3.html} version is also available.
\verb'::/2,3' are also supported as aliases.

\end{description}

  
\subsubsection{Arithmetic and logical expressions}

GFD supports expressions as arguments for relational and logical connective
constraints. Expressions can either evaluate to an integer (integer 
expression) or a truth value (constraint expression). Integer and 
constraint expressions maps to Gecode's {\it LinIntExpr\/} and {\it BoolExpr}, 
respectively. Unlike IC's expressions, user-defined constraints are not
allowed, but unrecognised ground terms in integer expressions are treated as
arithmetic functions and evaluated.

\begin{description}
\item[Relational Constraints]
These specify an arithmetic relationship between two integer expressions.
Constraint expressions are allowed as arguments of relational constraints,
with the truth value of the expression treated as the integer value 1 (true)
or 0 (false). 

All relational constraints have their reified counterparts, which has an extra
boolean argument that specify if the constraint is entailed or not. Expressions
in refied constraints are restricted to inlined (i.e.\ Gecode native) integer
expressions.

The  relational constraints are:
\biptxtrefni{\#</2,3 }{\#</2!gfd}{../bips/lib/gfd/HL-2.html} (less than),
\biptxtrefni{\#=/2,3}{\#=/2!gfd}{../bips/lib/gfd/HE-2.html} (equal), 
\biptxtrefni{\#=</2,3}{\#=</2!gfd}{../bips/lib/gfd/HEL-2.html} (less than or equal to),
\biptxtrefni{\#>/2,3}{\#>/2!gfd}{../bips/lib/gfd/HG-2.html} (greater than),
\biptxtrefni{\#>=/2,3}{\#>=/2!gfd}{../bips/lib/gfd/HGE-2.html} (greater than or equal to),
\biptxtrefni{\#\bsl=/2,3}{\#\\=/2!gfd}{../bips/lib/gfd/HRE-2.html} (not equal to).

\item[Logical Connective constraints]
Specifies a logical connection between constraint expression(s). 
. 
All logical connectives have their reified counterparts.

The available connectives are:

\biptxtrefni{<=>/2,3}{<=>/2!gfd}{../bips/lib/gfd/LEG-2.html} (equivalent),
\biptxtrefni{=>/2,3}{=>/2!gfd}{../bips/lib/gfd/EG-2.html} (implies),
\biptxtrefni{and/2,3}{and/2!gfd}{../bips/lib/gfd/and-2.html} (and),
\biptxtrefni{or/2,3}{or/2!gfd}{../bips/lib/gfd/or-2.html} (or),
\biptxtrefni{xor/2,3}{xor/2!gfd}{../bips/lib/gfd/xor-2.html} (exclusive or),
\biptxtrefni{neg/1,2}{neg/1!gfd}{../bips/lib/gfd/neg-1.html} (negation).

Boolean domain variables with domain \texttt{[0,1]} and constraints
which can be reified can occur as an argument of a logical connective, i.e.
as a constraint expression, evaluating to the reified truth value. 

For compatibility with IC, \texttt{\#=/2} can be used
instead of \texttt{<=>/2} and \texttt{\#\bsl=/2} can be used instead of
\texttt{xor}.

Gecode also supports {\em half reification\/} as well as the normal 
{\em full reification} for reified constraints. Half verification 
is when the propagation is in one direction only across the logical 
connective (i.e. \verb'=>'), rather than bi-directional (\verb'<=>'). 
In GFD, constraints can be half reified as follows:
\begin{verbatim}
Boolean => Constraint
Constraint => Boolean
\end{verbatim}
where Constraint is the half reified constraint, written without its boolean
argument \texttt{Boolean}. The two alternatives are for the two different
direction of propagation.

\end{description}

The syntax for the expressions closely follows that in IC. The 
following can be used inside integer expressions:

\begin{description}
\item[\texttt{X}]
            \emph{Variables}.  If \verb'X' is not yet a domain variable, it is turned 
            into one.

\item[\texttt{123}]
            Integer constants.

\item[\texttt{-Expr}]
            Sign change.

\item[\texttt{abs(Expr)}]
    The absolute value of Expr.

\item[\texttt{E1+E2}]
    Addition.

\item[\texttt{E1-E2}]
    Subtraction.

\item[\texttt{E1*E2}]
    Multiplication.

\item[\texttt{E1//E2}]
    Integer division, truncating towards zero.

\item[\texttt{E1/E2}]
    Division, defined only if E2 evenly divides E1 (non-inlined).

\item[\texttt{E1 rem E2}]
            Integer remainder, same sign as E1.

\item[\texttt{Expr}\textasciicircum{}{\texttt N}]
            N$^{th}$ power. N is a positive integer. .

\item[\texttt{min(E1,E2)}]
    Minimum.

\item[\texttt{max(E1,E2)}]
    Maximum.

\item[\texttt{sqr(Expr)}]
    Square. Logically equivalent to \verb|Expr*Expr|.

\item[\texttt{rsqr(Expr)}]
    Reverse of the \texttt{sqr} function. Differ from \texttt{sqrt} in that
    negative root is not excluded (non-inlined).

\item[\texttt{isqrt(Expr)}]
            Integer square root (always positive). Truncated towards zero.

\item[\texttt{sqrt(Expr)}]
            Square root, defined only if Expr is the square of an integer
	    (non-inlined).

\item[\texttt{inroot(Expr,N)}] 
    Integer N$^{th}$ root, N is a positive integer. Truncated to nearest smaller 
    integer. For even N, result is the non-negative root. 

\item[\texttt{rpow(Expr,N)}]
    Reverse of exponentiation, i.e.\ find \texttt{X} in \verb'E1 = X^N'..
    \texttt{N} is a positive integer. Differ from \texttt{inroot} in that
    the result is only defined for integral root, and negative root not
    excluded (non-inlined).

\item[\texttt{sum(ExprCol)}]
            Sum of a collection of expressions (ExprCol non-inlined).

\item[\texttt{sum(IntCol*ExprCol)}]
            Scalar product of a collection of integers and expressions.
            \verb'IntCol' and \verb'ExprCol' must be the same size
	    (ExprCol non-inlined).

\item[\texttt{min(ExprCol)}]
            Minimum of a collection of expressions (ExprCol non-inlined).

\item[\texttt{max(ExprCol)}]
            Maximum of a collection of expressions (ExprCol non-inlined).

\item[\texttt{element(ExprIdx, Col)}]
            Element constraint, Evaluate to the ExprIdx'th element of Col.
	    ExprIdx can be an integer expression.

\item[
    \texttt{\#>}, \texttt{\#>=}, \texttt{\#=}, \texttt{\#=<},
 \texttt{\#<},
    \texttt{\#\bsl=}]

    Posted as a constraint, both the left- and right- hand arguments are
    inlined expressions (non-inlined).

    Within the expression context, the constraint evaluates to its
    reified truth value.  If the constraint is entailed by the
    state of the constraint store then the (sub-)expression
    evaluates to \verb|1|.  If it is dis-entailed by the state of
    the constraint store then it evaluates to \verb|0|. If its
    reified status is unknown then it evaluates to a boolean domain
    variable \verb|0..1|.

\begin{sloppypar}
    Note: The simple cases (e.g.\ \verb|Bool #= (X #> 5)|) are
    equivalent to directly calling the reified forms of the basic
    constraints (e.g.\ \verb|#>(X, 5, Bool)|).
\end{sloppypar}

\item[\texttt{ConLev:Expr}]
    Post \texttt{Expr} at consistency level \texttt{ConLev}. Consistency
    levels are described in section~\ref{conlev}. 

\item[\texttt{eval(Expr)}]
            Logically equivalent to \verb'Expr'. Provided for compatiblity
	    with IC

\item[\texttt{Functional/reified constraints}]
            Reified constraints (whose last argument is a 0/1 variable)
            and functional constraints (whose last argument is an integer
            variable) can be written without their last argument within
            an expression context.  The expression then effectively
            evaluates to the value of the missing (unwritten) argument.
	    (non-inlined)

\end{description}
and for constraint expressions: 

\begin{description}
\item[\texttt{X}]
            \emph{Boolean Variables}.  Domain variables with domain 0/1. If cariable 
	    is not yet a domain variable, it is turned into one.

\item[\texttt{1}]
            truth value (0 for false, 1 for true).


\item[\texttt{E1 and E2}]
            Reified constraint conjunction.  e.g. \verb'X #> 3 and Y #< 8'.
	    E1 and E2 are constraint expressions.

\item[\texttt{E1 or E2}]
            Reified constraint disjunction.  e.g. \verb'X #> 3 or Y #< 8'.
	    E1 and E2 are constraint expressions.

\item[\texttt{E1 xor E2}]
            Reified constraint exclusive disjunction/non-equivalence.  e.g. \verb'X #> 3 xor Y #< 8'.
	    E1 and E2 are constraint expressions.

\item[\texttt{E1 => E2}]
            Reified constraint implication.  e.g. \verb'X #> 3 => Y #< 8'.
	    E1 and E2 are constraint expressions.

\item[\texttt{neg E}]
            Reified constraint negation.  e.g. \verb'neg X #> 3'
	    E is a constraint expressions.

\item[\texttt{E1 <=> E2}]
            Reified constraint equivalence.  e.g. \verb'X #> 3 <=> Y #< 8'.
            This is similar to {\texttt \#=} used in an expression context.
	    E1 and E2 are constraint expressions.

\item[\texttt{element(ExprIdx, BoolCol)}]
            Element constraint, Evaluate to the ExprIdx'th element of BoolCol.
	    ExprIdx can be an inlined integer expression. BoolCol is a
	    collection of boolean values or domain variable.

\item[
    \texttt{\#>}, \texttt{\#>=}, \texttt{\#=}, \texttt{\#=<},
 \texttt{\#<},
    \texttt{\#\bsl=}]

    Reified relational constraint, both the left- and right- hand arguments are
    inlined integer expressions.

\item[\texttt{ConLev:Expr}]
    Post \texttt{Expr} at consistency level \texttt{ConLev}. Consistency
    levels are described in section~\ref{conlev}. 

\item[\texttt{eval(Expr)}]
            Logically equivalent to \verb'Expr'. Provided for compatibility
	    with IC.

\item[\texttt{Reified constraints}]
            Reified constraints (whose last argument is a 0/1 variable)
            can be written without their last argument within
            an expression context.  The expression then effectively
            evaluates to the value of the missing (unwritten) argument.
	    (non-inlined)

\end{description}
 
The expressions allowed by GFD are a super-set of the expressions supported by 
Gecode. The components not supported by Gecode are indicated by `non-inlined'
in the above description. When an expression is posted, it is parsed and 
broken down into expressions and/or logical connectives supported by Gecode, 
along 
with any constraints). This is done to 
allow the user greater freedom in the code they write, and also to provide 
better compatibility with IC. 

Note that posting of complex expressions is relatively expensive: they are 
first parsed at the {\eclipse} level by GFD to extract the sub-expressions and 
any new domain variables, and these sub-expressions (in the form of 
{\eclipse} structures) are then parsed again at the GFD C++ level to convert
them to the appropriate Gecode data structures, which are then passed to
Gecode. Gecode itself will then convert these data structures
to the basic constraints that it supports. 

Unlike IC, GFD domain variables must have finite upper and lower 
bounds, and any arithmetic expression that evaluate to values outside
these bounds will fail. In particular, auxillary variables created by GFD
used to represent factored out subexpressions are given the default bounds,
and this may lead to unexpected failures, e.g.\ assuming the default bounds
was not changed, the following will fail:
\begin{verbatim}
    A :: 0..10^9, A #= 10^8/1.
\end{verbatim}
\noindent
because division (\texttt{/}) is non-inlined and is factored our, and 
\verb|10^8| is outside the default bounds.

%%% Not sure which point we are trying to make here:
%As mentioned above, one of the reason for parsing the expressions is to
%extract new domain variables. This is another difference between GFD and
%Gecode: Gecode requires the user to explicitly initialise domain variables 
%(IntVar) with a domain before using them, while GFD will give any new 
%domain variables a default domain, so variables do not need to be 
%initialised  with a domain before use. This GFD behaviour is compatible with
%IC (except the default domain is a finite integer domain like FD).



\subsubsection{Arithmetic constraints}

These constraints impose some form of arithmetic relation between their
arguments. Some of these constraints can occur inside expressions, while
others are ``primitive'' versions of the constraint where the arguments
are domain variables (or integers).

%%% Not sure which point we are trying to make here:
%Many of the constraints listed here are only available in IC inside
%expressions, i.e. they are not available as independent constraints.
%In GFD, all ``operators'' allowed in expressions have a constraint
%counterpart that can be posted outside of expressions, partly because
%Gecode does provide these constraints.


\begin{description}
%%% Commented out until we sort out the name
%\item[\biptxtrefni{divmod(?X,?Y,?Q,?M)}{divmod/4!gfd}{../bips/lib/gfd/divmod-4.html}]
%Constrains Q to X // Y, and M to X rem Y.

\item[\biptxtrefni{all_eq(?Collection,?Y)}{all_eq/2!gfd}{../bips/lib/gfd/all_eq-2.html}]
Constrains each element of Collection to be equal to Y. Similar constraints
for the other relations: 
\biptxtrefni{all_ge/2)}{all_ge/2!gfd}{../bips/lib/gfd/all_ge-2.html} (greater than or equal to), 
\biptxtrefni{all_gt/2}{all_gt/2!gfd}{../bips/lib/gfd/all_gt-2.html} (greater than),
\biptxtrefni{all_le/2}{all_le/2!gfd}{../bips/lib/gfd/all_le-2.html} (less than or equal to),
\biptxtrefni{all_lt/2}{all_lt/2!gfd}{../bips/lib/gfd/all_lt-2.html} (less than), and
\biptxtrefni{all_ne/2}{all_ne/2!gfd}{../bips/lib/gfd/all_ne-2.html} (not equal).


\item[\biptxtrefni{max(+Collection,?Max)}{max/2!gfd}{../bips/lib/gfd/max-2.html}]
Constrains Max to be the maximum of the values in Collection. Similarly,
\biptxtrefni{min(+Collection,?Min)}{min/2!gfd}{../bips/lib/gfd/min-2.html}
for minimum.

\item[\biptxtrefni{max_index(+Collection,?Index)}{max_index/2!gfd}{../bips/lib/gfd/max_index-2.html}]
Constrains Index to be the the index(es) of the variable(s) with the maximum of the values in Collection. Similarly,
\biptxtrefni{min_index(+Collection,?Index)}{min_index/2!gfd}{../bips/lib/gfd/min_index-2.html}\item[\biptxtrefni{max_index(+Collection,?Index)}{max_index/2!gfd}{../bips/lib/gfd/max_index-2.html}]
Constrains Index to be the the index(es) of the variable(s) with the maximum of the values in Collection. Similarly,
\biptxtrefni{min_index(+Collection,?Index)}{min_index/2!gfd}{../bips/lib/gfd/min_index-2.html}
for minimum.

\item[\biptxtrefni{max_first_index(+Collection,?Index)}{max_first_index/2!gfd}{../bips/lib/gfd/max_first_index-2.html}]
Constrains Index to be the the index of the first variable with the maximum of the values in Collection. Similarly,
\biptxtrefni{min_first_index(+Collection,?Index)}{min_first_index/2!gfd}{../bips/lib/gfd/min_first_index-2.html}
for minimum.

%%% does not exist (would clash with arithmetic builtin)
%\item[\biptxtrefni{max(?X,?Y,?Max)}{max/3!gfd}{../bips/lib/gfd/max-3.html}]
%Constrains Max to be the maximum of X and Y. Similarly,
%\biptxtrefni{min(?X,?Y,?Min)}{min/3!gfd}{../bips/lib/gfd/min-3.html} for
% minimum.

\item[\biptxtrefni{mem(+Vars,?Member [,?Bool])}{mem/2!gfd}{../bips/lib/gfd/mem-2.html}]
Constrains Member to be the a member element in Vars. The 
\biptxtrefni{reified}{mem/3!gfd}{../bips/lib/gfd/mem-3.html} version has the
\verb'Bool' argument.

\begin{sloppypar}
\item[\biptxtrefni{scalar_product(++Coeffs,+Collection,+Rel,?Sum [,?Bool])}{scalar_product/4!gfd}{../bips/lib/gfd/scalar_product-4.html}]
Constrains the scalar product of the elements of Coeffs
 and Collection to satisfy the relation {\em sum(Coeffs*Collection) Rel P}. 
\biptxtrefni{Reified}{scalar_product/5!gfd}{../bips/lib/gfd/scalar_product-5.html}
with \verb'Bool' argument.
\end{sloppypar}


\item[\biptxtrefni{sum(+Collection,?Sum)}{sum/2!gfd}{../bips/lib/gfd/sum-2.html}]
Constrains Sum to be the sum of the elements in Collection, or if the
argument is of the form IntCollection*Collection, the scalar product of the
connections.

\item[\biptxtrefni{sum(+Collection,+Rel,?Sum [,?Bool]}{sum/3!gfd}{../bips/lib/gfd/sum-3.html}]
Constrains the sum of the elements of Collection to satisfy the relation {\em sum(Collection) Rel Sum}.
\biptxtrefni{Reified}{sum/4!gfd}{../bips/lib/gfd/sum-4.html}
with \verb'Bool' argument.


\end{description}

\subsubsection{Ordering constraints}

These constraints impose some form of ordering relation on their arguments.

\begin{description}
\item[\biptxtrefni{lex_eq(+Collection1,+Collection2)}{lex_eq/2!gfd}{../bips/lib/gfd/lex_eq-2.html}]
Constrains Collection1 to be lexicographically equal to Collection2. 
Constraints for the other lexicographic relations:
\biptxtrefni{lex_ge/2}{lex_ge/2!gfd}{../bips/lib/gfd/lex_ge-2.html} (lexicographically greater or equal to),
\biptxtrefni{lex_gt/2}{lex_gt/2!gfd}{../bips/lib/gfd/lex_gt-2.html} (lexicographically greater than),
\biptxtrefni{lex_le/2}{lex_le/2!gfd}{../bips/lib/gfd/lex_le-2.html}
(lexicographically less or equal to), 
\biptxtrefni{lex_lt/2}{lex_lt/2!gfd}{../bips/lib/gfd/lex_lt-2.html},
(lexicographically less than),
\biptxtrefni{lex_neq/2}{lex_neq/2!gfd}{../bips/lib/gfd/lex_neq-2.html}
(lexicographically not equal to).

\item[\biptxtrefni{ordered(+Relation,+Collection)}{ordered/2!gfd}{../bips/lib/gfd/ordered-2.html}]
Constrains Collection to be ordered according to Relation.

\item[\biptxtrefni{precede(++Values,+Collection)}{precede/2!gfd}{../bips/lib/gfd/precede-2.html}]
Constrains each value in Values to precede its succeeding
value in Collection.

\item[\biptxtrefni{precede(+S,+T,+Collection)}{precede/3!gfd}{../bips/lib/gfd/precede-3.html}]
Constrains S to precede T in Collection.

\item[\biptxtrefni{sorted(?Unsorted, ?Sorted)}{sorted/2!gfd}{../bips/lib/gfd/sorted-2.html}]
Sorted is a sorted permutation of Unsorted.

\item[\biptxtrefni{sorted(?Unsorted, ?Sorted, ?Positions)}{sorted/3!gfd}{../bips/lib/gfd/sorted-3.html}]
Sorted is a sorted permutation (described by Positions) of Unsorted.

\end{description}

\subsubsection{Counting and data constraints}

These constraints impose restrictions either on the number of
 values that can be taken in one or more collections of domain
 variables, and/or on the positions of values in the collection.
\begin{description}
\item[\biptxtrefni{alldifferent(+Vars)}{alldifferent/1!gfd}{../bips/lib/gfd/alldifferent-1.html}]
Constrains all elements of Vars are different.

\item[\biptxtrefni{alldifferent_cst(+Vars,++Offsets)}{alldifferent_cst/2!gfd}{../bips/lib/gfd/alldifferent_cst-2.html}]
Constrains the values of each element plus corresponding offset to be pairwise different.

\item[\biptxtrefni{among(+Values, ?Vars, +Rel, ?N)}{among/4!gfd}{../bips/lib/gfd/among-4.html}]
The number of occurrences ({\em Occ}) in Vars of values taken from the set of
values specified in Values satisfies the relation {\em Occ Rel N}.

\item[\biptxtrefni{atleast(?N, +Vars, +V)}{atleast/3!gfd}{../bips/lib/gfd/atleast-3.html}]
At least N elements of Vars have the value V. Similarly 
\biptxtrefni{atmost(?N, +Vars, +V)}{atmost/3!gfd}{../bips/lib/gfd/atmost-3.html}.

\item[\biptxtrefni{count(+Value, ?Vars, +Rel, ?N)}{count/4!gfd}{../bips/lib/gfd/count-4.html}]
Constrains the number of occurrences of Value in Vars ({\em Occ}) to satisfy
the relation {\em Occ Rel N}.

\item[\biptxtrefni{count_matches(+Values, ?Vars, +Rel, ?N)}{count_matches/4!gfd}{../bips/lib/gfd/count_matches-4.html}]
The number of the elements in Vars that
 match their corresponding value in Values, {\em Matches}, satisfies the
 relation {\em Matches Rel N}.

\item[\biptxtrefni{element(?Index, +Collection, ?Value)}{element/3!gfd}{../bips/lib/gfd/element-3.html}]
Constrains Value to be the Index$^{th}$ element of the integer collection Collection.
 
\item[\biptxtrefni{gcc(+Bounds,+Vars)}{gcc/2!gfd}{../bips/lib/gfd/gcc-2.html}]
Constrains the number of occurrences of each Value in Vars according to the specification
in Bounds (global cardinality constraint).

\item[\biptxtrefni{nvalues(+Collection, +Rel, ?Limit)}{nvalues/3!gfd}{../bips/lib/gfd/nvalues-3.html}]
Constrains {\em N}, the number of distinct values occurring in 
Collection to satisfy the relation {\em N Rel Limit}.

\item[\biptxtrefni{occurrences(+Value,+Vars,?N)}{occurrences/3!gfd}{../bips/lib/gfd/occurrences-3.html}]
Constrains the value Value to occur N times in Vars.

\item[\biptxtrefni{sequence(+Low,+High,+K,+Vars,++Values)}{sequence/5!gfd}{../bips/lib/gfd/sequence-5.html}]
The number of values taken from Values is between Low and
High for all sequences of K variables in Vars. There is also a version for
binary (0/1) variables:
\biptxtrefni{sequence(+Low,+High,+K,+ZeroOnes)}{sequence/4!gfd}{../bips/lib/gfd/sequence-4.html}.
\end{description}

\subsubsection{Resource and scheduling constraints}
These constraints deal with scheduling and/or allocation of resources.

\begin{description}
\item[\biptxtrefni{bin_packing(+Items,++ItemSizes,+BinLoads)}{bin_packing/3!gfd}{../bips/lib/gfd/bin_packing-3.html}]
The one-dimensional bin packing constraint with loads: packing 
M items into N bins, each bin having a load specified in BinLoads.

\item[\biptxtrefni{bin_packing(+Items,++ItemSizes,+N,+BinSize)}{bin_packing/4!gfd}{../bips/lib/gfd/bin_packing-4.html}]
The one-dimensional bin packing constraint: packing M items
into N bins of size BinSize.

\item[\biptxtrefni{bin_packing_md(+Items,++ItemMDSizes,+BinMDLoads)}{bin_packing_md/3!gfd}{../bips/lib/gfd/bin_packing_md-3.html}]
The multi-dimensional bin packing constraint with loads: packing 
M L-Dimensional items into N L-Dimensional bins, each bin having a load in 
each dimension. The dimension L is implicitly specified in ItemMDSizes and
BinMDLoads.

\begin{sloppypar}
\item[\biptxtrefni{bin_packing_md(+Items,++ItemMDSizes,+N,+BinMDSize)}{bin_packing/4!gfd}{../bips/lib/gfd/bin_packing-4.html}]
The multi-dimensional bin packing constraint  loads: packing M L-Dimensional 
items into N L-Dimensional bins, each bin having a size in each dimension.
The dimension L is implicitly specified in ItemMDSizes and BinMDSize.
\end{sloppypar}

\item[\biptxtrefni{cumulative(+Starts,+Durations,+Usages,+ResourceLimit)}{cumulative/4!gfd}{../bips/lib/gfd/cumulative-4.html}]
Single-resource cumulative task scheduling constraint. A version with
optional tasks is also available:
\biptxtrefni{cumulative_optional(+StartTimes, +Durations, +Usages, +ResourceLimit, +Scheduled)}{cumulative_optional/5!gfd}{../bips/lib/gfd/cumulative_optional-5.html}.

\item[\biptxtrefni{cumulatives(+Starts,+Durations,+Heights,+Assigned,+Capacities)}{cumulatives/5!gfd}{../bips/lib/gfd/cumulatives-5.html}]
Multi-resource cumulatives constraint on specified tasks.

\item[\biptxtrefni{cumulatives_min(+Starts,+Durs,+Heights,+Assgn,+Mins)}{cumulatives_min/5!gfd}{../bips/lib/gfd/cumulatives_min-5.html}]
Multi-resource cumulatives constraint on specified tasks with
required minimum resource consumptions.

\item[\biptxtrefni{disjoint2(+Rectangles)}{disjoint2/1!gfd}{../bips/lib/gfd/disjoint2-1.html}]
Constrains the position (and possibly size) of the rectangles in
Rectangles so that none overlap. A version where placement of rectangles is 
optional is
\biptxtrefni{disjoint2_optional(+Rectangles)}{disjoint2_optional/1!gfd}{../bips/lib/gfd/disjoint2_optional-1.html}.

\item[\biptxtrefni{disjunctive(+StartTimes, +Durations)}{disjunctive/2!gfd}{../bips/lib/gfd/disjunctive-2.html}]
Constrains the tasks with specified start times and durations to not overlap in time. A version with optional tasks is also available:
\biptxtrefni{disjunctive_optional(+StartTimes, +Durations, +Scheduled)}{disjunctive_optional/3!gfd}{../bips/lib/gfd/disjunctive_optional-3.html}.

\end{description}

\subsubsection{Graph constraints}

In these constraints, the arguments represent a graph, and the
 constraint imposes some form of relation on the graph.

\begin{description}
\item[\biptxtrefni{circuit(+Succ)}{circuit/1!gfd}{../bips/lib/gfd/circuit-1.html}]
Constrains elements in Succ to form a Hamiltonian circuit.
A version allowing constant offsets is
\biptxtrefni{circuit_offset(+Succ,+Offset)}{circuit_offset/2!gfd}{../bips/lib/gfd/circuit_offset-2.html}.

\item[\biptxtrefni{circuit(+Succ,++CostMatrix,?Cost)}{circuit/3!gfd}{../bips/lib/gfd/circuit-3.html}]
Constrains elements in Succ to form a Hamiltonian circuit, with Cost
being the cost of the circuit, based on the edge cost matrix CostMatrix.
A version allowing constant offsets is
\biptxtrefni{circuit_offset(+Succ,+Offset,++CostMatrix,?Cost)}{circuit_offset/4!gfd}{../bips/lib/gfd/circuit_offset-4.html}.

\item[\biptxtrefni{circuit(+Succ,++CostMatrix,+ArcCosts,?Cost)}{circuit/4!gfd}{../bips/lib/gfd/circuit-4.html}]
Constrains elements in Succ to form a Hamiltonian circuit. ArcCosts
are the costs of the individual hops, and Cost their sum,
based on the edge cost matrix CostMatrix.
A version with constant offsets is available as
\biptxtrefni{circuit_offset(+Succ,+Offset,++CostMatrix,+ArcCosts,?Cost)}{circuit_offset/5!gfd}{../bips/lib/gfd/circuit_offset-5.html},

\item[\biptxtrefni{ham_path(?Start,?End,+Succ)}{ham_path/3!gfd}{../bips/lib/gfd/ham_path-3.html}]
Constrains elements in Succ to form a Hamiltonian path from Start to End.
A version with constant offsets is available as
\biptxtrefni{ham_path_offset(?Start,?End,+Succ,+Offset)}{ham_path_offset/4!gfd}{../bips/lib/gfd/ham_path_offset-4.html}.

\begin{sloppypar}
\item[\biptxtrefni{ham_path(?Start,?End,+Succ,++CostMatrix,?Cost)}{ham_path/5!gfd}{../bips/lib/gfd/ham_path-5.html}]
Constrains elements in Succ to form a Hamiltonian path from Start to End,
with Cost being the cost of the path, based on the edge cost matrix CostMatrix.
A version with constant offsets is available as
\biptxtrefni{ham_path_offset(?Start, ?End, +Succ, +Offset, ++CostMatrix, ?Cost)}{ham_path_offset/6!gfd}{../bips/lib/gfd/ham_path_offset-6.html}.

\item[\biptxtrefni{ham_path(?Start,?End,+Succ,++CostMatrix,+ArcCosts,?Cost)}{ham_path/6!gfd}{../bips/lib/gfd/ham_path-6.html}]
Constrains elements in Succ to form a Hamiltonian path from Start to End.
ArcCosts are the costs of the individual hops, and Cost their sum,
based on the edge cost matrix CostMatrix.
A version with constant offsets is available as
\biptxtrefni{ham_path_offset(?Start, ?End, +Succ, +Offset, ++CostMatrix, +ArcCosts, ?Cost)}{ham_path_offset/7!gfd}{../bips/lib/gfd/ham_path_offset-7.html}.

\item[\biptxtrefni{inverse(+Succ,+Pred)}{inverse/2!gfd}{../bips/lib/gfd/inverse-2.html}]
Constrains elements of Succ to be the successors and
Pred to be the predecessors of nodes in a digraph. A version with offsets
is also available:
\biptxtrefni{inverse(+Succ,+SuccOffset,+Pred,+PredOffset)}{inverse/4!gfd}{../bips/lib/gfd/inverse-4.html}.
\end{sloppypar}

\end{description}
 
\subsubsection{Extensional constraints}
These are ``user defined constraints'' (also known as ad-hoc
 constraints), i.e. the allowable tuples of values for a
collection of domain variables is defined as part of the constraint. These
predicate differs in the way the allowable values are specified.

\begin{description}
\item[\biptxtrefni{regular(+Vars, ++RegExp)}{regular/2!gfd}{../bips/lib/gfd/regular-2.html}]
Constrains Vars' solutions to conform to that defined in the regular expression RegExp.

\item[\biptxtrefni{table(+Vars, ++Table)}{table/2!gfd}{../bips/lib/gfd/table-2.html}]
Constrain Vars' solutions to be those defined by the tuples in Table.
The variant
\biptxtrefni{table(+Vars, ++Table, +Option)}{table/3!gfd}{../bips/lib/gfd/table-3.html}
allows the specification of the algorithm used.

\item[\biptxtrefni{extensional(+Vars, ++Transitions, +Start, +Finals)}{extensional/4!gfd}{../bips/lib/gfd/extensional-4.html}]
Constrain Vars' solutions to conform to the finite-state 
automaton specified by Transitions with start state Start and  final states Finals.

\end{description}

\subsubsection{Other constraints}

Constraints that don't fit into the other categories.

\begin{description}

\item[\biptxtrefni{bool_channeling(?Var, +DomainBools, +Min)}{bool_channeling/3!gfd}{../bips/lib/gfd/bool_channeling-3.html}]
Channel the domain values of Vars to the 0/1 boolean variables in DomainBools.

\item[\biptxtrefni{integers(+Vars)}{integers/1!gfd}{../bips/lib/gfd/integers-1.html}]
Pseudo constraint (i.e. no constraint will be posted in Gecode):
Vars' domain is the integer numbers (within default bounds).

\end{description}

%----------------------------------------------------------------------
\section{Search Support}
%----------------------------------------------------------------------

GFD allows search to be performed in two ways: 
completely encapsulated in the external Gecode solver, or
in {\eclipse}, supported by GFD's variable selection 
and value choice predicates. 

Additionally, GFD support various search facilities of Gecode that is
not directly supported by IC. These are described in section~\ref{addsearch}.

\subsection{Performing search completely inside Gecode}
\label{searcheng}
Search can be performed in Gecode using one of its search engines. 
In this 
case, the search to produce a solution appears as an atomic step at
the {\eclipse} level, and backtracking into the search will produce the next 
solution (or fail if there are none), again as an atomic step.

This direct interface to Gecode's search engines is provided by
\biptxtrefni{gfd:search/6}{search/6!gfd}{../bips/lib/gfd/search-6.html},
and uses a syntax similar to that of the generic search/6 predicates
(in {\tt lib(gfd_search)} (see below), {\tt lib(ic)} and {\tt lib(fd_search)}). 

As the search is performed in Gecode, it should be more efficient than doing
the search in \eclipse, where the system has to repeatedly switch between
Gecode and {\eclipse} when doing the search. As the search is a single atomic
step from the {\eclipse} level, it is not suitable if your code needs to
interact with the search, e.g. if you are using constraints defined at the
{\eclipse} level, and/or if you are using other solvers during the search.

On the other hand, GFD's search/6 is less flexible than the generic search
-- you can only use the predefined variable 
selection and value choice methods, i.e. you cannot provide user-defined
predicates for the Select and Choice arguments. 

The search engine to use is specified by the Method argument in search/6. 
One method provided by Gecode is bb_min -- finding a minimal solution using
branch-and-bound, which is not provided by the generic search. 

Instead, branch-and-bound in {\eclipse} is provided by {\tt lib(branch_and_bound)}, which can
be used with generic search's {\tt search/6} to provide a similar functionality as
the {\tt bb_min} method of GFD's {\tt search/6}. The {\eclipse} branch-and-bound is more 
flexible, but is likely to be slower. Note that {\tt lib(branch_and_bound)} can
be used in combination with GFD's search/6, but this is probably not useful
unless you are doing some search in your own code in addition to that done by 
search/6, or if you want to see the non-optimal solutions generated by the
search.


There are some differences in how search is performed by Gecode and generic
search;
the most significant is that all the built-in choice-operators of the generic
search library make repeated choices on one variable until it becomes ground,
before proceeding and selecting the next variable.  Gecode's built-in
strategies on the other hand always interleave value choices with variable
selection.

Other differences from generic search are: 
\begin{itemize} 

\item the partial search methods of generic search are not supported.
Instead, restart-based search are supported as search methods. This is 
discussed in section~\ref{restartsearch}.
\item Lightweight Dynamic Symmetry Breaking is supported natively 
(section~\ref{ldsb}).
\item when two or more variables can be selected by the variable selection
method, tie-breaking with a different method can be  used to chose between
these variables..
\item there are some differences in the available variable and value 
selection methods. Most importantly, variable selection methods based
on two dynamic measure of constraint propagations are available. This
is discussed in section~\ref{dynmeasure}.
\item parallel search is supported.
\end{itemize}

Here is the N-Queens example using GFD's \texttt{search/6}:
\begin{quote}
\begin{verbatim}
:- lib(gfd).

queens_list(N, Board) :-
    length(Board, N),
    Board :: 1..N,
    (fromto(Board, [Q1|Cols], Cols, []) do
        ( foreach(Q2, Cols), param(Q1), count(Dist,1,_) do
            Q2 #\= Q1,
            Q2 - Q1 #\= Dist,
            Q1 - Q2 #\= Dist
        )
    ),
    search(Board, 0, input_order, indomain_min, complete, []).
\end{verbatim}
\end{quote}


\subsection{Search in {\eclipse} using GFD primitives\label{searchgfd}}

The built-in Gecode search is appropriate when the problem consists
exclusively of GFD-variables and GFD-library-constraints, and when the
built-in search methods and search heuristics are sufficient to solve
the problem.

As soon as any user-defined constraints or any other {\eclipse}
solvers are involved, then the top-level search control should be
written in \eclipse, in order to allow non-gfd propagators to execute
between the labelling steps.  Also the implementation of problem-specific
search heuristics will usually make it necessary to lift the top-level
search control to the {\eclipse} level.
To make this possible, GFD provides primitives to support variable 
selection and value choice heuristics.

\begin{description}
\item[\biptxtrefni{gfd:select_var(-X, +Collection, -Rest, ++Arg, ++Select)}{select_var/5!gfd}{../bips/lib/gfd/select_var-5.html}]
Select a domain variable from Collection according to one of Gecode's
pre-defined selection criteria.  These include criteria not available in
other {\eclipse} solvers, like accumulated failure count.

\item[\biptxtrefni{gfd_search:delete(-X, +Collection, -Rest, ++Arg, ++Select)}{delete/5!gfd}{../bips/lib/gfd_search/delete-5.html}]
Select (and remove) a domain variable from Collection.  This is the
generic implementation, (compatible with IC and FD solvers), providing
a different choice of selection options, but likely to be less
efficient than select_var/5.

\item[\biptxtrefni{gfd:try_value(?Var, ++Method)}{try_value/2!gfd}{../bips/lib/gfd/try_value-2.html}]
This value choice predicate supports both Gecode-style binary choice and 
generic search's multi-way choice on the domain of a variable,
according to Method.
The binary-choice methods create two search alternatives, which reduce the variable domain
in complementary ways.  Because the variable is not necessarily instantiated,
this must be combined with a variable selection method that does not delete
the selected variable, such as select_var/5.

The multi-way choice methods make repeated choices on one variable as 
gfd_search's indomain/2.

\item[\biptxtrefni{gfd:indomain(?Var)}{indomain/1!gfd}{../bips/lib/gfd/indomain-1.html}]
Instantiate Var to elements in its domain, using a default method.

\item[\biptxtrefni{gfd_search:indomain(?Var, ++Method)}{indomain/2!gfd}{../bips/lib/gfd_search/indomain-2.html}]
A flexible way to nondeterministically assign values to finite domain
variables according to Method.  On success, Var is always instantiated.
This is the generic implementation
(compatible with IC and FD solvers), providing a different choice of methods,
and likely to be less efficient than try_value/5.
\end{description}

A simple search using GFD's primitives can be defined in the following way: 
\begin{quote}
\begin{verbatim}
labeling(Vars, Select, Choice) :-
        ( select_var(V, Vars, Rest, 0, Select) ->
            try_value(V, Choice),
            labeling(Rest, Select, Choice)
        ;
            true
        ).
\end{verbatim}
\end{quote}
For binary choice methods of try_value/2, the search will 
mimic Gecode's built-in search (where a variable selection step is usually
interleaved with a binary choice on the variable domain).

For multi-way choice methods of try_value/2,  
(possibly several) value choices on a variable are made until the
variable is ground, before proceeding to select the next variable: On
backtracking to the try_value/2, alternative values for the variable
will be tried. This mimics the behaviour of gfd_search's indomain/2,
but try_value/2 is likely to be more efficient as it is specifically
tailored for GFD.

The same effect can be achieved by using select_var/5 and try_value/2
together with the generic
\biptxtrefni{gfd_search:search/6}{search/6!gfd_search}{../bips/lib/gfd_search/search-6.html}
predicate:
\begin{quote}
\begin{verbatim}
gfd_search:search(Vars, 0,
        select_var(Select), try_value(Choice), complete, [])
\end{verbatim}
\end{quote}
\begin{sloppypar}
Several selection methods predicates that are designed to be used with
generic search, are provided:
\bipref{max_regret_lwb/2}{../bips/lib/gfd/max_regret_lwb-2.html}, \bipref{max_regret_upb/2}{../bips/lib/gfd/max_regret_upb-2.html}, \bipref{max_weighted_degree/2}{../bips/lib/gfd/max_weighted_degree-2.html},
\bipref{max_weighted_degree_per_value/2}{../bips/lib/gfd/max_weighted_degree_per_value-2.html}, \bipref{most_constrained_per_value/2}{../bips/lib/gfd/most_constrained_per_value-2.html}.
These allow selection methods supported in GFD but not in generic search
to be used with gfd_search:search/6.
\end{sloppypar}

For even more complex user-defined heuristics, various properties associated
with a variable and its domain can be obtained using predicates described
in section~\ref{gfdvarquery}. Note that these include properties that are not
available in
\eclipse's solvers, such as weighted degree (a.k.a. accumulated failure count).

\subsection{GFD specific search support\label{addsearch}}

GFD supports various search support facilities that are not found in 
\eclipse. This section discuss some of these features in more detail.

\subsubsection{Dynamic measure of propagation\label{dynmeasure}}

Gecode supports two ways of dynamically measuring constraint propagation for
variables. These measures can be used in variable selection heuristics 
that  'learn' from the actual propagations of the program. Such
heuristics may perform better, and is particularly suited to Gecode's
interleaving of variable selection and value choice. The two measures are:
\begin{description}
\item[weighted degree] Known as Accumulated Failure Count in Gecode. 
Count of the number of failures in
constraints associated with the variable. Count is updated after each
propagation failure.
\item[activity] Count of the number of domain
reduction on the variable during propagation. Count is updated after each
propagation.
\end{description}

As the search normally starts immediately after modelling, there would be
very little constraint propagation, so both measures
will not have much information to be used for variable selection.
One way to work around this is to assign initial
values to the counts, to give reasonable start values. This is the reason
that, by default, the initial value for weighted degree of a variable is set
to its degree. 

For {\bf gfd:search/6}, an alternative to setting the initial values is to
use the \texttt{tiebreak/1} option to use a tie-breaking
selection method. Another alternative is to use restart-based search, with
the initial search acting as a 'probe' to obtain good values for the measures
for the 'real' search.

Gecode also supports a {\it decay\/} factor for both measures, where the
count values can be reduced by the decay factor if their count is not 
incremented when the measure is updated: The idea is to favour those 
variables that are involved in the propagation most recently. 

GFD supports setting of both the initial values and the decay factor for
variable selection methods that uses weighted
degree and activity via parameters in
\biptxtrefni{gfd:search/6}{search/6!gfd}{../bips/lib/gfd/search-6.html} and
\bipref{select_var/5}{../bips/lib/gfd/select_var-5.html}. 

\begin{sloppypar}
Weighted degree is computed for all variables, and can be obtained with
\bipref{get_weighted_degree/2}{../bips/lib/gfd/get_weighted_degree-2.html}. It can be re-initialised using 
\bipref{init_weighted_degree/1}{../bips/lib/gfd/init_weighted_degree-1.html}. The decay factor can be obtained by
\bipref{get_weighted_degree_decay/1}{../bips/lib/gfd/get_weighted_degree_decay-1.html} and set with 
\bipref{set_weighted_degree_decay/1}{../bips/lib/gfd/set_weighted_degree_decay-1.html}.
\end{sloppypar}

Due to the way activity is implemented by Gecode, the use of activity in GFD 
is limited to the variable selection methods for {\bf search/6} and 
{\bf select_var/5}.

Note also activity is computed for recomputation as well as normal computation,
so changing the amount of recomputation (by changing the cloning distance)
can change the activity counts for the same problem, thus affecting the
search if an activity based selection method is used.

\subsubsection{Restart-based search methods\label{restartsearch}}

Restart is a technique supported by Gecode,
where the current search is abandoned and restarted from the root,
allowing the restarted search to make different branching choices,
and send the search to a different part of the search space.
This is useful if the old search was not fruitful in getting
a solution, so exploring a different part of the search space may prove
better than continuing the old search.

The restarted search will only be different from the old search
if different branching choices can be made, e.g. if at least some of these
choices are random, or if some of the choices have a learning component,
e.g. variable selection methods that uses activity or weighted degree.

The branching choices can also be different if the restarted search
is more constrained. Gecode supports no-goods learning, an automatic 
technique for remembering failures in the old search, and posting 
'no-good' constraints for the new search that prevent the search from 
repeating these failures.

Restart-based search is not complete, and is mostly useful for finding a
single solution, rather than many solutions.

GFD provides two search methods for \biptxtrefni{gfd:search/6}
{search/6!gfd}{../bips/lib/gfd/search-6.html} that are restart-based:
\texttt{restart} and \texttt{restart_min}. Both methods
will only find one solution.
 
\subsubsection{Lightweight Dynamic Symmetry Breaking}
\label{ldsb}
Lightweight Dynamic Symmetry Breaking (LDSB) is a symmetry breaking 
technique implemented for Gecode's search engines, and for IC in \eclipse
with \texttt{lib(ldsb)}. Currently, GFD supports LDSB for 
\biptxtrefni{gfd:search/6}{search/6!gfd}
{../bips/lib/gfd/search-6.html} only, as \texttt{lib(ldsb)} is written 
for IC only.

LDSB is supported in {\bf gfd:search/6} by the \texttt{ldsb_sym/1} option, 
where the  symmetries for the problem are specified, and can be used
for all value choice method. This is unlike 
\texttt{lib(ldsb)}, where LDSB is supported by separate predicates,
including predicates to perform value choice, rather than being integrated 
into the generic search

The syntax for the symmetry specifications follow those used in
\texttt{lib(ldsb)}. In particular, the definition of rows and columns 
for a 2-D matrix is the one used in \eclipse and in 
\texttt{lib(ldsb)}, but not in Gecode (where the definitions are reversed).
That is, the matrix are organised as collection of rows.

While LDSB is not currently available for GFD at the \eclipse level, Symmetry
breaking is available via SBDS
(Symmetry Breaking During Search) in \texttt{lib(gfd_sbds)}.

\section{User defined constraints and solver co-operation}
Like IC and FD solvers, GFD has facilities to allow the extension of the 
solver library so that GFD can co-operate with other solvers in solving a
problem, and also to allow the user to define their own constraints at the {\eclipse}
level. This is achieved by providing a suspension list with the gfd attribute,
which allows for the data-driven programming needed by solver co-operation and
constraint propagation, and a set of low-level predicates to process,
 query and  modify the domain of problem variables.

These facilities allow solver co-operation and user-defined 
constraints propagation at the {\eclipse} level, and not within Gecode directly.
So, search then must be done at the {\eclipse} level, i.e. not through Gecode's
search engines. The performance of constraints defined in this way will very
likely be less efficient than implementing the constraints directly in Gecode.

\subsection{The {\it gfd\/} attribute}

The GFD attribute is a meta-term which is attached to all GFD problem variables.
Many of the fields of the attribute are used for implementing the interface to
Gecode, and are of no interest to the user. The only field of interest is the
{\tt any} field, which is for the {\it any\/} suspension list, which is woken on 
any change in the domain of the variable:

\begin{verbatim}
gfd{
   ....
   any:SuspAny,
   ....
}
\end{verbatim}

The {\it any\/} suspension list has the same waking behaviour as the 
{\it any\/} suspension
list of FD, and is sufficient for implementing constraints -- the other 
suspension lists found in IC and FD are specialisations of the {\it any\/} 
suspension list, in that they provide more precise waking conditions. 
The reason that
only one suspension list is provided by GFD is to minimise the overhead in
normal use of the solver. 


In addition to waking the attribute's {\it any\/} suspension list, the 
{\it constrained\/}
suspension list will also be woken when a GFD variable's domain is changed,
and the {\it inst\/} suspension list will be woken if the variable is bound.

The suspension lists allow constraint propagation to be implemented at the
{\eclipse} level, which is distinct from the propagation of ``native'' Gecode
constraints, where each propagation phase (and in the case of using the 
search engine, the whole search) is an atomic step at the {\eclipse} level. 
This has a similar effect to running all ``native'' propagations
at a higher (more urgent) priority.
 
As only the {\it any\/} suspension list is provided, some rewriting of existing
user-defined constraints for IC and FD may be needed when such code is ported
for GFD.

\subsection{Modifying variable domains}

Like IC, GFD provides a set of predicates to modify the domain of GFD 
variables to support the writing of new constraints. Unlike normal constraints,
no Gecode level propagation or waking of other suspended goals (such as 
scheduled by other {\eclipse} level constraints) occurs with these predicates.

With the exception of
\bipref{impose_bounds/3}{../bips/lib/gfd/impose_bounds-3.html} none of
the goals call \bipref{wake/0}{../bips/kernel/suspensions/wake-0.html}, so
the programmer is free to do so at a convenient time.

Some of these predicates are provided for compatibility with IC, as these 
predicates have the same name and similar semantics to their IC counter-parts
(including the waking behaviour for {\tt impose_bounds/3}).
However, due to the difference in the way domains are represented in IC and
Gecode, these predicates may be inefficient for use with Gecode, particularly
if you need to modify multiple variables and/or multiple domain values. 
The predicates with names that begin with {\tt gfd_vars} are specific
to GFD and are designed to be more efficient than their IC compatible 
counter-parts.

The ``native'' primitives are:

\begin{description}
\item[\biptxtrefni{gfd_vars_exclude(+Vars,+Excl)}{gfd_vars_exclude/2!gfd}{../bips/lib/gfd/gfd_vars_exclude-2.html}]
Exclude the element Excl from the domains of Vars.

\item[\biptxtrefni{gfd_vars_exclude_domain(+Vars, ++Domain)}{gfd_vars_exclude_domain/2!gfd}{../bips/lib/gfd/gfd_vars_exclude_domain-2.html}]
Exclude the values specified in Domain from the domains of Vars.

\item[\biptxtrefni{gfd_vars_exclude_range(+Vars, +Lo, +Hi)}{gfd_vars_exclude_range/3!gfd}{../bips/lib/gfd/gfd_vars_exclude_range-3.html}]
Exclude the elements Lo..Hi from the domains of Vars.

\item[\biptxtrefni{gfd_vars_impose_bounds(+Vars, +Lo, +Hi)}{gfd_vars_impose_bounds/3!gfd}{../bips/lib/gfd/gfd_vars_impose_bounds-3.html}]
Update (if required) the bounds of Vars.

\item[\biptxtrefni{gfd_vars_impose_domain(+Vars,++Domain)}{gfd_vars_impose_domain/2!gfd}{../bips/lib/gfd/gfd_vars_impose_domain-2.html}]
Restrict (if required) the domain of Var to the domain specified  in Domain.

\item[\biptxtrefni{gfd_vars_impose_max(+Vars,+Bound)}{gfd_vars_impose_max/2!gfd}{../bips/lib/gfd/gfd_vars_impose_max-2.html}]
Update (if required) the upper bounds of Vars.

\item[\biptxtrefni{gfd_vars_impose_min(+Vars,+Bound)}{gfd_vars_impose_min/2!gfd}{../bips/lib/gfd/gfd_vars_impose_min-2.html}]
Update (if required) the lower bounds of Vars.

\end{description}

The IC-compatible primitives are:

\begin{description}
\item[\biptxtrefni{exclude(?Var, +Excl)}{exclude/2!gfd}{../bips/lib/gfd/exclude-2.html}]
Exclude the element Excl from the domain of Var.

\item[\biptxtrefni{exclude_range(?Var, +Lo, +Hi)}{exclude_range/3!gfd}{../bips/lib/gfd/exclude_range-3.html}]
Exclude the elements Lo..Hi from the domain of Var.

\item[\biptxtrefni{impose_bounds(?Var,+Lo,+Hi)}{impose_bounds/3!gfd}{../bips/lib/gfd/impose_bounds-3.html}]
Update (if required) the bounds of Var.

\item[\biptxtrefni{impose_domain(?Var,++Domain)}{impose_domain/2!gfd}{../bips/lib/gfd/impose_domain-2.html}]
Restrict (if required) the domain of Var to the domain of DomVar.

\item[\biptxtrefni{impose_max(?Var, +Hi)}{impose_max/2!gfd}{../bips/lib/gfd/impose_max-2.html}]
Update (if required) the upper bound of Var.

\item[\biptxtrefni{impose_min(?Var, +Lo)}{impose_min/2!gfd}{../bips/lib/gfd/impose_min-2.html}]
Update (if required) the lower bound of Var.

\end{description}


\subsection{Variable query predicates}
\label{gfdvarquery}

These predicates are used to retrieve various properties of a domain variable 
(and usually work on integers as well). 

In most cases, the property is obtained directly from Gecode. Many of these
properties are useful for selecting a variable for labelling. Here are some
examples:

\begin{description}
\item[\biptxtrefni{get_bounds(?Var, -Lo, -Hi)}{get_bounds/3!gfd}{../bips/lib/gfd/get_bounds-3.html}]
Retrieves the current bounds of Var.

\item[\biptxtrefni{get_constraints_number(?Var, -Number)}{get_constraints_number/2!gfd}{../bips/lib/gfd/get_constraints_number-2.html}]
Returns the number of propagators attached to the Gecode variable representing Var.

\item[\biptxtrefni{get_delta(?Var, -Width)}{get_delta/2!gfd}{../bips/lib/gfd/get_delta-2.html}]
Returns the width of the interval of Var.

\item[\biptxtrefni{get_domain(?Var, -Domain)}{get_domain/2!gfd}{../bips/lib/gfd/get_domain-2.html}]
Returns a ground representation of the current GFD domain of a variable.

\item[\biptxtrefni{get_domain_size(?Var, -Size)}{get_domain_size/2!gfd}{../bips/lib/gfd/get_domain_size-2.html}]
Returns the number of elements in the GFD domain of Var.

\item[\biptxtrefni{get_max(?Var,-Hi)}{get_max/2!gfd}{../bips/lib/gfd/get_max-2.html}]
Retrieves the current upper bound of Var. Similarly, \biptxtrefni{get_min/2)}{get_min/2!gfd}{../bips/lib/gfd/get_min-2.html} returns the lower bound.


\item[\biptxtrefni{get_median(?Var,-Median)}{get_median/2!gfd}{../bips/lib/gfd/get_median-2.html}]
Returns the median domain value of the GFD domain variable Var.

\item[\biptxtrefni{get_regret_lwb(?Var, -Regret)}{get_regret_lwb/2!gfd}{../bips/lib/gfd/get_regret_lwb-2.html}]
Returns the regret value for the lower bound of Var. Similarly, \biptxtrefni{get_regret_upb/2}{get_regret_upb/2!gfd}{../bips/lib/gfd/get_regret_upb-2.html}
for the upper bound.

\item[\biptxtrefni{get_weighted_degree(?Var,-WD)}{get_weighted_degree/2!gfd}{../bips/lib/gfd/get_weighted_degree-2.html}]
Returns the weighted degree (wdeg, accumulated failure count) of domain 
variable Var.

\item[\biptxtrefni{is_in_domain(+Val,?Var,[-Result])}{is_in_domain/2!gfd}{../bips/lib/gfd/is_in_domain-2.html}]
Succeeds iff Val is in the domain of Var. The \biptxtrefni{version}{is_in_domain/3!gfd}{../bips/lib/gfd/is_in_domain-3.html} with the \verb'+Result' argument
binds Result instead.


\item[\biptxtrefni{is_solver_var(?Term)}{is_solver_var/1!gfd}{../bips/lib/gfd/is_solver_var-1.html}]
Succeeds iff Term is an GFD domain variable.
\end{description}

\section{Low-level control of Gecode computation}
This section gives some information on the low-level workings of GFD and how
to adjust it. This information is not needed for most users, and can be 
skipped and consulted only when GFD is not behaving well with the user's
program.

GFD is designed so that the user can write programs that will run with
 Gecode without knowing any details about Gecode or how GFD interfaces
 to it. However, GFD does provide some parameters to control its behaviour.
While the default settings should work well under most circumstances, some
understanding of the inner workings of GFD is needed to change this default
behaviour. 

\subsection{Recomputation and Cloning}


Gecode implements search using a recomputation and cloning model,
which is fundamentally different from the backtracking model of \eclipse.
When failure occurs, Gecode does not backtrack to a previous computation
state; instead, the previous state is recomputed. To reduce the amount
of recomputation, the computation state is {\it cloned\/} periodically
during execution, and the recomputation would start from the nearest
such cloned state rather than from the start.

With GFD, when the search is done in Gecode (via GFD's 
\biptxtrefni{search/6}{search/6!gfd}{../bips/lib/gfd/search-6.html}), the
recomputation and cloning is handled by Gecode. When the search is done
at the {\eclipse} level, GFD handles the recomputation and cloning automatically,
so that the user does not need to be aware of it.

GFD will create clones of the Gecode state periodically, and when the
current Gecode computation state becomes invalid through {\eclipse} backtracking,
GFD will recompute the new state from the closest cloned state. 
The frequency at which GFD create clones can be adjusted by the user --
the more frequently clones are created, the more memory will be used, but
the cost of recomputation will be less. Cloning itself will take time,
and depends on the size of the state. Normally, the cost of cloning is
quite low, so GFD by default will clone frequently during search, as 
this leads to faster execution times. However, if the program has a large
state (many variables and constraints), then frequent cloning may lead to
excessive memory consumption (and larger computation state will also be 
more expensive to clone), and reducing the frequency of cloning may
improve performance.

The frequency of cloning in GFD is controlled by the
\texttt{cloning_distance} parameter, which can be changed by
\bipref{gfd_set_default/2}{../bips/lib/gfd/gfd_set_default-2.html}
(and the current value can be accessed with
\bipref{gfd_get_default/2}{../bips/lib/gfd/gfd_get_default-2.html})
The \texttt{cloning_distance} specifies a threshold for the number of
changes GFD makes to the Gecode state before a clone is created.
Note that GFD does not create a clone simply when cloning_distance is
exceeded, as it only creates a new clone when the cloned state would
be the state just before a choice-point, so clones will normally only
be created during search phase of the user program.

While it is not necessary for the user to specify the creation of a clone,
under very unusual circumstances -- when the \texttt{cloning_distance} is
set very high -- GFD may not produce a clone at the right place. So
\bipref{gfd_update/0}{../bips/lib/gfd/gfd_update-0.html} is provided to force the creation of a clone.
The expected usage is to call this predicate just before search starts.
For example, \texttt{labeling/3} can be written as:

\begin{quote}
\begin{verbatim}
labeling(Vs, Select, Choice) :-
    gfd_update,
    labeling1(Vs, Select, Choice, _).

labeling1(Vs, Select, Choice, VsH) :-
    ( select_var(V, Vs, 0, Select, VsH) ->
         indomain(V, Choice),
         labeling1(Vs, Select, Choice, VsH)
    ;
         true
    ).

\end{verbatim}
\end{quote}
Calling \texttt{gfd_update} before calling \texttt{labeling1} ensures that
GFD will only recompute inside the search.

\section{Main differences between GFD and IC}

Although GFD was designed to be code compatible with IC in terms of
syntax, there are still some unavoidable differences, because of differences 
between Gecode and IC. In addition, Gecode is 
implemented using very different implementation techniques from IC, and 
although such differences are mostly not visible in terms of syntax, there
are still semantics and performance implications. 

The main visible differences between GFD and IC are:
\begin{itemize}
       \item Real interval arithmetic and variables are not supported in GFD.

       \item Domain variables always have finite integer bounds, and the maximum 
       bounds are
       determined by Gecode. Like FD, default finite bounds are given to 
       domain variables that do not have explicit bounds, and the default
       settings for these bounds are below the maxima that Gecode allows.

       \item Constraint propagation is performed within Gecode, and each propagation
       phase is atomic at the {\eclipse} level. Posting of constraints and 
       propagation of their consequences are separate in Gecode. GFD uses a
       demon suspended goal to perform the propagation: after the posting
       of any constraint (and other changes to the problem that need
       propagation), this suspended goal is scheduled and woken. When the
       woken goal is executed, propagation is performed. 

       \item All constraints can be called from the gfd module, and in
       addition, some constraints can be called from modules that specify
       the consistency level: gfd_gac (generalised arc consistency, also
       known as domain consistency), gfd_bc (bounds consistency), gfd_vc (value
       consistency (naive)). The gfd module versions use the 
       default consistency 
       defined for the constraint by Gecode. These consistency levels map
       directly to those defined in Gecode for the constraints.

       \item gfd:search/6 interfaces to Gecode's search-engines, where the
       entire search is performed in Gecode, and the whole search appears
       atomic at the {\eclipse} level. 

       \item The suspension lists supported by GFD are different from IC.
       Currently, only the 'any' suspension list (for {\em any} changes to the
       variable's domain) found in FD but not IC, is supported. Note that
       the GFD constraints are implemented in Gecode directly, and therefore
       do not use GFD's suspension lists. 

      \item Constraint and integer expressions are designed to be compatible with 
      IC's, and the arithmetic operators and logical connectives supported 
      by Gecode are supported, and these largely overlaps those of IC's.
      In addition, ``functional'' (where the last argument is a domain 
      variable) and reified constraints can appear in expressions without the
      last argument, as in IC.

      The differences from IC are:
      \begin{itemize}
        \item User defined constraints are not allowed in expressions. 
 
        \item The operators and connectives supported are those supported by
        Gecode, so most of the IC operators for real arithmetic are not 
        supported.

        %%% Is this correct?
        \item Only inlined arithmetic (sub-)expressions are allowed between 
        logical connectives.

        \item GFD expressions are broken down into sub-expressions and 
       constraints that are supported natively by Gecode, where the additional
       sub-expressions are replaced by a domain variable in the original 
       expression. These domain variables are given the default bounds. 
      IC does something similar, but what constitutes additional sub-expressions
      will differ between GFD and IC, and the variables substituted
      for the sub-expressions would be given infinite bounds in IC.
 
      \end{itemize}

\item GFD variables cannot be copied to non-logical storage, and an error is 
raised if a GFD variable occurs in a term that is being copied for such purpose
(assert, non-logical variables, shelves, etc.). Note that this means that 
GFD is incompatible with Propia, as this library makes use of non-logical 
storage.
\end{itemize}

%HEVEA\cutend
