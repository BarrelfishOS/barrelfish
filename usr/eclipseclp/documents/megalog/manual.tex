% BEGIN LICENSE BLOCK
% Version: CMPL 1.1
%
% The contents of this file are subject to the Cisco-style Mozilla Public
% License Version 1.1 (the "License"); you may not use this file except
% in compliance with the License.  You may obtain a copy of the License
% at www.eclipse-clp.org/license.
% 
% Software distributed under the License is distributed on an "AS IS"
% basis, WITHOUT WARRANTY OF ANY KIND, either express or implied.  See
% the License for the specific language governing rights and limitations
% under the License. 
% 
% The Original Code is  The ECLiPSe Constraint Logic Programming System. 
% The Initial Developer of the Original Code is  Cisco Systems, Inc. 
% Portions created by the Initial Developer are
% Copyright (C) 2006 Cisco Systems, Inc.  All Rights Reserved.
% 
% Contributor(s): 
% 
% END LICENSE BLOCK


% File		: manual.tex
% Date		: March 1992
% Author	: Michael Dahmen
% Modified by	: Luis Hermosilla, August 1992
% Project	: MegaLog-Sepia User Manual
% Content	: Root file

%
%		Preamble
%

\documentstyle[11pt,html,a4wide,epsf]{book}
%
% $Id: sepiachip.tex,v 1.1 2006/09/23 01:49:41 snovello Exp $
%
% BEGIN LICENSE BLOCK
% Version: CMPL 1.1
%
% The contents of this file are subject to the Cisco-style Mozilla Public
% License Version 1.1 (the "License"); you may not use this file except
% in compliance with the License.  You may obtain a copy of the License
% at www.eclipse-clp.org/license.
% 
% Software distributed under the License is distributed on an "AS IS"
% basis, WITHOUT WARRANTY OF ANY KIND, either express or implied.  See
% the License for the specific language governing rights and limitations
% under the License. 
% 
% The Original Code is  The ECLiPSe Constraint Logic Programming System. 
% The Initial Developer of the Original Code is  Cisco Systems, Inc. 
% Portions created by the Initial Developer are
% Copyright (C) 2006 Cisco Systems, Inc.  All Rights Reserved.
% 
% Contributor(s): 
% 
% END LICENSE BLOCK

% This is not the original sepiachip.sty,
% but a drastically simplified one.
%

\parindent 0pt
\itemsep 0pt

\newcommand{\enableunderscores}{\catcode`\_=8}
\newcommand{\disableunderscores}{\catcode`\_=\active}
\newcommand{\newitem}[1]{\item[#1]}
\newcommand{\bip}[1]{{\bf #1}\index{#1}}
\newcommand{\bipref}[2]{{\bf #1}\index{#1}}
\newcommand{\biptxt}[2]{{\bf #1}\index{#2}}
\newcommand{\txtbip}[2]{{\bf #1}\index{#1}}
\newcommand{\biptxtref}[3]{{\bf #1}\index{#2}}
\newcommand{\txtbipref}[3]{{\bf #1}\index{#1}}

\typeout{We are in latex mode}

\newcommand{\vbar}{$\mid$}
\newcommand{\uparr}{$\wedge$}
\newcommand{\bsl}{$\backslash$}
\newcommand{\andsy}{$/\backslash$}
\newcommand{\orsy}{$\backslash/$}
\newcommand{\tld}{$\sim$}
\newcommand{\lbr}{$[$}
\newcommand{\rbr}{$]$}
\newcommand{\nil}{$[~]$}
\newcommand{\lt}{$<$}
\newcommand{\gt}{$>$}
\newcommand{\chr}{\sf CHR}
\newcommand{\chrs}{{\sf CHR}s}
\def\eclipse{ECL$^i$PS$^e$}
\def\tkeclipse{TkECL$^i$PS$^e$}
\def\sepia{SEPIA\ }


% Allow underscores as normal characters (but lose subscripts...)
% [moved into include file because of latex2html problem]
\catcode`_=\active

\let\ifonline=\iffalse

\hyphenation{me-ta-term me-ta-terms}



%
% Document
%

\begin{document}

\bibliographystyle{plain}

\title{
    {\Large \eclipse}\\
    \vspace{1cm}
    {\Huge Knowledge Base User Manual}\\
    \vspace{1cm}
    Release 3.4}
\date{July 1994}
\author{ }

\maketitle

% Needed to adjust left/right pages properly
\setcounter{page}{2}
% Suppress printing of the page number on this page
\pagestyle{empty}


Copyright \copyright\ 1992 -- 2006 Cisco Systems, Inc.

\cleardoublepage
\pagestyle{plain}
\pagenumbering{roman}

\begin{latexonly}
\tableofcontents
\end{latexonly}

\cleardoublepage
\pagenumbering{arabic}

% BEGIN LICENSE BLOCK
% Version: CMPL 1.1
%
% The contents of this file are subject to the Cisco-style Mozilla Public
% License Version 1.1 (the "License"); you may not use this file except
% in compliance with the License.  You may obtain a copy of the License
% at www.eclipse-clp.org/license.
% 
% Software distributed under the License is distributed on an "AS IS"
% basis, WITHOUT WARRANTY OF ANY KIND, either express or implied.  See
% the License for the specific language governing rights and limitations
% under the License. 
% 
% The Original Code is  The ECLiPSe Constraint Logic Programming System. 
% The Initial Developer of the Original Code is  Cisco Systems, Inc. 
% Portions created by the Initial Developer are
% Copyright (C) 2006 Cisco Systems, Inc.  All Rights Reserved.
% 
% Contributor(s): 
% 
% END LICENSE BLOCK

% File		: intro-sec.tex    
% Date		: March 1992
% Author	: Michael Dahmen
% Modified by	: Luis Hermosilla, August 1992
% Project	: ECLiPSe User Manual
% Content	: Introduction to MegaLog-Sepia 

\chapter{Introduction}

\eclipse provides a powerful programming environment for building 
next generation Data\-base \& Knowledge Base Management Systems.  
\eclipse integrates a knowledge base with a logic programming 
language to provide large scale persistent storage of knowledge in such a 
way that it can be efficiently accessed and processed by logic programs.

\eclipse integrates all the functionality of the MegaLog system
(\cite{BOC90}, \cite{BDH90}) and the Sepia system 
(\cite{SP91}) in a single system.
This manual is restricted to the database and knowledge base aspects of
\eclipse -- 
i.e.\ the MegaLog part. A description of the Prolog functionality 
-- i.e. the Sepia part -- is found in \cite{SP91}.

This manual is directed at users with some experience of Prolog and 
relational databases, so in areas where \eclipse is similar to conventional
systems the explanations are kept brief. 

\section{Features of \eclipse}

\begin{itemize}
\item  Efficient retrieval of knowledge for any size of 
       knowledge base.
\item  Deductive rules can be stored in the 
       database.
\item  Complex structures and lists are valid data types
       for the database.
\item  Database transparency.  When making a query the user 
       does not have to specify whether the data is stored in 
       main memory or in the database.
\item  The database can be queried with either set-oriented  
       or single-tuple operations.  Backtracking can be used
       to navigate through the database and extract all 
       solutions to a goal.
\item  A full multi-user environment is provided (e.g.\ 
       concurrent access and transactions).
\item  Full database recovery procedures are included.
\item  Full garbage collection.  
\end{itemize}

\newpage

\section{Modules for Persistent \eclipse}

The database and knowledge base systems are implemented as several \eclipse
modules, which are 
listed below. The separation into several modules allows the user
of \eclipse to view the system as either a pure
Prolog, a Prolog augmented with different database functionalities or as a
pure MegaLog. 
All predicates exported by these modules are listed in 
chapter \ref{bip-summary} and fully described in the 
Knowledge Base Built-In Specification Manual (or Knowledge Base BIP Book),
\cite{BIP92}. 

\begin{itemize}

\item{database\_kernel}

This module is the lowest implementation level of the database and knowledge
base. It provides a direct access to the database without going through
processing steps of higher levels. Most of this interface is for more
advanced users, e.g.\ as target language for a deductive database system
(for example EKS-V1 \cite{EKS91}).

\item{db}

This module provides a relational algebra language embedded into
Prolog. A tutorial style description of the relational algebra
is given in chapter \ref{database-sec}.

\item{kb}

This module provides deductive relations i.e. relations that contain
Prolog clauses as tuples. A tutorial style description of the 
deductive relations is given in chapter \ref{knowbase-sec}.

\item{knowledge\_base}

This module contains the user's knowledge base itself. For further details
see chapter \ref{knowbase-sec}.

\item{megalog}

This is a library module that provides backwards compatibility to the
MegaLog system
that worked independently of {\eclipse}. It is useful for those users who
intend to migrate their existing MegaLog programs to \eclipse.
New developments should not use this module.

\end{itemize}

The modules {\bf database\_kernel}, {\bf db} and {\bf kb}
are present in the system by default.
However, as long as none of these modules is explicitly loaded, they
will not come into existence.
All modules that want to use database functionalities should load one of
the above modules. This is achieved by adding one of the following lines
to the source code of those modules:

\begin{verbatim}
:- import database_kernel.

:- lib( db).

:- lib( kb).
\end{verbatim}

Note that the operators
are defined locally in the modules where {\tt lib/1} has been called.

The module {\bf megalog} is loaded with {\tt lib(megalog)}.
Most existing MegaLog applications can then run unchanged in this
module. This is achieved by

\begin{verbatim}

[sepia]: lib(megalog).
[sepia]: module(megalog).
[megalog]: compile(application).

\end{verbatim}

The module {\bf megalog} provides an environment where most differences
between \eclipse and the Prolog part of MegaLog are hidden.
The remaining differences are discussed in chapter \ref{backwards-compat}.

Applications that are newly developed in \eclipse should not
use the MegaLog compatibility module. Such new developments
may themselves be broken up into several modules.

% What is MegaLog-Sepia

% BEGIN LICENSE BLOCK
% Version: CMPL 1.1
%
% The contents of this file are subject to the Cisco-style Mozilla Public
% License Version 1.1 (the "License"); you may not use this file except
% in compliance with the License.  You may obtain a copy of the License
% at www.eclipse-clp.org/license.
% 
% Software distributed under the License is distributed on an "AS IS"
% basis, WITHOUT WARRANTY OF ANY KIND, either express or implied.  See
% the License for the specific language governing rights and limitations
% under the License. 
% 
% The Original Code is  The ECLiPSe Constraint Logic Programming System. 
% The Initial Developer of the Original Code is  Cisco Systems, Inc. 
% Portions created by the Initial Developer are
% Copyright (C) 2006 Cisco Systems, Inc.  All Rights Reserved.
% 
% Contributor(s): 
% 
% END LICENSE BLOCK

% File		: database-sec.texx
% Date		: March 1992
% Author	: Michael Dahmen
% Modified by	: Luis Hermosilla, August 1992
% Project	: MegaLog-Sepia User Manual
% Content	: Tutorial on the Relational Algebra

\newpage

\chapter{The \eclipse Database}
\label{database-sec}
\index{database}

\section{Introduction}

\eclipse uses the Relational Database Model as the basis 
for its database design, and adapts it to 
allow {\em deductive rules} to be stored in relations.
This removes the restriction found in conventional relational 
databases of only being able to store knowledge as 
explicit facts, and allows more complex representations of knowledge to be
kept in the database.  

The purpose of \eclipse is to give the user a {\em logic programming environment}
in which to build large scale D/KBMSs.  This requires the storage limitations 
found in Prolog compilers (i.e. small capacity and no persistency) to be removed
without losing any of the logic programming power.
Therefore the \eclipse database has been designed
to provide mass storage of knowledge in such a way that it can be conveniently
and efficiently accessed by logic programs.

The relational approach
used by the \eclipse database is 
very well suited to this as a relation
has a one-to-one correspondence with a predicate
(a relation's name \& attributes correspond to a predicate's functor \& arguments, and
the tuples in the relation define the clauses of the predicate).
A logic program can therefore obtain clauses
by extracting single tuples from a relation with {\eclipse}'s `tuple-at-a-time' operations.
\eclipse also provides backtracking through the database to step
through all the solutions of a goal.

As an alternative method for accessing the database \eclipse provides the 
set-oriented operations found in relational databases (i.e. selection,
union, difference and join). 
This gives the user the
freedom to select either method, or even to mix them, depending on what
is best for the particular application.

By integrating the database within the logic programming environment
the database access has been made very efficient.  Its performance
is comparable with both Prolog compilers (which only access clauses in main memory),
and conventional relational databases.  The user is therefore provided 
with the functionality of both without losing the performance of either.

\eclipse provides two versions of its database, one with the deductive
capability described above and one without.  This is because DBMSs, unlike
KBMSs, do not need the facility to store deduction rules
and by removing this functionality from the database it is possible to 
enhance the access performance. Therefore there is the deductive {\bf KB} version
of the database and the non-deductive
{\bf DB} version.
The rest of this chapter describes how to implement a database using the DB version
, and the next chapter shows how to implement a knowledge base using the 
KB version.  Chapter~\ref{multi} covers Multi User \eclipse, which 
provides concurrent access to both DB and KB versions of the database.

\newpage
\section{Building a Database}

To build a database the user needs to be
able to perform the following operations:

\begin{itemize}
\item Create the database.
\item Define a relation.
\item Insert clauses.
\item Access clauses. 
\item Remove clauses.
\item Remove a relation.
\end{itemize}
This section covers these operations for the DB version of the \eclipse database.

\subsection{Creating the Database}
\label{create}

A database is created with the predicate {\bf createdb/1}. \index{createdb/1}
The argument of the predicate is either the 
full pathname of the database, or just its name 
(in which case it will be put in the current directory).
For example  

\begin{verbatim}
?- createdb('/usr/applic/bank_data').
\end{verbatim}
or
\begin{verbatim}
?- createdb('bank_data').
\end{verbatim}

In both these cases (assuming that the current
directory is `/usr/applic/') a directory will
be created within `/usr/applic/' with the name `bank\_data'. 
This directory contains all the necessary files for storing
data and maintaining the database.

If the database already exists and just needs to be opened, then the 
predicate {\bf opendb/1} \index{opendb/1} can be used. 
This opens the database which has its pathname as the argument.
Again, if the argument is just the database name it is assumed
to be in the current directory.

The database is automatically closed when you leave \eclipse or
when you open another database (note that only one database
can be open at a time).  To close a database in other circumstances the 
{\bf closedb/0} \index{closedb/0} predicate is used.

\subsection{Defining a Relation}	
\index{relation}
\index{attribute}
A relation is added to the database by declaring its name and all its 
attributes. This information is entered in the database with the 
{\bf $<=>$/2} predicate. \index{$<=>$/2}

As an example let's assume we have just created a database,
and wish to add a relation describing the employees of a certain 
firm. The relation is called `employee' and has the 
attributes `name', `salary',  `department' and
`number', where `number' is a unique identifier for the employee
and is to be used as the key of the relation. `number' is an integer
that takes up to four bytes to store, `name' is restricted to being
no more than twelve characters in length, `department' is stored in 4  bytes,
and `salary' in 2 bytes.  In \eclipse this would be entered in the
database as follows

\begin{verbatim}
?- employee <=> [integer(number,4,+),
                 atom(name,12,-),
                 integer(department,4,-),
                 integer(salary,2,-)].
\end{verbatim}
To understand this predicate we look at its syntax

\begin{verbatim}
Relation_Name <=> [ Attribute1,
                    Attribute2,
                           ....
                    AttributeN ].
\end{verbatim}

where each attribute is defined in the following way

\begin{verbatim}
Type(Name, Length, Index)
\end{verbatim}

The first argument of the {\bf \verb-<=>-} predicate is the name of the
relation.  This can either be an atom or a variable depending
on whether the relation is to be {\em permanent} or {\em temporary}.
\index{permanent relation} \index{temporary relation}
A permanent relation is one that persists in the database when 
the database is closed, while a temporary one will be destroyed.
In the permanent case the {\bf Relation\_Name} is an atom that
is unique (i.e. no other relation shares this name) and of no more
than 31 characters. When the relation is to be temporary a variable is used
for {\bf Relation\_Name}, and this will be instantiated by the system 
to a name for the relation. For example
\begin{verbatim}
?- boy <=> [ atom(name,20,+),
             integer(age,1,-) ].
yes

?- X   <=> [ atom(name,20,+),
             integer(age,1,-) ].
X = p1364r1
yes
\end{verbatim}

The two examples produce relations with the same attributes, but
the first has the name `boy' allocated by the user and is permanent, while
the second has the name `p1364r1' allocated by the system and is temporary.


For each attribute we have to give four pieces of information:

\begin{itemize}

\item{{\bf Type}}

An attribute can be one of four types:
{\bf integer}, {\bf real} , {\bf atom}, and {\bf term}.  Floating point numbers
are given by the real type, and character strings by the atom type. Any Prolog
term can be stored as type term, however general Prolog terms are
less efficient and should not be used where one of the three other types
would be sufficient.

\item{{\bf Name}}
 
The attribute's name is an atom or a string, which is silently
truncated to 31 characters. Attribute names should be unique within
a relation i.e. two different relations can share the same
attribute name. However, the uniqueness is not tested and duplicate names
are not rejected.

\item{{\bf Length}}
 
Depending on the type chosen there are certain
restrictions placed on the length of an attribute value.
Integers can be 1, 2, or 4 bytes in length, reals are always 4 bytes and atoms
can be any length.  For the term type this argument is ignored and there
is no restriction on the length; the
system allocates space as required.  An extra restriction is that the maximum number
of bytes for the whole relation (i.e. the total of the field lengths
for all the attributes) must not exceed 1000 bytes, where a term field counts
12 bytes.

\item{{\bf Index}}

\index{Indexing}
\index{Relation Index}
 
This indicates whether the relation's index should include the attribute.
\eclipse provides a very powerful multi-attribute indexing facility
({\bf BANG} -- Balanced And Nested Grid file), which
allows to index equally on several attributes using a single index
tree. This achieves good performance even with varying and unpredictable
access patterns.

The index field is {\bf +} to indicate preference in participation
in index and {\bf --} otherwise. Attributes of type term cannot be included
in the index. Since a relation must always have at least one index
attribute there must always be at least one attribute that is not
of type term.
If no index attribute is given, all attributes (except the
terms) are indexing.
 
\end{itemize}

When a relation is created, the field length and index
values need not be specified (use a free variable instead).
The default values for field length
are four bytes for integers and reals, and 24 characters for atoms.
The default setting for index is {\bf --}, e.g.
\begin{verbatim}
?- boy <=> [atom(name,T1,I1), integer(age,T2,I2)].

T1 = 24
I1 = -
T2 = 4
I2 = -
yes.
\end{verbatim}

Whenever values are entered into a relation that exceed
the field length defined in the schema they will be
truncated as necessary, but no error message is given.
The maximum number of attributes allowed in a relation is 50.
\index{attribute maximum number of}


If a relation already exists, it is possible to use the predicate 
{\bf \verb-<=>-} to find its schema. 
This is done by making the first argument the name of the relation
and the second a variable. The variable then becomes instantiated to
the schema. For example

\begin{verbatim}
?- boy <=> X.

X = [atom(name, 24, +), integer(age, 4, +)]
yes.
\end{verbatim}

It is often useful to refer to the same relation by a number of names.
To be able to do this {\em synonyms} must be defined. This is
done as follows: 
\index{synonym}
\begin{verbatim}
Relation_Name <-> Synonym.
\end{verbatim}
\index{$<->$/2}
where {\bf Relation\_Name} and {\bf Synonym} are atoms.
\label{syn}
\begin{verbatim}
e.g. boy <-> son
\end{verbatim}

{\bf son} becoming a synonym for {\bf boy}.


The same predicate may be 
used to find all defined synonyms for a relation if the right
hand side is a variable, or to find the real name of a relation
whose synonym is known if the left hand side is a variable.

A particular situation when it is necessary to refer to a 
relation by its synonym is when it is being joined with itself
(the synonym allows the attributes of the two occurrences of the 
relation to be distinguished in the selection conditions).

\subsection{Removing a Relation}
\index{relation - remove}
\index{$<=>$/2}
Temporary relations are automatically deleted at the end of the session
or when the database is closed.
Permanent ones can be removed from the database by using
the {\bf \verb-<=>/2-} predicate. The call 
\begin{verbatim}
Relation_Name <=> [].
\end{verbatim}
will destroy the given relation if it exists; otherwise the call fails.

\subsection{Examining the Database State}
Two simple predicates are available that give some basic information
about the current state of the database: 
{\bf helpdb/0} lists the names of all the relations in the database; 
{\bf helprel/1} gives information about a single relation. 
\index{helpdb/0} \index{helprel/1}

\begin{verbatim}
?- helprel(department).


RELATION : department    [real name: department]

ARITY: 3

ATTRIBUTES :
    atom(dname, 20, +)
    integer(floor, 4, +)
    atom(manager, 20, +)

NUMBER OF TUPLES : 3

yes
\end{verbatim}

To get a complete listing of all the tuples stored in a
relation along with the above information the {\bf printrel/1}
predicate can be used.  The argument is the name of any relation in the
database. \index{printrel/1}


The arity and cardinality (number of tuples) of a relation may 
be queried through two similarly named predicates
\index{arity/2} \index{cardinality/2}
\begin{verbatim}
?- arity(manager, Ary), cardinality(manager, NoTups).
Ary = 2
NoTups = 5
	more? -- 

yes
\end{verbatim}


Another useful predicate is a utility provided for the comparison
of relational schemas. The call
\index{schema comparisons}
\index{$<$@$>$/2}
\begin{verbatim}
relation1 <@> relation2
\end{verbatim}
will succeed if the number and types of the attributes of the 
two relations are the same. 


\section{Data Manipulation - Relational Algebra}
\label{alg1}
Data can be retrieved from the database by using either
relational algebra or tuple-at-a-time operations.  We start
with relational algebra.

\label{isr}
Just as a predicate {\bf is/2} is provided to allow the writing of arithmetic 
expressions in Prolog, the predicate {\bf isr/2} \index{isr/2}
permits the inclusion of relational algebraic \index{relational algebra}
expressions in \eclipse programs.  
The following sections will illustrate
its use, and introduce some additional predicates specific to \eclipse.
A call to the {\bf isr} predicate has the form
\begin{verbatim}
Relation_Name   isr   Relational_Expression
\end{verbatim}

This creates a temporary relation and uses the {\bf Relational\_Expression}
to determine which tuples from other relations should be inserted into it. 
If {\bf Relation\_Name} is an atom the relation takes it for its name, 
and if it is a variable then \eclipse generates its own name for the relation.  
In both cases the relation is temporary, being destroyed when the database 
is closed. If the relation already exists the tuples given by
{\bf Relational\_Expression} are added to it.
The {\bf Relational\_Expression} consists of relations, relational operators, 
and conditions.  The operators are given in table~\ref{ops}.
The following sections demonstrate how different operators are used
to retrieve the required sets of data.

\begin{table}
\centering
\begin{tabular}{||c|c||}        
 \multicolumn{2}{c}{\em Operators}  \\ \hline
 :\verb+*+: & join \\
 :\verb-+-: & union  \\
 :\verb+-+: & difference  \\
 :\verb+^+: & projection   \\ \hline
 \multicolumn{2}{c}{ } \\
 \multicolumn{2}{c}{\em Selection}  \\ \hline
 \verb-where- & selection condition \\
 \verb-and- & conjunction \\  \hline
 \multicolumn{2}{c}{ } \\
 \multicolumn{2}{c}{\em Predicates}  \\ \hline
 \verb+isr+ & relational assignment \\
 \verb+<--+ & set deletion \\
 \verb-<++- & set insertion \\
 \verb-++>- & set retrieval  \\ \hline
\end{tabular}
\caption{ Relational Operators and Predicates in \eclipse}
\label{ops}
\end{table}

\subsection{Selection Conditions}
\label{selection} \index{selection}

When we want  tuples of a relation
satisfying some condition (e.g. `All employees earning over 100000 DM', or
`Any pupil who studies English'),  a boolean condition may  be
used in an {\bf isr} relational expression to perform selection
on a relation. This condition is introduced by the word {\bf where}.
\index{where condition}
\begin{verbatim}
high_paid isr employee where salary >= 100.
% All employees earning over 100 thousand

EngStudent isr pupil where degree == 'English'
% Any pupil who studies English
\end{verbatim}

Both these examples have {\em simple attribute expressions} for their
conditions; these are
a single comparison of attributes and atomic Prolog terms using the operators
\verb-==-, \verb-\==-, \verb->=-, \verb-=<-, \verb-<- and \verb->-. 
Any operator may compare either two attributes or an attribute with
a constant. Note that attributes of 
type term cannot be used in these expressions.

{\em Complex attribute expressions} can be formed to give a conjunction 
by using the connective {\bf and}. Further examples
\index{attribute expression}
\index{conjunction} 
\begin{verbatim}
?- grade2 isr emp where salary < 100 and salary > 50.

?- f_depts isr department where manager == 'F*'.
\end{verbatim}

The last example illustrates the use of the wild character `\verb-*-' in 
\index{wild character}
comparisons involving attributes of atom type. When a string containing
the wild character is compared against an attribute, the 
wild character matches any string of zero or more characters. So, in 
the example above, {\bf \verb-f_depts-} is created from all tuples of 
{\bf department} whose manager attribute is a string beginning
with `F'. Note that the wild character `\verb-*-' may only occur as last
character of the pattern string.



Similar the wild character `\verb-?-' in an atom comparison matches
exactly one character, e.g. in the example below all names of three
characters length are selected. 
\begin{verbatim}
?- three_long isr department where name == '???'.
\end{verbatim}

The wild characters are only interpreted in equality and inequality
conditions. In range 
conditions (\verb->=-, \verb-=<-, \verb-<- and \verb->-)
they are taken as normal characters,
since the interpretation would make no sense.



There is no disjunctive `or' for complex attribute expressions,
but disjunctive conditions can be formed as follows:

\begin{verbatim}
% Giving: answer isr Rel1 where ( Cond1 or Cond2 )

% poor performance solution

?- X1 isr Rel1 where Cond1,
   X2 isr Rel1 where Cond2,
   answer isr X1 :+: X2.

% better performance

?- answer isr Rel1 where Cond1,
   answer <++ Rel1 where Cond2,

\end{verbatim}

The first solution inserts each tuple twice in a relation. Tuple
insertions are fairly expensive and much more expensive than 
the retrievals. Therefore the second solution requires only about
half the time.

\subsection{Join}
\label{joins} \index{join} \index{$:*:$}

The join of two relations is performed by the operator \verb-:*:-.
An optional {\bf where} condition is given to indicate the condition on which
the join is made. Thus a join takes the form
\begin{verbatim}
ResultRel isr JoinRel1 :*: JoinRel2 where Condition
\end{verbatim}
where {\bf JoinRel1} and {\bf JoinRel2} are relations 
which are joined according to  {\bf Condition}
to produce the relation {\bf ResultRel}. Some examples of 
joins are
\begin{verbatim}
manager_grade1 isr employee :*: manager 
               where employee_id == manager_id and salary > 200.
Manages isr manager :*: department where manager_department == department_id.
\end{verbatim}

The join condition may compare attributes from both relations
with each other or with constants using any comparison operator.
This allows all kind of joins, including {\em theta joins}.
\index{join theta}
If no join condition is given a {\em cartesian product} is performed.

It is possible for an ambiguity to arise in the  expression
specifying the join condition if the relations participating in the
join have commonly named attributes.\index{join ambiguity (\verb-^-)}
To resolve the ambiguity the notation \verb-^- is introduced. 
If the relations {\em r1(a, b)} and {\em r2(a, c)} exist, they may be 
joined on their first attributes by:
\begin{verbatim}
?- rel isr r1 :*: r2 where r1^a == r2^a.
\end{verbatim}

\subsection{Difference}
\index{difference of relations} \index{$:-:$}
The difference of two relations is performed by the operator \verb+:-:+. 
An optional {\bf where} condition is given to indicate the condition on which
the difference is made. Thus a difference takes the form
\begin{verbatim}
ResultRel isr DiffRel1 :-: DiffRel2 where Condition
\end{verbatim}
The simplest example of a difference is
\begin{verbatim}
?- X isr a :-: b.
X = p2249r7

yes
\end{verbatim}
which creates the temporary relation `p2249r7' containing those
tuples of `a' which do not occur in `b'.

The difference condition may compare attributes from both relations
with each other or with constants using any comparison operator.
This allows not only simple differences as above but also general
difference, which are also called {\em complementary joins}.
\index{join complementary}
If no difference condition is given the implicit condition is `all
attributes are equal'.

\subsection{Union}
\index{union of relations} \index{$:+:$}
The union of two relations is performed by the operator \verb-:+:-.
An optional {\bf where} condition is given to indicate the condition on which
the union is made. Thus a union takes the form
\begin{verbatim}
ResultRel isr UnionRel1 :+: UnionRel2 where Condition
\end{verbatim}

Note that argument names appearing in {\bf Condition} are used to select
in {\em both} {\bf UnionRel1} and {\bf UnionRel2}, therefore the
{\bf Condition} can only refer to attributes of both relations.
Actually the union is only included for syntactical completeness, a sequence
of {\bf isr} and {\bf \verb-<++-} achieves the same effect.
\begin{verbatim}
ResultRel isr UnionRel1 where Condition
ResultRel <++ UnionRel2 where Condition
\end{verbatim}


\subsection{Projection}
\index{projection} \index{$:\verb-^-:$}
The operations presented so far create relations with all the attributes
of the relations on the right hand side of the {\bf isr} 
predicate. We usually want our result relation to have only some of
these attributes. The projection operator {\bf \verb-:^:-} acts as a filter
on the relation created by the selections, joins etc. in an {\bf isr}
call, and lets the result relation consist of only those attributes
in a given list, called the {\em projection list}. The projection
operator is used as follows:
\begin{verbatim}
Result_Relation isr Projection_List :^: Relational_Expression
\end{verbatim}

First a relation is constructed from the {\bf Relational\_Expression} as 
described above (i.e. as if there 
was no projection operator), and then the result relation 
is produced by extracting from it the attributes listed in the projection list.
The attributes in the result relation are in the same order as in the
{\bf Projection\_List}. 
We can adapt the example given in section \ref{joins} so that
only the attributes of name and salary are put in the result relation
\begin{verbatim}
man_grade1 isr [name, salary] :^: 
        employee :*: manager where employee_id == manager_id and salary > 200
\end{verbatim}
This creates the relation {\bf man\_grade1/2}, with a schema that satisfies 
\begin{verbatim}
man_grade1 <=> [atom(name,_,_), real(salary,_,_)].
\end{verbatim}


The {\bf \verb-^-} notation may also be used in a projection list if there
is an ambiguity about which attributes are to be projected.


\subsection{Adding to Relations}
\label{insertion into rels}
\index{relation - set insertions} \index{$<++$/2}
Adding tuples to a relation is performed with the \verb-<++/2- predicate,
which takes either a relational expression or a list of tuples and puts them
in the specified relation.
The relation is named on the left hand side of the operator and it
must already exist. If the relation does not exist the goal will fail.

\begin{verbatim}
highpaid <++ [name, salary] :^: emp where salary > 20.

employee <++  [ [5421, aaron, 3, 22],
                [4529, schindler, 1, 30],
                [8796, deuker, 4, 51] ].
\end{verbatim}
In the first example tuples are selected from a relation and projected
into the target. The syntax allowed on the right hand side of the insertion
operator is the same as the syntax used on the right hand side of the
{\bf isr/2} predicate.

The second example inserts tuples explicitly given in a list.
Each tuple is in the form of a list of values enclosed 
in square brackets. If any tuple does not match the schema in the 
relation declaration, for instance does not have the correct
number of attributes, an error exception is raised. In this case the remaining
tuples in the list are not inserted.

\subsection{Retrieval from Relations}
\index{relation - set retrieval}
The operator \verb-++>- \index{$++>$/2} retrieves a set of tuples
as a Prolog list. 
This is in a sense the inverse of the \verb-<++/2- predicate above.
The set retrieval operator takes the form
\begin{verbatim}
Projection_List :^: Relational_Expression ++> List
\end{verbatim}
The syntax of {\bf Relational_Expression} is described above, the
{\bf Projection_List} is optional. For example
\begin{verbatim}
?- employee where name \== schindler ++> X.

X = [ [5421, aaron, 3, 22], [8796, deuker, 4, 51] ]
\end{verbatim}
This retrieval is much more efficient than retrieving tuple-at-a-time
inside a {\bf findall/3} operator as in the next example.

\begin{verbatim}
?- findall([A,B,C,D],
           (retr_tup(employee,[A,B,C,D]),B \== schindler),
           X).
X = [ [5421, aaron, 3, 22], [8796, deuker, 4, 51] ]
\end{verbatim}
Note that set retrieval is limited by the about of global stack
that is available. If there are more tuples selected from
the relation than fit into the main memory stack, a stack overflow
will be signaled.

\subsection{Deleting from Relations}
\index{relation - set deletions}
The operator \verb+<--+ \index{$<--$/2} performs the deletion of 
tuples from a relation. 
It may be used to delete tuples from a relation that are given 
either by a relational expression or as an
explicit list. For example
\begin{verbatim}
employee <-- emp2 where number > 5000.

employee <-- [ [5421, aaron, 3, 22],
               [4529, schindler, 1, 30] ].
\end{verbatim}
Again, as illustrated in the second case above, the right hand side
of the deletion operator may be any expression that is legal for 
the right hand side of  {\bf isr/2}.

Some attributes may be left uninstantiated, as in the following
example:

\begin{verbatim}
?- employee <-- [ [Number, eder, _] ].
Number = Number
	more? -- 

yes
\end{verbatim}

The result of such a goal is to delete {\em all} the tuples 
that have attribute values that match the instantiated values given
in the goal.  Any variables in the goal
are left {\em uninstantiated} and the goal cannot be resatisfied.
So in the example all the tuples with 
the second attribute equal to `eder' will be removed,
and Number is returned as an uninstantiated variable.

The \verb+<--+ predicate may  be used to empty a relation. 
\begin{verbatim}
r <-- r where true
\end{verbatim}
will delete all tuples from {\bf r} which are in {\bf r}, leaving the
relation empty. Note that it is much more efficient to remove the relation
and to create it again.

\subsection{Aggregates}
\index{aggregates}
An attribute of an \eclipse database relation may have an aggregating
property. This means that on each tuple insertion an aggregating
operation is performed between the tuples in the relation and
the new inserted tuple. There are four aggregates in \eclipse relations.

\paragraph{min}
This attribute will always hold the minimum value of all inserted
tuples. This is possible on attributes of type real, integer and
atom.

\paragraph{max}
This attribute will always hold the maximum value of all inserted
tuples. This is possible on attributes of type real, integer and
atom.

\paragraph{sum}
This attribute will always hold the sum of all inserted
tuples. This is possible on attributes of type real and integer.

\paragraph{count}
This attribute will count the number of tuples inserted.
This requires an integer type attribute.


If a relation has {\em only} aggregating attributes it will
contain at most one tuple. The more interesting case is when 
some attributes are aggregating and others not. The
non aggregating attributes are called grouping attributes, as
they serve to compute the aggregates over several groups of tuples.


In the example below there is an employee relation with three attributes.
To compute the average salary of employees per department a temporary
relation with two aggregating attributes is setup.
The average is then simply computed by retrieving each tuple from
the result relation and a simple division.

\begin{verbatim}
?- employee <=> Format.
Format = [atom(name,30,+),
          atom(department,30,+),
          integer(salary,4,-)]

?- R <=> [atom(department,30,+),
          sum(integer(salary_sum,4,-)),
	  count(integer(employees_per_department,4,-))],
   R <++ [department, salary, salary] :^: employee.
R = p456r2

?- retr_tup(p456r2,[Dep,Sum,Count]),
   Avg is Sum / Count.
\end{verbatim}


\section{Data Manipulation - Tuple-at-a-Time Operations}
\index{tuple-at-a-time operations}
The operations of section \ref{isr} are performed on relations, that
is, upon whole sets of tuples. In a logic programming environment 
we need to be able to extract single instantiations of a predicate and to
use backtracking to find all the instantiations that satisfy a goal.
This leads to a `tuple-at-a-time'
approach, where the tuples of a relation are accessed one at a 
time.  

\subsection{Adding to Relations}
\index{relation - tuple insertion}
\index{ins_tup/1}
The predicates {\bf ins\_tup/1} and {\bf ins\_tup/2} insert single
tuples into a relation. In the first form of the predicate the
tuple is viewed as a Prolog structure; in the second, as a list
of attribute values.
\begin{verbatim}
ins_tup( employee(8282, mann, 1, 43) ),
ins_tup( employee, [2324, doerr, 3, 22]).
\end{verbatim}
In both cases a single tuple is added to the employee relation,
which has four attributes. The next example shows the insertion
of values into a relation with an attribute of type term:
\begin{verbatim}
data <=> [ integer(id,2,+), term(info,_,X) ],

ins_tup( data(1, smith(male,single,young)) ),
ins_tup( data(2, (young(Y) :- Y =< 35 )) ).
\end{verbatim}
    
\subsection{Retrieving Data}
\index{relation - tuple retrieval}
\index{retr_tup/2}
We may view a database relation as a definition of a predicate,
in which case a retrieval of values from the database may be seen
as the satisfaction of a goal. The predicate {\bf retr\_\/tup/2} 
is used to express this goal. The first argument names the relation from which
tuples are to be taken. This name may be the relation name occurring
in the database, or a synonym from a frame declaration. 
The second argument is a list of length equal 
to the arity of the relation. The elements of this list are matched
against the tuples of the relation. Thus
\begin{verbatim}
?- retr_tup(employee, [Number, Name, Dept, Salary]).
Number = 5426
Name = 'acher'
Dept = 2
Salary = 19
        more? -- ;
Number = 4342
Name = 'adolf'
Dept = 4
Salary = 34
        more? -- 

yes

?- retr_tup(data, [2, (young(20) :- Body)]), call(Body).
Body = 20 =< 35)
	more? -- 

yes
\end{verbatim}
The second example is an extension of the example in the previous section, 
and shows how Prolog clauses can be saved in the database and then
retrieved and executed.

The predicate {\bf retr\_\/tup} is resatisfiable. Successive calls 
to {\bf retr\_\/tup} through backtracking will retrieve successive tuples 
from the relation in question.

If the arity of the relation is unknown, it is also possible to ask
\begin{verbatim}
?- retr_tup(employee, Tuple).
Tuple = [5426, 'acher', 2, 19]
	more? -- 

yes
\end{verbatim}
thus returning the tuple as a list.


A variant is {\bf retr\_tup/1}. This views the tuples of the relation as 
Prolog structures, for example
\begin{verbatim}
?- retr_tup(employee(A,B,C,D)).
A = 5426
B = 'acher'
C = 2
D = 19
	more? -- 


yes
\end{verbatim}

A third variant {\bf retr\_tup/3} allows the inclusion
of a condition restricting the tuples considered to a 
subset of the relation. 
\begin{verbatim}
retr_tup(employee, [A, B, C, D], salary > 3001.20).
\end{verbatim}
The condition in the third argument is any condition that
is valid in an {\bf isr} clause behind the {\bf where}.

An equality retrieval condition can either be given as third argument
to {\bf retr\_tup/3} or as instantiation of the second argument.
\begin{verbatim}
retr_tup(employee, [A, acher, C, D]).
retr_tup(employee, [A, B, C, D], name == acher).
\end{verbatim}
These two forms are identical, both in semantics and in the performance
of execution. However, there is one exception. 
The wild characters `\verb-*-' and `\verb-?-' are interpreted only in
the second form. In the first form wild characters are taken as
normal characters.

Both tuple access and set oriented operations can be used to perform
the same operations. Below we illustrate this duality with an example
\begin{verbatim}
tmp isr [a1, b3] :^: a :*: b where a2 == b1,
result isr [a1, c1] :^: tmp :*: c where b3 == c2.
\end{verbatim}
is set oriented, while
\begin{verbatim}
retr_tup(a(A1,A2,_)), 
retr_tup(b(A2,_,B3)), 
retr_tup(c(C1,B3)).
\end{verbatim}
is tuple-at-a-time.

In most cases the set oriented operations are more efficient, because
the database can do more intelligent join than simple nested loops.
On the other hand the tuple access allows an easy interference with
further Prolog goals, that can express more complicated conditions.
The set oriented example needs space for the intermediate relations
{\bf tmp} and {\bf result}. If these relations are very big, a
space-time tradeoff may favour spending more time by using tuple
access that does not need these intermediate relations.

To make the retrieval of tuples from the database transparent (i.e. so there
is no difference between main-memory and database access of unit clauses) 
a predicate can be defined as follows:
\index{database transparency}
\begin{verbatim}
p(X,Y) :- retr_tup(p(X,Y)).
\end{verbatim}

Then a tuple can be retrieved from the relation {\bf p/2} with the 
following form of goal instead of having to use the {\bf retr\_tup/1}
predicate each time:

\begin{verbatim}
?- p(X,a).
\end{verbatim}

    
\subsection{Deleting from Relations}
\index{relation - tuple deletion}
\index{del_tup/1}
\index{del_tup/2}
\index{del_tup/3}
The predicates {\bf del\_tup/1} and {\bf del\_tup/2} delete single
tuples from a relation. In the first form of the predicate the
tuple is viewed as a Prolog structure; in the second, as a list
of attribute values.  
\begin{verbatim}
del_tup(employee(4576, hainz, 2, 23)),
del_tup(employee,  [9939, N, _, _]).        
N = 'beck'
        more? -- 

yes
\end{verbatim}
There exists also a predicates {\bf del\_tup/3}, which is like 
{\bf del\_tup/2} but takes a condition as third argument.

An attribute can be left uninstantiated in the goal, as in the above example,
and when a matching tuple is found the variable will become bound.
Backtracking can then be invoked to delete other tuples from the database
that match the goal.

% Database connection 

% BEGIN LICENSE BLOCK
% Version: CMPL 1.1
%
% The contents of this file are subject to the Cisco-style Mozilla Public
% License Version 1.1 (the "License"); you may not use this file except
% in compliance with the License.  You may obtain a copy of the License
% at www.eclipse-clp.org/license.
% 
% Software distributed under the License is distributed on an "AS IS"
% basis, WITHOUT WARRANTY OF ANY KIND, either express or implied.  See
% the License for the specific language governing rights and limitations
% under the License. 
% 
% The Original Code is  The ECLiPSe Constraint Logic Programming System. 
% The Initial Developer of the Original Code is  Cisco Systems, Inc. 
% Portions created by the Initial Developer are
% Copyright (C) 2006 Cisco Systems, Inc.  All Rights Reserved.
% 
% Contributor(s): 
% 
% END LICENSE BLOCK

% File		: knowbase-sec.tex
% Date		: March 1992
% Author	: Michael Dahmen
% Modified by	: Luis Hermosilla, August 1992
% Project	: MegaLog-Sepia User Manual
% Content	: Tutorial on the deductive relations

\newpage

\chapter{The \eclipse Knowledge Base}.
\label{knowbase-sec}
\label{bang2}

\section{Introduction}

In the previous chapter we used the DB version of the \eclipse database.
We now consider the full KB
version that allows deduction rules to be stored
in relations, making them  {\em deductive relations}.
Deduction rules enable facts to be derived rather than 
stated explicitly, and therefore greatly extend the range
of knowledge we can store.
For example, if there is a flight  
from London to Munich at 14:00 on every day of the year 
we can store this knowledge
as :
\begin{verbatim}
flight(london,munich,1400,Date) :- valid_date(Date). 
\end{verbatim}

If, on the other hand, we could only store facts
in our database 
then it would be necessary to have an entry for each flight.
\begin{verbatim}
flight(london,munich,1400,'01/01/90'). 
flight(london,munich,1400,'02/01/90'). 
flight(london,munich,1400,'03/01/90'). 
           etc. 
\end{verbatim}


Traditionally the rules and facts that make up the knowledge base
of an application have been treated separately,
with the facts being stored in a database
on secondary storage and the rules being programmed in main memory.
This results in very high maintenance overheads when
rules are altered or added, and therefore makes systems with
regularly changing rules impractical to implement.
By storing rules together with the facts in the \eclipse deductive database
it becomes much easier to maintain a knowledge base and
therefore broadens the range of possible applications.


The deductive database can also be used in a multi
user environment to allow shared access.  This is described
in chapter~\ref{multi}.

\section{Building a Knowledge Base }


The predicates used to implement a knowledge base are very similar to those 
described for the DB version of the database.  In many cases they only differ
in their names, with
the ending \verb+-db+ being replaced with \verb+-kb+ (e.g. {\bf createdb}
becomes {\bf createkb})
or a symbol being repeated (e.g.\verb+<=>+ becomes \verb+<==>+). 


The {\bf createkb/1}, {\bf openkb/1}, and {\bf closekb/0} predicates all 
perform the same job for the knowledge base as the \verb+-db+ versions
did for the basic database. Therefore only a brief description
is given here. For a full explanation see section~\ref{create} or the
Knowledge Base BIP Book \cite{BIP92}. 

\paragraph{createkb(KB)} creates a knowledge base. 
Physically, the knowledge base is
a collection of files kept in the directory specified by `KB'.  If
the full pathname is not given the knowledge base is created in
the current directory.
\index{createkb/1}
\paragraph{openkb(KB)} opens the knowledge base specified by `KB'.
\index{openkb/1}
\paragraph{closekb} closes the knowledge base that is currently open. 
\index{closekb/0}


\begin{table}[h]
\centering
\begin{tabular}{||c|c|c||}        
\hline 
 Process &  DB Version & KB Version \\ \hline
 Create database & createdb/1 & createkb/1   \\
 Open database & opendb/1 & openkb/1   \\ 
 Create Relation & $<=>$/2 & $<==>$/2 \\
 Relation Info. & helprel/1 & helpdrel/1 \\
 Create Synonym &  $<$$-$$>$/2 & $<$$-$$-$$>$/2 \\
 Insert Clause & ins\_tup/1 & insert\_clause/1 \\
 Retrieve Clause &  retr\_tup/1 & retrieve\_clause/2 \\ \hline
\end{tabular}
\caption{ Comparison of Some of the DB and KB Predicates}
\label{comp}
\end{table}

\subsection{Defining a Deductive Relation}
\index{$<==>$/2} \index{deductive relation}
The predicate used to define a deductive relation is {\bf \verb+<==>/2+}, where
the first argument is the relation's name and the second the
schema.  The full syntax is as follows 

\begin{verbatim}
Relation_Name <==> [ {+}Attribute_Name1,
                     {+}Attribute_Name2,
                             ...
                     {+}Attribute_NameN ]
\end{verbatim}

where

\begin{itemize}

\item The {\bf Relation\_Name} can be either an atom or a variable
depending on whether the relation is to be permanent or temporary.

\item Each {\bf Attribute\_Name} is an atom. A prefix  
$+$ indicates that the attribute is preferred in the index.

\item The attribute type and field length do not need to be specified,
as \eclipse allocates memory as required. 

\end{itemize}


To define the employee relation that we used before we would enter
the following

\begin{verbatim}
?- employee <==> [ +number, name, dept, salary ].
\end{verbatim}



An important difference to the DB version is that the KB version of the 
database allows to store any Prolog term, including 
{\em complex structures}, {\em lists} and {\em variables},
without any type or size declaration. This is however less efficient
than the DB version, i.e. one should not use a KB relation in cases
where a DB relation would do, too.

Therefore, having defined the above relation, we can store any of the following:

\begin{verbatim}
employee( 1002, smith, [d146,d149], 10000).
employee( 1002, name(fred,smith), dept(accounts,d146), 
          salary(10000, [insurance,pension,bonus])).
employee( 1002, smith, d146, Salary ) :- salary_gen(1002, Salary).
\end{verbatim}

 
Note that there is no restriction on the size of lists or complex structures.


Synonyms are created and queried with the {\bf \verb+<-->/2+} predicate, 
which works in the same way as  
{\bf \verb+<->/2+ } described in section~\ref{syn} .  For example
\index{synonym} \index{$<-->$/2}
\begin{verbatim}
?- employee <--> personnel.
\end{verbatim}

Note that there is only one name space for both relations and synonyms
that is shared by the KB and DB version.

\subsection{Querying a Relation's Schema}

Once a relation has been defined there are a number
of queries that can be made to get information
about its schema and the number of clauses it contains.
These queries are made with the following
predicates:

\paragraph{ helpkb } gives the schema, arity and number of clauses
\index{helpkb/0}
for each of the relations in the open knowledge base.

\paragraph{ helpdrel(Relation\_Name) } gives the same information
\index{helpdrel/1}
as above, but only for
the relation specified as the argument. 

\paragraph{ degree(Relation\_Name, Arity) } returns the arity (in the
\index{degree/2}
variable Arity) of the specified relation.

\paragraph{ cardinality(Relation\_Name, Nclauses) } returns the number of  
\index{cardinality/2}
clauses in the specified relation. 

\paragraph{ Rel1 $<$@@$>$ Rel2 } succeeds if the schema of  
\index{$<$@@$>$/2}
the two relations
(Rel1 and Rel2) are identical (in which case they are union compatible).


\section{Data Manipulation - Tuple-At-A-Time Operations }

We now consider how to insert and retrieve data from a relation
using tuple-at-a-time operations. 

\subsection{Inserting Clauses}
\label{insert}

Clauses are added to a relation using the following predicates:

\paragraph{ insert\_clause(Clause) } inserts one clause into the relation
\index{insert_clause/2}
designated by the head of the clause. For example

\begin{verbatim} 
?- insert_clause( flight(london,munich,1400,sunday) ).
\end{verbatim}

and

\begin{verbatim} 
?- insert_clause( ( flight(london,munich,1400,Day) :- week_day(Day) ) ).
\end{verbatim}

will both insert a clause into the flight relation. Note that a second pair of
brackets is needed round the clause.  There is no restriction
on the number of sub-goals in the complex clause, so the following 
example is equally valid:

\begin{verbatim}
flight( munich, Destination, Day, Depart, Arrive) :-
        valid_des(munich, Destination),
        valid_day(Day),
        depart_time( Destination, Day, Depart),
        arrive_time( Depart, munich, Destination, Arrive).
\end{verbatim}


If groups of clauses are to be added then the batch insertion predicates
can be used.  These allow clauses to be entered from the terminal
or from file. 

\paragraph{insert\_clauses\_from(user)}
\index{insert_clauses_from/1}
is used to add clauses from the terminal. Enter the predicate 
{\bf insert\_clauses\_from(user)} and then type in 
the clauses.  When they have all been entered type \verb+^D+
and they will be inserted.  If there is an error in a
clause it will be skipped and insertion will continue with
the next clause. This is similar to compilation of clauses.

\paragraph{insert\_clauses\_from(File\_Name)} 
is for adding clauses from a file.  {\bf File\_Name} is
an atom that can either be just the name of the file,
if it is in the current directory, or the full pathname.
The file is a text file containing clauses, similar to the files
accepted by the {\bf compile} predicate. Directives are
executed in module {\bf knowledge_base}, not in the current module.
For example
\begin{verbatim}
?- insert_clauses_from('flight.dat').
\end{verbatim}

where {\bf flight.dat} contains

\begin{verbatim}
flight(munich,london,1600,Day) :- week_day(Day).
flight(munich,london,1800,friday).
            ...
\end{verbatim}

\subsection{Retrieving Clauses}
\index{retrieve_clause/1}
The {\bf retrieve\_clause((Head :- Body))} predicate will try to find a
clause in the database with a head that unifies with {\bf Head} and
a body that unifies with {\bf Body}. In most cases {\bf Body} will
be a variable at the time of calling, and will get instantiated to 
the body of the clause with the unifying head.  Backtracking can be invoked 
to find further solutions to the goal. Simple facts have a body that
consist of the atom {\bf true} only.
For example:
\begin{verbatim}
?- retrieve_clause(( flight(munich, london, X, Y) :- Body )).
\end{verbatim}
would retrieve the two clauses we used to illustrate
the {\bf insert\_clause\_from(File)} predicate as follows:
\begin{verbatim}
Body = week_day(_g8)
Y = _g8
X = 1600
	more? -- ;
Body = true
Y = friday
X = 1800
	more? -- ;

no
\end{verbatim}

\subsection{Deleting Clauses}
\index{delete_clause/1} \index{retract_clause/1}

To remove a clause from a relation there is the {\bf delete\_clause/1 }
predicate, with its argument being the clause that is to be deleted.
\eclipse will look through the relevant relation until it finds
a clause that is a variant of the argument. Two clauses are variants 
if they are identical apart from a consistent renaming of variables.
This clause is then deleted and the user is asked if the search should 
continue for another occurrence of the argument.
For example let us assume the following clauses exist in the knowledge
base :

\begin{verbatim}
(i)   flight(london,munich,1400,saturday).
(ii)  flight(london,munich,1400,Day) :- week_day(Day).
\end{verbatim}

and that we enter the following predicates:
\begin{verbatim}
?- delete_clause( (flight(london,munich,1400,X) :- week_day(X)) ).
\end{verbatim}
This will delete clause (ii) as it is identical to the argument except
for the name of the variable.
  
\begin{verbatim}
?- delete_clause( flight(From,munich,1400,saturday) ). 
\end{verbatim}
This will have no effect as clause (i) has its first argument instantiated
and clause (ii) has a completely different structure. 

There is also a predicate {\bf retract\_clause/1} which is similar
to the {\bf retract/1} of the Prolog main memory database. 
{\bf retract\_clause/1} is like {\bf delete\_clause/1}, except that
not only variants but {\em unifying} clauses are retracted.

\subsection{Executing Clauses}

To execute a clause stored in a deductive relation it is first retrieved
using {\bf retrieve_clause/1} and then meta-called using {\bf call/1}.
Example 

\begin{verbatim}
?- retrieve_clause(( flight(munich,london,Time,friday) :- Body )), 
   call(Body).
Body = true
Time = 1600
        more? -- ;
Body = true
Time = 1800
        more? -- ;

no
\end{verbatim}

\index{database transparency}
If these two actions are done automatically when a goal referring 
to a deductive relation must be resolved, the knowledge base
becomes {\em transparent}.  In other words there is no difference
in the way that tuples are extracted from main-memory and the
knowledge base. This is easyly achieved by compiling into main
memory a clause
\begin{verbatim}
Relation_Name(Att1, Att2, ...,AttN) :- 
     retrieve_clause(( Relation_Name(Att1, Att2, ..., AttN) :- Body )),
     call(Body).

for example 

flight(A,B,C,D) :- 
     retrieve_clause(( flight(A,B,C,D) :- Body)),
     call(Body).
\end{verbatim}

\index{define_implicit/1}
For convenience there is the predicate {\bf define\_implicit/1}
that creates such a clause and add it to the current `definitions'
in the knowledge base and therefore 
making the relation permanently transparent. Example :
\begin{verbatim}
?- define_implicit( flight/4 ).
\end{verbatim}

Note that the definition is made in the module {\tt knowledge\_base},
which conceptually contains the knowledge base.
{\bf define\_implicit/1} can only be invoked on deductive
relations that already exist. On permanent relations it may only
be invoked in single user mode.

\subsection{Definitions}
\index{definitions}
Once entered into the system some predicates are very stable and rarely
get updated.
For example the predicate {\bf week\_end/1} with the facts
{\bf week\_end(saturday)} and {\bf week\_end(sunday)} is very unlikely
to change. So that these `reference predicates' can be separated from
other predicates \eclipse provides {\em definitions}.
A definition is a persistent predicate that is stored separately from
the database, but is automatically compiled into main-memory when the
database is opened.   

The definitions are compiled into the module {\bf knowledge\_base}, which
is created if it does not already exist. When a clause stored in the
knowledge base is executed the body is evaluated in that module. Conceptually
the knowledge base stored on disk is part of the module
{\bf knowledge\_base}. 

\index{define/1} \index{update_defs/0} \index{display_defs/0}
Definitions can be loaded in from file by using the {\bf define(File\_Name)}
predicate, where the file contains a list of clauses.
They can be updated and displayed using 
{\bf update\_defs/0} and {\bf display\_defs/0} respectively. The 
{\bf update\_defs/0} predicate calls up the editor and allows the user 
to edit the definitions.  When the editor is exited the definitions are 
then compiled into module {\bf knowledge\_base}. Note that such changes
of the definitions are only possible in single user mode.
The definitions file is not subject to the recovery mechanism. 

An example of a database's definition file is given below

\begin{verbatim}

work_day(monday).
work_day(tuesday).
work_day(wednesday).
work_day(thursday).
work_day(friday).
week_end(saturday).
week_end(sunday).

day(Day) :-
        work_day(Day).
day(Day) :-
        week_end(Day).


employee(Name,Age,Position) :-  
        retrieve_clause(( employee(Name,Age,Position) :- Body )),
	call(Body).
\end{verbatim}

The last clause is one created by {\bf define\_implicit/1}. Such clauses
should not be added manually to the definitions file to prevent 
duplicate definitions.


\subsection{Displaying the Contents of a Relation}

There are two built-ins for displaying the full 
contents of a relation on the screen.  To list  all the 
clauses in a relation the {\bf isdr(Relation\_Name)} predicate is used,
\index{isdr/1} \index{expand/1}
and to print the expanded version of the relation (i.e. with all the 
deductive clauses evaluated) the {\bf expand(Relation\_Name)}
predicate is used. For example:

\begin{verbatim}
?- isdr flight.

RELATION : flight    [real name: flight]

ARITY: 4

ATTRIBUTES :
    + from
    + to
    day
    time

CLAUSES :
flight(munich, frankfurt, _g286, 800) :-
    work_day(_g286).
flight(munich, frankfurt, _g372, 1000) :-
    week_end(_g372).

NUMBER OF CLAUSES : 2



?- expand flight.

RELATION : flight    [real name: flight]

ARITY: 4

ATTRIBUTES :
    + from
    + to
    day
    time

CLAUSES :
flight(munich, frankfurt, monday, 800).
flight(munich, frankfurt, tuesday, 800).
flight(munich, frankfurt, wednesday, 800).
flight(munich, frankfurt, thursday, 800).
flight(munich, frankfurt, friday, 800).
flight(munich, frankfurt, saturday, 1000).
flight(munich, frankfurt, sunday, 1000).

NUMBER OF CLAUSES : 7
\end{verbatim}

\section{Data Manipulation - Relational Algebra}
\index{relational algebra}

\subsection{Retrieving and Inserting Clauses}
To insert and retrieve clauses from a deductive relation
using relational algebra there are the \verb-<+++/2-, {\bf isdr/2}
and {\bf expand/2} predicates.  \verb-<+++/2- performs the task of
\index{$<+++$/2} \index{isdr/2} \index{expand/2}
retrieving the clauses from existing relations that satisfy a relational
expression and inserting them into another predefined permanent
relation.  {\bf isdr/2} and {\bf expand/2} perform exactly the same
process but the retrieved clauses are put into a new 
relation.



The {\bf isdr} and {\bf expand} predicates were introduced
above in their single argument form for displaying 
the contents of a relation. In the same way that {\bf expand/1}
differs from {\bf isdr/1} by expanding deductive clauses
into an equivalent set of unit clauses before displaying the
contents of a relation, 
the {\bf expand/2} differs from {\bf isdr/2} by
taking the deductive clauses generated from the relational expression
and instead of storing them directly expands them first.


The syntax for the three predicates is:

\begin{verbatim}

Relation_Name isdr Relational_Expression.

Relation_Name expand Relational_Expression.

Relation_Name <+++ Relational_Expression.

\end{verbatim}

For {\bf isdr} and {\bf expand} the {\bf Relation\_Name} can either
be an atom or a variable.  An atom will be used as the name of
a permanent relation, while a variable will result in the system
generating its own name for a temporary relation.
For \verb-<+++- only an atom can be given,
and this must be the name of an existing relation.

The {\bf Relational\_Expressions} are predominantly the same for the
DB and KB predicates, so only a brief summary is given here.  For
greater detail refer to section~\ref{alg1}.  The syntax for the 
{\bf Relational\_Expression} for each of the operators is:
 
\begin{itemize}
\item{Selection}\\
     \{Projection\_List \verb+:^:+\} Relation\_Name1 \{where Condition\}
\item{Union}\\
     \{Projection\_List \verb+:^:+\} Relation\_Name1 :+: Relation\_Name2 
                                                \{where Condition\} 
\item{Difference}\\
     \{Projection\_List \verb+:^:+\} Relation\_Name1 :-: Relation\_Name2
                                                \{where Condition\}  
\item{Join}\\
     \{Projection\_List \verb+:^:+\} Relation\_Name1 :*: Relation\_Name2   
                                               \{where Condition\}
\end{itemize}

where \{\} donates an optional part.
The defaults for when no projection list or conditions are given 
are  `all attributes', and `where true' respectively.



The valid conditions are given in the table~\ref{cond}. Where {\bf Att}
 denotes an attribute, {\bf Const} a constant (either numeric or atom), 
and {\bf Term} an \eclipse term.

\begin{table}
\begin{tabular}{||l|l|l|l||}
\hline
 Type      &  Condition     &  Variation       & Description \\ \hline
 Constant  &  Att == Const       &  (Const == Att)     & Equal to\\
           &  Att $<$ Const      &  (Const $>$ Att)    & Less than\\
           &  Att =$<$ Const     &  (Const $>$= Att)   & Less or equal to\\
           &  Att $>$= Const     &  (Const =$<$ Att)   & Greater or equal to\\
           &  Att $>$ Const      &  (Const $<$ Att)    & Greater than\\ 
           &  Att1 == Att2       &                     & Attributes' equality\\ 
\hline
 Term      &  Att = Term         &  (Term = Att)       & Unify\\
           &  Att @$<$ Term      &  (Term @$>$ Att)    & less than\\
           &  Att @=$<$ Term     &  (Term @$>$= Att)   & less or equal to\\
           &  Att @$>$= Term     &  (Term @=$<$ Att)   & greater or equal to\\
           &  Att @$>$ Term      &  (Term @$<$ Att)    & greater than\\ 
           &  Att1 = Att2        &                     & Unify attributes\\
\hline \hline
 Other     &  true               &                      & Always valid\\ 
           &  Cond1 and Cond2    &                      & logical `and'\\ \hline
\end{tabular}
\caption{ Valid Conditions}
\label{cond}
\end{table}



\eclipse provides two types of comparison, namely `constant'
and `term'.  Constant comparisons are between numerics
or atoms, while term comparisons allow structures
to be compared as well. The term comparisons have the same semantics
as the standard Prolog term comparisons i.e. they rely on the standard
ordering of Prolog terms.

Another important difference between the two types of
comparison, is explained using the following example:

\begin{verbatim}
flight(munich,london,Day,1200)    :- week_end(Day).
flight(munich,london,monday,1400).
flight(munich,paris,friday,Time)  :- time(munich,paris,weekday,Time).
flight(munich,paris,saturday,1800).
\end{verbatim}

When selecting clauses from a relation we set a condition on an attribute.  
If the attribute has a fixed value (i.e. not deduced), as in the case of 
the first two arguments of the example clauses, then a simple comparison is 
done between this value and the term specified in the condition.
If, on the other hand, the attribute is a variable (e.g. {\bf Day} or 
{\bf Time}) then its valid instantiations have to be derived first, and 
then a comparison is made with each instantiation.

The `term' type conditions can cope with both cases while the `constant' 
type can only deal with the first. The reason for keeping both types is 
that if we know that an argument is fixed for all clauses in a relation we 
can achieve greater selection speeds using the `constant' type conditions 
on that argument than we would get with the more general `term' type.

Some example queries are:

\begin{verbatim}
X isdr flight
           where to   == paris
           and   time @> 16:00.

X isdr [from, time] :^: flight 
           where to   == london 
           and   day  =  saturday. 
\end{verbatim}

When a {\bf Relational\_Expression} involves attributes with the same name 
but from different relations the ambiguity can be removed by
prefixing them with {\bf Relation\_Name\verb+^+}. An example {\bf where} 
condition is:

\begin{verbatim}
where flight^from == train^from.
\end{verbatim}

\subsection{Deleting Clauses}
\index{$<---$/2}
Sets of clauses can be deleted using the {\tt <---/2} predicate.
The syntax is

\begin{verbatim}
Relation_Name <--- Relational_Expression
\end{verbatim}

where {\bf Relation\_Name} is the name of an existing relation, and
the {\bf Relational\_Expression} is of the same form as described
above. An example is:

\begin{verbatim}
?- employee <--- trial_employee where performance < 6. 
\end{verbatim}

where {\bf employee} and {\bf trial\_employee} are two relations with 
identical schema, and performance is one of their attributes.

Note that only variant clauses are deleted.

% Deductive Database connection

% BEGIN LICENSE BLOCK
% Version: CMPL 1.1
%
% The contents of this file are subject to the Cisco-style Mozilla Public
% License Version 1.1 (the "License"); you may not use this file except
% in compliance with the License.  You may obtain a copy of the License
% at www.eclipse-clp.org/license.
% 
% Software distributed under the License is distributed on an "AS IS"
% basis, WITHOUT WARRANTY OF ANY KIND, either express or implied.  See
% the License for the specific language governing rights and limitations
% under the License. 
% 
% The Original Code is  The ECLiPSe Constraint Logic Programming System. 
% The Initial Developer of the Original Code is  Cisco Systems, Inc. 
% Portions created by the Initial Developer are
% Copyright (C) 2006 Cisco Systems, Inc.  All Rights Reserved.
% 
% Contributor(s): 
% 
% END LICENSE BLOCK

% File		: multiuser-sec.tex
% Date		: March 1992
% Author	: Michael Dahmen
% Modified by	: Luis Hermosilla, August 1992
%		  Joachim Schimpf, July 1994
% Project	: MegaLog-Sepia User Manual
% Content	: The multi user system, recovery, transactions

\newpage

\chapter{Multi User \eclipse}
\label{multi}
\index{multi user}
\index{transaction}
The \eclipse multi user system allows to share databases or 
knowledge bases among several users. Every user is running an \eclipse
process and is able to retrieve and update information.
To the user, multi user database access is similar to single user 
access described in the previous chapters except that any
access of shared relations must be performed within a {\em transaction} context.
This guarantees that the database is kept in a consistent 
state at all times.

Multi User \eclipse is designed for client-server networks.
A database server process manages the database, and the \eclipse client
processes communicate with the server, possibly across a network.
The standard \eclipse configuration allows up to 32 concurrent user
processes per database.  This restriction may be lifted in future releases.

\section{The Database server}

Any database can be used in either single or multi user mode:
\begin{itemize}
\item In single user mode, a single \eclipse process opens the database
exclusively, preventing other users from using it at the same time.
\item In multi user mode, a database server process manages the
database and several \eclipse client processes can use the database
concurrently via this server.
\end{itemize}
The multi user mode is enabled for a database by simply starting
a database server for it.  A server is started with the command 
\begin{quote}\begin{verbatim}
% bang_server /data/base/path &
\end{verbatim}\end{quote}
where the argument specifies the database that the server should control.
The server should be started as background process, which is achieved 
by the \verb+&+ suffix. 
The server must be started before the first user process tries to
open the database (otherwise the first user process would open it
in single user mode).
When there is already a server running, or when the database is
already open in single user mode, an attempt to start a server
results in an error.

The database server may be terminated by the following command
\begin{quote}\begin{verbatim}
% kill -INT pid
\end{verbatim}\end{quote}
where {\tt pid} is the process identification 
number of the database server process.

\section{Accessing a Multi User Database}

After invoking the \eclipse process the user opens a database or
knowledge base as usual, by just specifying the database path name.
The system will check whether the database is controlled
by a database server. If so, it will connect to the server and
switch to multi user mode, otherwise the database is opened in single
user mode.
 
After opening the shared database a user has access to all permanent 
relations, as they are all shared. The temporary relations are private 
per user and invisible to other users. Access to both types of relation 
is provided by the interface described in chapter \ref{database-sec} and
\ref{knowbase-sec}. However, any access to shared relations is only
possible within a transaction context.
If a database access is made outside a transaction context 
an error is raised.
\index{transaction}
\index{shared relation}

A transaction context is established as follows:
\begin{quote}\begin{verbatim}
?- transaction(Goal).
\end{verbatim}\end{quote}
which executes {\bf Goal} as a transaction. Any changes the execution of
the goal makes to the database are only {\em committed} (i.e.
made permanent) if the goal succeeds. This 
gives a transaction an all-or-nothing property,
with either all the updates of the goal being committed to the database for
a valid goal, or the database being left in
the old state (i.e. as before the transaction started)
for an invalid goal. An example:

\begin{quote}\begin{verbatim}
?- openkb(test).

yes
?- transaction(( flight <==> S1, passenger <==> S2 )).
S1 = [+flight_no, +from, +to, time, day].
S2 = [+name, +flight_no].
yes
?- transaction(
       insert_clause(flight(ba100,london,munich,1200,tuesday))
              ).

yes
?- read(Name), read(Flight),	% should not be done inside transactions
   transaction( (insert_clause( passenger(Name,Flight) ),
                 flight(Flight, From, To, Time, Day),  
                 % assuming flight is transparent
                 write('From: '), writeln(From), 
                 write('To:   '), writeln(To), 
                 write('Time: '), writeln(Day), 
                 write('Day:  '), writeln(Time), 
                 write('Confirm (y/n): '),
                 read(Conf), 
                 Conf == y)   ). 

smith.
ba100.
From: london
To:   munich
Time: tuesday
Day:  1200
Confirm (y/n):
\end{verbatim}\end{quote}
This example takes a passenger's name (smith) and flight request
(ba100) and prints the flight details.  If the request is confirmed
the passenger name and flight is added to the passenger relation.
The addition to the database required by the sub-goal 
\begin{quote}\begin{verbatim}
insert_clause( passenger(Name,Flight) )
\end{verbatim}\end{quote}
 will only be committed if 'y.' is entered to confirm the flight booking.  
Otherwise the final sub-goal fails and no database changes are made.

Note that it is not possible to modify the shared part of the database
schema while using a database in the multi user mode. This restriction
might be lifted in future releases.

\section{Concurrency Control}
\index{concurrency}
\eclipse uses the {\em Two Phase Locking} algorithm to control 
concurrent usage of the database.
\index{two phase locking}
Two phase locking uses read and write locks on  
all permanent relations. Before a read or write
operation is performed within a transaction the
corresponding lock is obtained.  This is done during
the first phase of the transaction.  The second phase consists
of only two operations, namely committing or undoing the  
changes and releasing all obtained locks.  Several transactions
can have a read lock on a single item, but if a transaction
has a write lock no other can have any lock on it at the
same time.  When a transaction is unable to
obtain a lock it is suspended until the lock comes free,
and then it is continued.
This strategy guarantees serialisability i.e.\
the result of the interleaved execution is equivalent
to a sequential execution of non-interleaved transactions.
Concurrency control is performed automatically behind
the scenes and there is neither a need nor a possibility for the user
to influence it. Only the lock granularity can be changed by the 
user between relation level and page level locking (see Knowledge Base BIP
Book, {\bf database_parameter/2}).

\section{Deadlock Detection and Handling}
\index{deadlock}
Since a transaction is suspended when a lock cannot
be obtained there is the chance of a {\em deadlock}.
A deadlock is a situation where a set of transactions is
suspended, each waiting for an item locked by another member
of the set.  A deadlock can only be resolved by aborting at 
least one of the transactions. 

The transaction that is aborted when deadlock occurs is called 
the victim.  The strategy for victim selection must prevent 
lifelock. A lifelock happens when the victim is restarted and
leads to the same deadlock as before. A lifelock is prevented in 
\eclipse by always aborting the youngest transaction, and by
fairness of lock distribution (i.e. on a first-come first-served
basis).

When a transaction is aborted due to deadlock an error is raised
and the error handler executes the goal

\begin{quote}\begin{verbatim}
?- exit_block(transaction_abort).
\end{verbatim}\end{quote}
This {\em exit\_block/1} operation aborts the transaction and restarts it.
A transaction is restarted up to 10 times. If it is chosen as victim
10 times another is raised and no further attempt to restart is made.

Before the transaction is restarted all changes done to shared relations
are undone. However, changes to private relations or the Prolog
main memory (e.g. dynamic database, global variable and arrays) are not
undone. Programs that are executed as transactions must therefore be
written in such a way that they can cope with restarts. 
One way to achieve this is to trap the {\em exit\_block/1} operation with
{\em block/3} construct. Another possibility is to remove all temporary
relations at the start of any transaction.

A simple example how to use the {\em block/3} construct is given below.
Let us assume \verb+s1+ and \verb+s2+ are shared relations and \verb+p+ 
a private, all with a single attribute of type atom.

\begin{quote}\begin{verbatim}
unsafe(X) :- ins_tup(s1,X), ins_tup(p,X), ins_tup(s2,X).

?- transaction(unsafe(new_item)).
\end{verbatim}\end{quote}
Such a program is unsafe with respect to transaction abort in the case 
of deadlocks. Let us assume that a deadlock occurs when an attempt is 
made to insert into \verb+s2+ and that this transaction is selected as 
victim. The change done to \verb+s1+ will be undone, but the change 
of \verb+p+ will not, because it is part of the private database. 
A safe version using {\em block/3} looks as follows.

\begin{quote}\begin{verbatim}
safe(X) :- ins_tup(s1,X), 
           block(ins_tup(p,X), 
                 Tag, 
                 ( del_tup(p,X), exit_block(Tag) )),
           ins_tup(s2,X).

?- transaction(safe(new_item)). 
\end{verbatim}\end{quote}
If the same deadlock occurs in this transaction the tuple inserted will
be deleted before the transaction restarts. It is important that 
the {\em block/3} does invoke the {\em exit\_block/1} afterwards, 
otherwise the transaction will not be restarted.

Please note that the example above simplifies the problem, e.g. it
does not handle the case that the insertion was not effective because
the tuple is already stored. In general it is simpler to
initialise the private relations at the start of the transaction,
the method sketched aboved should only used where that approach
is not possible for other reasons.


\section{Recovery}

The capability to undo transactions requires a recovery mechanism.
The \eclipse recovery algorithm does also handle recovery after
a system failure
\footnote{A `system failure' is when there is a hardware or software 
failure that requires a process(es) to be restarted, but does not affect secondary storage.}.
If system failure occurs while \eclipse is executing transactions a 
{\em consistent recovery} is guaranteed. This means that after such 
a failure the database will be left in a consistent state,
with any changes made by aborted or incomplete transactions being undone.  
A {\em shadow page} technique is used by \eclipse to implement the 
recovery procedure.  
\index{recovery}

Recovery only applies to permanent relations and not temporary ones.  
This is because temporary relations are conceptually intended to last 
for just the life of the transaction in which they are produced, and 
therefore recovery is unnecessary.  In actual fact temporary relations
continue to exist until the end of the owner's \eclipse session. 
This provides a useful way of passing sets of tuples or clauses 
outside a transaction. 

Recovery after a system failure is also provided in single user \eclipse. 
The {\em transaction/1} predicate does exist in all variants and 
defines the unit of recovery.
Note that recovery from system failure is only guaranteed to work if
the operating system supports acknowledge disk writes and the
controlling \eclipse parameter is turned on (see Knowledge Base BIP Book,
{\bf database_parameter/2}).

\section{Old \& New States}

During a transaction a permanent relation has two states. 
The old state is the state of the relation before the transaction starts 
and the new state is the state of the relation
after all the changes of previous subgoals have been made. 
Within a transaction the new
state is always used (unless the old state is explicitly selected), and when 
the transaction completes successfully the changes are 
committed and the old state becomes equal to the new state.

To obtain the old state within a transaction the operator {\bf old} is
used in the following way:
\index{old/1}
\begin{quote}\begin{verbatim}
old(RelationName)
\end{verbatim}\end{quote}
where {\bf RelationName} is the name of a relation.
Therefore a predicate can be directed to work with the
old state of a relation by adding the old operator
to the relation's name.  An example is

\begin{quote}\begin{verbatim}
1 ?- transaction( digits <++ [ 1,2,3,4,5 ] ).

yes
2 ?- transaction( ( ins_tup( digits(6) ), 
                    findall(X,retr_tup( old(digits), X), L))).
L = [[1], [2], [3], [4], [5]]
X = _g16

yes
3 ?- transaction( findall(X,retr_tup(digits,X),L) ).
L = [[1], [2], [3], [4], [5], [6]]
X = _g4

yes
\end{verbatim}\end{quote}
The second transaction inserts a tuple into the digits relation, 
and then generates a list of all the tuples in that relation.
Since the {\bf retr\_tup} predicate is directed to work with the old
state the newly added tuple does not appear in the list.


Even though the new state of a predicate is used by default a 
new operator is included for completeness i.e.
\index{new/1}
\begin{quote}\begin{verbatim}
new(RelationName)
\end{verbatim}\end{quote}
The old state exists both in the DB and the KB version, however, in
the KB version access is a bit tricky. An expression like
\begin{quote}\begin{verbatim}
retrieve_clause(( (old(name))(Arg1, Arg2) :- Body ))
\end{verbatim}\end{quote}
is {\em not} legal Prolog. One must therefore first introduce a synonym
\begin{quote}\begin{verbatim}
old(name) <--> old_name
retrieve_clause(( old_name(Arg1, Arg2) :- Body ))
\end{verbatim}\end{quote}

\section{Deterministic Transactions}

Since a transaction must release all its locks on completion
it cannot backtrack.  Therefore {\bf transaction/1} will
succeed at most once (i.e. it is deterministic).
This does not limit the use of the transaction primitive
as the {\bf findall} predicate can be used to collect
sets of solutions (see previous example).
Also temporary relations can be used.  As mentioned
above these conceptually die when their transaction dies,
but since it is a useful way of passing sets of clauses
from a transaction (which can then be used for backtracking
{\em outside} a transaction) they are allowed to live until
the end of the session.


% Features to support multi-user MegaLog

% BEGIN LICENSE BLOCK
% Version: CMPL 1.1
%
% The contents of this file are subject to the Cisco-style Mozilla Public
% License Version 1.1 (the "License"); you may not use this file except
% in compliance with the License.  You may obtain a copy of the License
% at www.eclipse-clp.org/license.
% 
% Software distributed under the License is distributed on an "AS IS"
% basis, WITHOUT WARRANTY OF ANY KIND, either express or implied.  See
% the License for the specific language governing rights and limitations
% under the License. 
% 
% The Original Code is  The ECLiPSe Constraint Logic Programming System. 
% The Initial Developer of the Original Code is  Cisco Systems, Inc. 
% Portions created by the Initial Developer are
% Copyright (C) 2006 Cisco Systems, Inc.  All Rights Reserved.
% 
% Contributor(s): 
% 
% END LICENSE BLOCK

% File		: backwards-sec.tex
% Date		: March 1992
% Author	: Michael Dahmen, Luis Hermosilla
% Modified by	: Luis Hermosilla, August
% Project	: MegaLog-Sepia User Manual
% Content	: Backwards compatibility wrt. MegaLog

\newpage

\chapter{Backwards Compatibility}
\label{backwards-compat}

\eclipse is backwards compatible to the previous
MegaLog system with a Prolog part independent of \eclipse. This
compatibility is achieved by a special module that redefines all
predicates with conflicting syntax or semantics.

\section{Module {\bf megalog}}

This module
achieves a very high degree of compatibility, but there are a
few aspects remaining where minor changes in an application
may be necessary.

\begin{itemize}

\item{Arithmetic}

The hyperbolic radial functions are missing in \eclipse. The 
expression 
\begin{verbatim}
?- X is cputime.
\end{verbatim}
produces the current time in \eclipse, not the time since the
last call. However, {\bf cputime/1} and related built-ins 
have the semantics as before.

\item{Asserted Clauses}

The errors generated when asserted clauses are overwritten with compiled
code or vice versa differ.

\item{Global Variable and Arrays}

The errors generated differ. In general, \eclipse catches more
errors.

\item{Error Handling}

The error codes are completely different. The compatibility module
forbids definition of error handlers to prevent undesired effects.

\item{Input and Output}

The difference in argument order is handled by the compatibility module.
\eclipse does not detect cyclic terms when printing, however by
default there is a maximal depth and length set. \eclipse does
not flush after each output. The predicates {\bf set\_io/2},
{\bf writeqvar/2} and {\bf writeqvar/3} are not provided.

\item{Object Files}

The capability to generate and load object files is not provided.
However, there is the capability to generate and load saved states.

\item{String Conversion}

The predicate {\bf term\_string/3} is not provided, however
{\bf term\_string/2} is.

\item{Environment}

There are different predicates to
obtain statistics. The on-line editor is not provided.

\item{Structures}

\eclipse does not distinguish structures and lists, i.e.\ where ever
possible a list is seen as {\bf ./2}. The compatibility module 
redefines {\bf compound/1}, {\bf list/1} and {\bf type\_of/2} to 
achieve the MegaLog semantics. However, {\bf arg/3}, {\bf functor/3} and
{\bf =../2} are not redefined i.e.\ they generate no errors in cases 
where they did before.

\item{Debugger}

The debugger is different and also some of the predicates that
control the debugger.

\item{Command Line Options}

\eclipse has different command line options, which are based on the
Sepia ones. Please refer to the Sepia manual in order to find the
appropriate replacement.

\end{itemize}

More details can be looked up in the source file of module
{\bf megalog}, which is extensively documented.


\section{Shared Memory Multi User System}

The "shared memory" multi user variant of MegaLog and
earlier releases of \eclipse is no longer supported.


% compatibility with previous MegaLog

% \include {profile-sec}
% Profiler manual

\chapter{The Built-In Predicates}
\label{bip-summary}

This chapter provides a summary of each of the built-in 
predicates available in the database and knowledge base of \eclipse. The
built-ins are listed 
in alphabetical order within each type of built-in. The information
provided here is only a one-line description for each built-in.
A more complete specification can be found by
referring to the Knowledge Base Built-In Specification Manual (or Knowledge
Base BIP Book), 
\cite{BIP92}. The Knowledge Base BIP Book is on-line available by the
predicate {\bf help/1}.

{\bf help(Functor/Arity)} prints the page from the Knowledge Base BIP Book
describing the  
built-in {\bf Functor/Arity} to the current output. If only {\bf Functor}
is specified, the system looks for all predicates that have this atom in 
their name. If there are more than one predicate
matching, a short description for each one is printed, rather than the
whole page. There is also a glossary, where each page does not describe
a single built-in but a concept that is used in other descriptions.
Examples :

\begin{verbatim}
% display page from Knowledge Base BIP Book
?- help(createdb/1).

% list a short description of all built-ins matching "relation"
?- help(relation).    

% list a short description of all glossary entries
?- help(glossary).

% display glossary on Attribute Specification
?- help(attribute_specification).
\end{verbatim}

The built-ins are classified into the following types,
which correspond to the defining module.

\paragraph{List of built-in types:}

\begin{enumerate}

\item{\eclipse Database Kernel}
\item{\eclipse DB}
\item{\eclipse KB}

\end{enumerate}

% BEGIN LICENSE BLOCK
% Version: CMPL 1.1
%
% The contents of this file are subject to the Cisco-style Mozilla Public
% License Version 1.1 (the "License"); you may not use this file except
% in compliance with the License.  You may obtain a copy of the License
% at www.eclipse-clp.org/license.
% 
% Software distributed under the License is distributed on an "AS IS"
% basis, WITHOUT WARRANTY OF ANY KIND, either express or implied.  See
% the License for the specific language governing rights and limitations
% under the License. 
% 
% The Original Code is  The ECLiPSe Constraint Logic Programming System. 
% The Initial Developer of the Original Code is  Cisco Systems, Inc. 
% Portions created by the Initial Developer are
% Copyright (C) 2006 Cisco Systems, Inc.  All Rights Reserved.
% 
% Contributor(s): 
% 
% END LICENSE BLOCK

% File		: kernel-lst.tex  
% Date		: March 1992
% Author	: Michael Dahmen (edit of automatic generated file)
% Modified by	: Luis Hermosilla, August 1992
% Project	: MegaLog-Sepia User Manual
% Content	: predicates of module Database Kernel

\section{\eclipse Database Kernel}

The following predicates are exported by the module {\bf database\_kernel}.

%\biponelinesection{\eclipse Database Kernel}
\label{kernel-sub}

\begin{description}
\item[bang_arity(+RelationName, ?Arity)]{Get the number of attributes in a relation.}
\index{bang_arity/2}
\item[bang_attribute(+RelationName, +Position, ?Spec)]{Get the specification of an attribute in a relation.}
\index{bang_attribute/3}
\item[bang_cardinality(+RelationName, ?Count)]{Get the number of tuples in a relation.}
\index{bang_cardinality/2}
\item[bang_createrel(?RelationName, ?SpecList, +OptionList)]{Create a relation.}
\index{bang_createrel/3}
\item[bang_delete(+RelationName, +Condition)]{Delete all tuples in relation that are selected by condition.}
\index{bang_delete/2}
\item[bang_delete_tup(+RelationName, +Tuple)]{Delete all occurrences of the tuple from the relation.}
\index{bang_delete_tup/2}
\item[bang_delete_tup(+RelationName, +Tuple, -Status)]{Delete all occurrences of the tuple from the relation and return a status.}
\index{bang_delete_tup/3}
\item[bang_destroyrel(+RelationName)]{Destroy a relation i.e. remove it from the database.}
\index{bang_destroyrel/1}
\item[bang_diff(+Rel1, +Rel2, +Condition, +Projection, ?RelOut)]{Relational difference operation, also known as complementary join.}
\index{bang_diff/5}
\item[bang_diff(+Rel1, +Rel2, +Condition, +Projection, ?RelOut, +Action)]{Relational difference operation, also known as complementary join.}
\index{bang_diff/6}
\item[bang_exist(+RelationName, +Condition)]{Succeeds if there exist at least one tuple in the relation that fulfills the selection condition.}
\index{bang_exist/2}
\item[bang_existrel(+RelationName)]{Succeed if the relation exists.}
\index{bang_existrel/1}
\item[bang_format(+RelationName, ?SpecList)]{Obtain the list of attribute specification.}
\index{bang_format/2}
\item[bang_format(+RelationName, ?SpecList, ?Arity)]{Obtain the list of attribute specification and the arity of a relation.}
\index{bang_format/3}
\item[bang_free_cursor]{Closes all tuple retrievals that are currently open.}
\index{bang_free_cursor/0}
\item[bang_insert(+RelationName, +Tuple)]{Insert a tuple into a relation.}
\index{bang_insert/2}
\item[bang_insert(+RelationName, +Tuple, -Status)]{Insert a tuple into a relation and return a status.}
\index{bang_insert/3}
\item[bang_join(+Rel1, +Rel2, +Condition, +Projection, ?RelOut)]{Relational join operation.}
\index{bang_join/5}
\item[bang_join(+Rel1, +Rel2, +Condition, +Projection, ?RelOut, +Action)]{Relational join operation.}
\index{bang_join/6}
\item[bang_recover(+DatabasePath)]{Recover a consistent database after a failed multi-database transaction.}
\index{bang_recover/1}
\item[bang_register(+Flag, ?Value)]{Obtain several internal values of the database.}
\index{bang_register/2}
\item[bang_renamerel(+OldName, +NewName)]{Change the name of a relation.}
\index{bang_renamerel/2}
\item[bang_renamerel(+OldName, +NewName, +NewAttributes)]{Change the name and/or the attribute names of a relation.}
\index{bang_renamerel/3}
\item[bang_retrieve(+RelationName, ?Tuple, +Condition)]{Tuple at a time retrieval. On backtracking the next tuple is retrieved.}
\index{bang_retrieve/3}
\item[bang_retrieve_delete(+RelName, ?Tuple, +Condition)]{Tuple at a time retrieval with deletion. On backtracking the next tuple is retrieved and deleted.}
\index{bang_retrieve_delete/3}
\item[bang_retrieve_delete(+RelName, ?Tuple, +Condition, +Test)]{Tuple at a time retrieval with deletion. On backtracking the next tuple is retrieved and deleted.}
\index{bang_retrieve_delete/4}
\item[bang_retrieve_list(+RelationName, +Condition, ?List)]{Retrieve the list of all tuples that satisfy the given condition.}
\index{bang_retrieve_list/3}
\item[bang_select(+Rel, +Condition, +Projection, ?RelOut)]{Relational selection and projection operation.}
\index{bang_select/4}
\item[bang_select(+Rel, +Condition, +Projection, ?RelOut, +Action)]{Relational selection and projection operation.}
\index{bang_select/5}
\item[closedb]{Close a database and remove all temporary relations. Fails if no database open.}
\index{closedb/0}
\item[closedb(+DBhandle)]{Close the open database specified by {\it DBhandle} and remove all temporary relations. Fails if no database is open.}
\index{closedb/1}
\item[createdb(+DatabasePath)]{Create a new database or open a database if it already exists.}
\index{createdb/1}
\item[createdb(+DatabasePath, ?DBhandle)]{Create or open a database and assign a database handle.}
\index{createdb/2}
\item[current_relation(?RelationName)]{Succeeds if {\it RelationName} is a permanent relation in the database.}
\index{current_relation/1}
\item[current_temp_relation(?RelationName)]{Succeeds if {\it RelationName} is a temporary relation in the database.}
\index{current_temp_relation/1}
\item[current_time(+TimerName, ?Value)]{Get the current time from an internal timer.}
\index{current_time/2}
\item[delta_time(+TimerName, ?Delta)]{Get the time since the last call from an internal timer.}
\index{delta_time/2}
\item[destroy_temprels]{Destroy all temporary relations in the database.}
\index{destroy_temprels/0}
\item[destroydb]{Destroy the currently open database.}
\index{destroydb/0}
\item[destroydb(+DBhandle)]{Destroy the currently open database specified by {\it DBhandle}.}
\index{destroydb/1}
\item[database_parameter(+Name, ?Value)]{Set or retrieve the value of an internal parameter.}
\index{database_parameter/2}
\item[opendb(+DatabasePath)]{Open a database.}
\index{opendb/1}
\item[opendb(+DatabasePath, ?DBhandle)]{Open a database and assign a database handle to it.}
\index{opendb/2}
\item[resource(?PageReclaims, ?PageFaults, ?Swaps)]{Get information about paging and swapping from the operating system.}
\index{resource/3}
\item[statistics_bang]{Print information about activity of the page buffer management.}
\index{statistics_bang/0}
\item[statistics_desc]{Print information about internal management of relations and corresponding UNIX files.}
\index{statistics_desc/0}
\item[statistics_lock]{Print information about lock management.}
\index{statistics_lock/0}
\item[statistics_relation(+RelationName)]{Print information on the pages of {\it RelationName} that are currently in the page buffers.}
\index{statistics_relation/1}
\item[transaction(+Goal)]{Meta calls {\it Goal} in a transaction context.}
\index{transaction/1}
%\item[transaction_commit]{Commit a transaction started with {\pbf transaction_start/0}.}
%\index{transaction_commit/0}
%\item[transaction_start]{Start a transaction.}
%\index{transaction_start/0}
%\item[transaction_undo]{Undo a transaction started with {\pbf transaction_start/0}.}
%\index{transaction_undo/0}
\end{description}


% database predicates

% BEGIN LICENSE BLOCK
% Version: CMPL 1.1
%
% The contents of this file are subject to the Cisco-style Mozilla Public
% License Version 1.1 (the "License"); you may not use this file except
% in compliance with the License.  You may obtain a copy of the License
% at www.eclipse-clp.org/license.
% 
% Software distributed under the License is distributed on an "AS IS"
% basis, WITHOUT WARRANTY OF ANY KIND, either express or implied.  See
% the License for the specific language governing rights and limitations
% under the License. 
% 
% The Original Code is  The ECLiPSe Constraint Logic Programming System. 
% The Initial Developer of the Original Code is  Cisco Systems, Inc. 
% Portions created by the Initial Developer are
% Copyright (C) 2006 Cisco Systems, Inc.  All Rights Reserved.
% 
% Contributor(s): 
% 
% END LICENSE BLOCK

% File		: database-lst.tex
% Date		: March 1992
% Author	: Michael Dahmen (edit of automatic generated file)
% Modified by	: Luis Hermosilla, August 1992
% Project	: MegaLog-Sepia User Manual
% Content	: predicates of module Relational Algebra

\section{\eclipse DB}

The following predicates are exported by the module {\bf db}.

%\biponelinesection{\eclipse DB}

\begin{description}
\item[+Expression \hspace{1mm} ++$>$ \hspace{1mm} ?TupleList]{Set retrieval specified by an algebraic expression.}
\index{$++>$/2}
\item[+RelationName \hspace{1mm} $<$++ \hspace{1mm} +Expression]{Set insertion of an algebraic expression or a list of tuples.}
\index{$<++$/2}
\item[+RelationName \hspace{1mm} $<$$-$$-$ \hspace{1mm} +Expression]{Set deletion of an algebraic expression or a list of tuples.}
\index{$<--$/2}
\item[+RelationName1 \hspace{1mm} $<$@$>$ \hspace{1mm} +RelationName2]{Test for schema compatibility.}
\index{$<@>$/2}
\item[?RelationName \hspace{1mm} $<$$-$$>$ \hspace{1mm} ?Synonym]{Add, remove or query synonyms for relations.}
\index{$<->$/2}
\item[?RelationName \hspace{1mm} $<$=$>$ \hspace{1mm} ?SpecList)]{Create or remove a relation. Query relation format.}
\index{$<=>$/2}
\item[?RelationName isr +Expression]{Set evaluation of an algebraic expression and creation of a target relation.}
\index{isr/2}
\item[arity(+RelationName, ?Arity)]{Get the number of attributes in a relation.}
\index{arity/2}
\item[cardinality(+RelationName, ?Count)]{Get the number of tuples in a relation.}
\index{cardinality/2}
\item[del_tup(?TupleTerm)]{Tuple at a time retrieval with deletion. On backtracking the next tuple is retrieved and deleted.}
\index{del_tup/1}
\item[del_tup(+RelationName, ?Tuple)]{Tuple at a time retrieval with deletion. On backtracking the next tuple is retrieved and deleted.}
\index{del_tup/2}
\item[del_tup(+RelationName, ?Tuple, +Selection)]{Tuple at a time retrieval with deletion. On backtracking the next tuple is retrieved and deleted.}
\index{del_tup/3}
\item[helpdb]{Print a list of all relations to the standard output.}
\index{helpdb/0}
\item[helprel(+RelationName)]{Print some info on {\it RelationName} to the standard output.}
\index{helprel/1}
\item[ins_tup(+TupleTerm)]{Insert a tuple into a relation.}
\index{ins_tup/1}
\item[ins_tup(+RelationName, +Tuple)]{Insert a tuple into a relation.}
\index{ins_tup/2}
\item[+DBoperation ondb +DBhandle]{Apply {\it DBoperation} to the database specified by {\it DBhandle}.}
\index{ondb/2}
\item[printrel(+RelationName)]{Print some info on {\it RelationName} and the content of {\it RelationName} to the standard output.}
\index{printrel/1}
\item[rename_attributes(+RelationName, +NewAttributes)]{Change the attribute names of a relation.}
\index{rename_attributes/2}
\item[rename_relation(+OldName, +NewName)]{Change the name a of a relation.}
\index{rename_relation/2}
\item[retr_tup(?TupleTerm)]{Tuple at a time retrieval. On backtracking the next tuple is retrieved.}
\index{retr_tup/1}
\item[retr_tup(+RelationName, ?Tuple)]{Tuple at a time retrieval. On backtracking the next tuple is retrieved.}
\index{retr_tup/2}
\item[retr_tup(+RelationName, ?Tuple, +Selection)]{Tuple at a time retrieval. On backtracking the next tuple is retrieved.}
\index{retr_tup/3}
\end{description}

% relational algebra predicates

% BEGIN LICENSE BLOCK
% Version: CMPL 1.1
%
% The contents of this file are subject to the Cisco-style Mozilla Public
% License Version 1.1 (the "License"); you may not use this file except
% in compliance with the License.  You may obtain a copy of the License
% at www.eclipse-clp.org/license.
% 
% Software distributed under the License is distributed on an "AS IS"
% basis, WITHOUT WARRANTY OF ANY KIND, either express or implied.  See
% the License for the specific language governing rights and limitations
% under the License. 
% 
% The Original Code is  The ECLiPSe Constraint Logic Programming System. 
% The Initial Developer of the Original Code is  Cisco Systems, Inc. 
% Portions created by the Initial Developer are
% Copyright (C) 2006 Cisco Systems, Inc.  All Rights Reserved.
% 
% Contributor(s): 
% 
% END LICENSE BLOCK

% File		: knowbase-lst.tex
% Date		: March 1992
% Author	: Michael Dahmen (edit of automatic generated file)
% Modified by	: Luis Hermosilla, August 1992
% Project	: MegaLog-Sepia User Manual
% Content	: predicates of module Deductive Relations

\section{\eclipse KB}

The following predicates are exported by the module {\bf kb}.

%\biponelinesection{\eclipse KB}
\begin{description}
\item[+RelationName \hspace {1mm} $<$$-$$-$$-$ \hspace {1mm} +Expression]{Set evaluation of a deductive algebraic expression and deletion from a deductive relation.}
\index{$<---$/2}
\item[+RelationName \hspace {1mm} $<$+++ \hspace {1mm} +Expression]{Set evaluation of a deductive algebraic expression and insertion into a deductive relation.}
\index{$<+++$/2}
\item[+RelationName1 \hspace {1mm} $<$@@$>$ \hspace {1mm} +RelationName2]{Test for schema compatibility.}
\index{$<@@>$/2}
\item[?RelationName \hspace {1mm} $<$$-$$-$$>$ \hspace {1mm} ?Synonym]{Add, remove or query synonyms for relations.}
\index{$<-->$/2}
\item[?RelationName \hspace {1mm} $<$==$>$ \hspace {1mm} ?SpecList)]{Create or remove a relation. Query relation format.}
\index{$<==>$/2}
\item[expand +Expression]{Set evaluation of a deductive algebraic expression and display on standard output.}
\index{expand/2}
\item[?RelationName expand +Expression]{Set evaluation of a deductive algebraic expression and creation of a deductive target relation.}
\index{expand/2}
\item[isdr +Expression]{Set evaluation of a deductive algebraic expression and display on standard output.}
\index{isdr/2}
\item[?RelationName isdr +Expression]{Set evaluation of a deductive algebraic expression and creation of a deductive target relation.}
\index{isdr/2}
\item[closekb]{Close a knowledge base and remove all temporary relations. Fails if no knowledge base is open.}
\index{closekb/0}
\item[createkb(+KnowledgeBasePath)]{Create a new knowledge base or open a knowledge base if it already exists.}
\index{createkb/1}
\item[define(+FileName)]{Add a Prolog source file to the definitions file.}
\index{define/1}
\item[define_implicit(+Relation/+Arity)]{Make a relation transparent.}
\index{define_implicit/1}
\item[degree(+RelationName, ?Degree)]{Get the number of attributes in a relation.}
\index{degree/2}
\item[delete_clause(+Clause)]{Delete a variant clause from a deductive relation.}
\index{delete_clause/1}
\item[destroykb]{Destroy the currently open knowledge base.}
\index{destroykb/0}
\item[display_defs]{Display the definitions file.}
\index{display_defs/0}
\item[helpdrel(+RelationName)]{Print some info on {\it RelationName} to the standard output.}
\index{helpdrel/1}
\item[helpkb]{Print a list of all relations to the standard output.}
\index{helpkb/0}
\item[insert_clause(+Clause)]{Insert a clause into a deductive relation.}
\index{insert_clause/1}
\item[insert_clauses_from(+FileName)]{Insert a file of program clauses into the deductive database.}
\index{insert_clauses_from/1}
\item[openkb(+KnowledgeBasePath)]{Open a knowledge base.}
\index{openkb/1}
\item[retract_clause(?Clause)]{Retract a unifying clause from a deductive relation.}
\index{retract_clause/1}
\item[retrieve_clause(?Clause)]{Retrieve a unifying clause from a deductive relation.}
\index{retrieve_clause/1}
\item[update_defs]{Edit the definitions file.}
\index{update_defs/0}
\end{description}


% knowledge base predicates

\newpage
% BEGIN LICENSE BLOCK
% Version: CMPL 1.1
%
% The contents of this file are subject to the Cisco-style Mozilla Public
% License Version 1.1 (the "License"); you may not use this file except
% in compliance with the License.  You may obtain a copy of the License
% at www.eclipse-clp.org/license.
% 
% Software distributed under the License is distributed on an "AS IS"
% basis, WITHOUT WARRANTY OF ANY KIND, either express or implied.  See
% the License for the specific language governing rights and limitations
% under the License. 
% 
% The Original Code is  The ECLiPSe Constraint Logic Programming System. 
% The Initial Developer of the Original Code is  Cisco Systems, Inc. 
% Portions created by the Initial Developer are
% Copyright (C) 2006 Cisco Systems, Inc.  All Rights Reserved.
% 
% Contributor(s): 
% 
% END LICENSE BLOCK
\begin{theindex}

  \item /2, 8
  \item /3, 8
  \item /4, 8

  \indexspace

  \item eplex, 5

  \indexspace

  \item fd, 5

  \indexspace

  \item ic, 5

  \indexspace

  \item lib(graph\_algorithms), 7
  \item lib(java\_vc), 9
  \item lib(viewable), 3

  \indexspace

  \item range, 5
  \item ria, 5

  \indexspace

  \item start\_vc/1, 9

  \indexspace

  \item viewable, 3--11, 13--16, 19
  \item viewable\_create/2, 4, 5, 10
  \item viewable\_create/3, 5, 10
  \item viewable\_create/4, 6, 7, 10
  \item viewable\_expand/3, 5, 10
  \item viewable\_expand/4, 6, 10

  \indexspace

  \item write/1, 11

\end{theindex}

\newpage

\bibliography{sepiachip}

\end{document}
