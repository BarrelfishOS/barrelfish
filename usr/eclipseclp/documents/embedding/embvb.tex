% BEGIN LICENSE BLOCK
% Version: CMPL 1.1
%
% The contents of this file are subject to the Cisco-style Mozilla Public
% License Version 1.1 (the "License"); you may not use this file except
% in compliance with the License.  You may obtain a copy of the License
% at www.eclipse-clp.org/license.
% 
% Software distributed under the License is distributed on an "AS IS"
% basis, WITHOUT WARRANTY OF ANY KIND, either express or implied.  See
% the License for the specific language governing rights and limitations
% under the License. 
% 
% The Original Code is  The ECLiPSe Constraint Logic Programming System. 
% The Initial Developer of the Original Code is  Cisco Systems, Inc. 
% Portions created by the Initial Developer are
% Copyright (C) 2006 Cisco Systems, Inc.  All Rights Reserved.
% 
% Contributor(s): 
% 
% END LICENSE BLOCK
%
% $Id: embvb.tex,v 1.1 2006/09/23 01:49:04 snovello Exp $
%

\chapter{Embedding into Visual Basic}
\label{chapvb}

This is a set of Visual Basic classes built around the scripting
language interface of {\eclipse}. They are provided in source form
in directory doc/examples/vb within the installation.
The interface is similar in spirit to the {\eclipse} embedding
interfaces for other languages.


\section{The EclipseThread Project}
\index{EclipseThread}
This contains the classes which form the interface to eclipse.
\begin{description}
\item[class		EclipseClass]\ \\
	An object of this class is a thread running eclipse code. Only
	one such object may exist in a process.

\item[class		EclipseStreams]\ \\
	A collection of queue streams for communicating with {\eclipse}.

\item[class		EclipseStream]\ \\
	A stream of on which data can be sent to or from {\eclipse}.

\end{description}

\section{Public Enumerations}
\begin{description}

\item[Enum EC_Status]\ \\
	Symbolic names for the status values returned by goal execution.

\item[Enum EclipseStreamMode]\ \\
	Symbolic names for the direction in which data flows within
	an EclipseStream.

\item[Enum EC_ERROR]\ \\
	Symbolic names for error conditions in the interface.

\end{description}

\section{The EclipseClass class}
\index{EclipseClass}
An object of this class is an entity running eclipse code. Only
one such object may exist in a process.

The class provides methods to execute goals and to access queue streams
to communicate with the running goal.

\begin{description}
\item[Function Init() As Long]\ \\
	Initialise the {\eclipse} engine. This is required before any other
	functions of this interface can be used.

\item[Sub Send(EventName As String)]\ \\
	Post an event to the {\eclipse} engine. This will lead to the
	execution of the corresponding event handler
	See also \bipref{event/1}{../bips/kernel/event/event-1.html} and the User Manual
	chapter on event handling for more information. The event name
	is given as a string. Note that if {\eclipse} was not running,
	the event stays in its queue until it is resumed.

\item[Function Post(Goal As String) As EC_Status]\ \\
	Post a goal (constraint) that will be executed when \eclipse
	is resumed.  The goal is given as a string in {\eclipse} syntax.

\item[Function ResumeAsync() As EC_Status]\ \\
	Resume execution of the {\eclipse} engine: All posted goals will
	be executed. The return value will be 'Success' if the goals succeed
	'Fail' is returned if the goals fail, and 'Yield' if control was
	yielded because of a
	\bipref{yield/2}{../bips/kernel/externals/yield-2.html}
	predicate call in the {\eclipse} code.
	No parameters can be passed.

	The function returns when the posted goals have finished executing.
	Since a separate thread is actually executing the goals though,
	events may be received by the Visual Basic program during the
	execution of this function. It is an error to call this function
	recursively while handling one of these events.

\item[Function HandleEvents() As EC_Status]\ \\
	Resume execution of the {\eclipse} engine, but do not let it execute
	any posted goals. Only {\eclipse} events will be handled. Sources
	of such events are the Post() Function or writing to an event-raising
	{\eclipse} queue stream.
	The function returns when the events have all been handled by
	{\eclipse} and the return value is 'Success'.
	It is an error to call this function while a ResumeAsync()
	is still active.

\item[Sub RPC(Goal As Variant, Response As Variant)]\ \\
        {\eclipse} Remote Predicate Call.
	An {\eclipse} goal is constructed from {\it Goal}
	according to the conversion rules explained in chapter \ref{chapexdr}.
	Goal is called in the default module.  The Goal should be
	simple in the sense that it can only succeed, fail or throw.
	It must not call
	\bipref{yield/2}{../bips/kernel/externals/yield-2.html}.  Any
	choicepoints the goal leaves will be discarded.  On success,
	the (possibly more instantiated) Goal is returned as Response,
	otherwise "fail" or "throw" respectively.

	Unlike {\bf ResumeAsync}, calls to {\bf RPC} can be nested
	and can be used from within VB Stream event handlers.

\item[Property Let EclipseDir(Dir As String)]\ \\
	The directory where {\eclipse} is installed.
	See Chapter \ref{chapecoptions}.

\item[Property Let Module(Mod As String)]\ \\
	The default module for calling goals.
	See Chapter \ref{chapecoptions}.

\item[Property Let GlobalSize(Size As Long)]\ \\
	The maximum size of the {\eclipse} global/trail stack in bytes.
	See Chapter \ref{chapecoptions}.

\item[Property Let LocalSize(Size As Long)]\ \\
	The maximum size of the {\eclipse} local/control stack in bytes.
	See Chapter \ref{chapecoptions}.

\item[Property Let SharedSize(Size As Long)]\ \\
	The maximum size of the {\eclipse} shared heap.
	See Chapter \ref{chapecoptions}.

\item[Property Let PrivateSize(Size As Long)]\ \\
	The maximum size of the {\eclipse} private heap.
	See Chapter \ref{chapecoptions}.

\item[Property Get Streams]\ \\
	The EclipseStreams collection associated with this EclipseClass.
\end{description}


\section{The EclipseStreams Collection Class}
\index{EclipseStreams}
This is a collection of EclipseStream objects. The keys to this collection
are the symbolic name of {\eclipse} streams. Initially it
will contain the 'input' 'output' and 'error' streams.

\begin{description}
\item[Function Add(Key As String, Mode As EclipseStreamMode) As EclipseStream]\ \\
	Create a new EclipseStream. 'Key' must be the symbolic name of an
	existing {\eclipse} queue stream. These are created using the
	\bipref{open/3}{../bips/kernel/iostream/open-3.html} or
	\bipref{open/4}{../bips/kernel/iostream/open-4.html}
	built-in. If 'Mode' is 'FromEclipse' the {\eclipse} stream must
	have been opened in 'write' mode. If it is 'ToEclipse' the \eclipse
	stream must have been opened in read mode.

\item[Property Get Item(vntIndexKey As Variant) As EclipseStream]\ \\
	Used to retrieve streams from the collection. 'vntIndexKey' can
	be either the symbolic steam name or an integer index into the
	collection.

\item[Property Get Count() As Long]\ \\
	The number of items in the collection

\item[Sub Remove(vntIndexKey As Variant)]\ \\
	Remove an EclipseStream from the collection. (This does not
	close the corresponding {\eclipse} stream though).

\end{description}


\section{The EclipseStream Class}
\index{EclipseStream}
This class allows exchanging data with an embedded {\eclipse} via
queue streams created by the {\eclipse} code.
\begin{description}
\item[Event Flush]\ \\
	Raised whenever the {\eclipse} program flushes this stream.

\item[Property Get Key() As String]\ \\
	The symbolic name of this stream

\item[Property Get Mode() As EclipseStreamMode]\ \\
	The direction in which data is sent over this EclipseStream

\item[Property Get/Let Prompt() As String]\ \\
	A prompt string. This appears in an input box that pops up
	when the {\eclipse} program attempts to read from a queue
	stream if no data is available.

\item[Sub StreamWrite(Data As String)]\ \\
	Send 'Data' over this stream.

\item[Function Read(l As Long) As String]\ \\
	Receives at most 'l' characters from the EclipseStream.
	No flushing is necessary.

\item[Function NewData() As String]\ \\
	Receives all available characters from the EclipseStream
	that has been written on the stream since the last flush.

\item[Sub WriteExdr(Data As Variant)]\ \\
	Writes the given data structure onto the stream in EXDR-encoded form.
	See chapter \ref{chapexdr} for details about EXDR format.

\item[Sub ReadExdr(Data As Variant)]\ \\
	Reads one EXDR-encoded term from the stream and returns its
	VB-representation in Data.
	See chapter \ref{chapexdr} for details about EXDR format.
\end{description}

