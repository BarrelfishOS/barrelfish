% BEGIN LICENSE BLOCK
% Version: CMPL 1.1
%
% The contents of this file are subject to the Cisco-style Mozilla Public
% License Version 1.1 (the "License"); you may not use this file except
% in compliance with the License.  You may obtain a copy of the License
% at www.eclipse-clp.org/license.
% 
% Software distributed under the License is distributed on an "AS IS"
% basis, WITHOUT WARRANTY OF ANY KIND, either express or implied.  See
% the License for the specific language governing rights and limitations
% under the License. 
% 
% The Original Code is  The ECLiPSe Constraint Logic Programming System. 
% The Initial Developer of the Original Code is  Cisco Systems, Inc. 
% Portions created by the Initial Developer are
% Copyright (C) 2006 Cisco Systems, Inc.  All Rights Reserved.
% 
% Contributor(s): Joachim Schimpf, IC-Parc
% 
% END LICENSE BLOCK
%
% $Id: embexdr.tex,v 1.1 2006/09/23 01:48:59 snovello Exp $
%
% Author:	Joachim Schimpf, IC-Parc
%

%----------------------------------------------------------------------
\chapter{EXDR Data Interchange Format}
\label{chapexdr}
%HEVEA\cutdef[1]{section}
%----------------------------------------------------------------------

We have defined a data interchange format called EXDR for the
communication between {\eclipse} and other languages.  The data types
available in this format are integer, double, string, list, nil,
structure and anonymous variable.  This is intended to be the subset
of {\eclipse} types that has a meaningful mapping to many other
languages' data types.  The mapping onto different languages is given
in the following table. For details of the mapping between Java
classes/interfaces and EXDR/{\eclipse} types see Section
\ref{sec:ji-type-correspondence}.

\begin{quote}
\begin{tabular}{|llll|}
\hline
\bf EXDR type     & \bf ECLiPSe type  &  \bf TCL type  &  \bf Java type\\
\hline
Integer       & integer       &  int	&	java.lang.Integer\\
 e.g.         & 123           &  123	&	123\\
\hline
Long          & integer       &  string	&	java.lang.Long\\
 e.g.         & 5000000000    &  5000000000 &	5000000000\\
\hline
Double        & float         &  double	&	java.lang.Double/Float\\
 e.g.         & 12.3          &  12.3	&	12.3\\
\hline
String        & string        &  string	&	java.lang.String\\
 e.g.         & "abc"         &  abc	&	"abc"\\
\hline
List          & ./2           &  list	&	java.util.Collection\\
 e.g.         & \lbr a,b,c\rbr  &  \{a b c\} &	\\
\hline
Nil           & \nil /0          &  empty string &	java.util.Collection\\
 e.g.         & \nil            &  \{\} "" & \\
\hline
Struct        & compound      &  list	&	CompoundTerm/Atom\\
 e.g.         & foo(bar,3)    &  \{foo bar 3\}&\\
\hline
Variable      & variable      &  string	&	null\\
 e.g.         & _             &  _	&	\\
\hline
\end{tabular}
\end{quote}

The EXDR Integer data type is a 32-bit signed integer, the EXDR Long
data type is a 64-bit signed integer, bigger {\eclipse} integers cannot
be represented.
The EXDR Variable type only allows singleton, anonymous variables,
which means that it is not possible to construct a term where a variable
occurs in several places simultaneously. The main use of these variables
is as placeholders for result arguments in remote procedure calls.


%----------------------------------------------------------------------
\section{{\eclipse} primitives to read/write EXDR terms}
%----------------------------------------------------------------------

The {\eclipse} predicates to create and interpret EXDR-representation
read from and write directly to {\eclipse} streams. This means that
EXDR-format can be used readily to communicate via files, pipes,
sockets, queues etc.
\begin{description}
\item[\biptxtref{write_exdr(+Stream, +Term)}{write_exdr/2}
	{../bips/kernel/ioterm/write_exdr-2.html}]\ \\
    	This predicate writes terms in exdr format.
	The type of the generated EXDR-term is the type resulting
	from the "natural" mapping of the Eclipse terms.
	Atoms are written as structures of arity 0 (not as strings).
	Note that all information about variable sharing, variable
	names and variable attributes is lost in the EXDR representation.

\item[\biptxtref{read_exdr(+Stream, -Term)}{read_exdr/2}
	{../bips/kernel/ioterm/read_exdr-2.html}]\ \\
    	This predicate reads exdr format and constructs a
	corresponding Eclipse term.
\end{description}

Please refer to chapter \ref{chaptcl} for the Tcl primitives,
and to chapter \ref{chapjava} for the Java primitives for manipulating
EXDR terms. 


%----------------------------------------------------------------------
\section{Serialized representation of EXDR terms}
%----------------------------------------------------------------------

The following is the specification of what is actually send over the
communication channels.
This is all the information needed to create new language mappings
for EXDR terms. This definition corresponds to EXDR_VERSION 2:
\begin{quote}\begin{verbatim}
ExdrTerm      ::=   'V' Version CompactFlag? Term
CompactFlag   ::=   'C'
Term          ::=   (Integer|Double|String|List|Nil|Struct|Variable)
Integer       ::=   ('B' <byte> | 'I' XDR_int | 'J' XDR_long)
Double        ::=   'D' XDR_double
String        ::=   ('S' Length <byte>* | 'R' Index)
List          ::=   '[' Term (List|Nil)
Nil           ::=   ']'
Struct        ::=   'F' Arity String Term*
Variable      ::=   '_'
Length        ::=   XDR_nat
Index         ::=   XDR_nat
Arity         ::=   XDR_nat
Version       ::=   <byte>
XDR_int       ::=   <4 bytes, msb first>
XDR_long      ::=   <8 bytes, msb first>
XDR_double    ::=   <8 bytes, ieee double, exponent first>
XDR_nat       ::=   ( <8 bits: 1 + seven bits unsigned value>
                    | XDR_int )                    // >= 0
\end{verbatim}\end{quote}
The version byte is 1 or 2.  EXDR version 1 encodings are also valid
version 2 encodings, and version version 2 decoders can read
version 1 encoded terms.

XDR_long, XDR_int and byte are all signed integers in two's complement
representation.

The string reference code R means that the string is the same as the
Index'th S-encoded string that occurred in the EXDR term earlier.
The presence of the CompactFlag C in the header indicates that the term
may actually contain such string references. If the flag is absent, the
term does not contain any.

%HEVEA\cutend
