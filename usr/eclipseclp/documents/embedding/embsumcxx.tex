% BEGIN LICENSE BLOCK
% Version: CMPL 1.1
%
% The contents of this file are subject to the Cisco-style Mozilla Public
% License Version 1.1 (the "License"); you may not use this file except
% in compliance with the License.  You may obtain a copy of the License
% at www.eclipse-clp.org/license.
% 
% Software distributed under the License is distributed on an "AS IS"
% basis, WITHOUT WARRANTY OF ANY KIND, either express or implied.  See
% the License for the specific language governing rights and limitations
% under the License. 
% 
% The Original Code is  The ECLiPSe Constraint Logic Programming System. 
% The Initial Developer of the Original Code is  Cisco Systems, Inc. 
% Portions created by the Initial Developer are
% Copyright (C) 2006 Cisco Systems, Inc.  All Rights Reserved.
% 
% Contributor(s): 
% 
% END LICENSE BLOCK
%
% $Id: embsumcxx.tex,v 1.2 2015/01/14 01:31:09 jschimpf Exp $
%

\chapter{Summary of C++ Interface Functions}
%HEVEA\cutdef[1]{section}
\label{chapsumcxx}
Note that apart from the methods and functions described here,
all functions from the C interface which operate on simple types
(int, long, char*) can also be used from C++ code.

\section{Constructing {\eclipse} terms in C++}

\subsection{Class EC_atom and EC_functor}
The {\eclipse} dictionary provides unique identifiers for
name/arity pairs. EC_atoms are dictionary identifiers with zero arity,
EC_functors are dictionary identifiers with non-zero arity.
\begin{description}
\item[EC_atom(char*)]\ \\
	looks up or enters the given string into the {\eclipse}
	dictionary and returns a unique atom identifier for it.

\item[char* EC_atom::name()]\ \\
	returns the name string of the given atom identifier.

\item[EC_functor(char*,int)]\ \\
	looks up or enters the given string with arity into the {\eclipse}
	dictionary and returns a unique functor identifier for it.

\item[char* EC_functor::name()]\ \\
	returns the name string of the given functor identifier.

\item[int EC_functor::arity()]\ \\
	returns the arity of the given functor identifier.
\end{description}


\subsection{Class EC_word}
The EC_word is the basic type that all {\eclipse} data structures are
built from (because within {\eclipse} typing is dynamic).
The following are the functions for constructing {\eclipse} terms from
the fundamental C++ types:
\begin{description}
\item[EC_word(const char *)]\ \\
	converts a C++ string to an {\eclipse} string. The string is copied.

\item[EC_word(const int, const char *)]\ \\
	converts a C++ string of given length to an {\eclipse} string.
	The string is copied and can contain NUL bytes.

\item[EC_word(const EC_atom)]\ \\
	creates an {\eclipse} atom from an atom identifier.

\item[EC_word(const long)]\ \\
	converts a C++ long to an {\eclipse} integer.

\item[EC_word(const long long)]\ \\
	converts a C++ long long to an {\eclipse} integer.

\item[EC_word(const double)]\ \\
	converts a C++ double to an {\eclipse} double float.

\item[EC_word(const EC_ref\&)]\ \\
	retrieves the {\eclipse} term referenced by the EC_ref (see below).

\item[EC_word term(const EC_functor,const EC_word args\nil)]\ \\
\item[EC_word term(const EC_functor,const EC_word arg1, ...)]\ \\
	creates an {\eclipse} compound term from the given components.

\item[EC_word list(const EC_word hd, const EC_word tl)]\ \\
	Construct a single {\eclipse} list cell.

\item[EC_word list(int n, const long*)]\ \\
	Construct an {\eclipse} list of length n from an array of long integers.

\item[EC_word list(int n, const char*)]\ \\
	Construct an {\eclipse} list of length n from an array of chars.

\item[EC_word list(int n, const double*)]\ \\
	Construct an {\eclipse} list of length n from an array of doubles.

\item[EC_word array(int, const double*)]\ \\
	creates an {\eclipse} array (a structure with functor \nil of
	appropriate arity) of doubles from the given C++ array.
	The data is copied.

\item[EC_word matrix(int rows, int cols, const double*)]\ \\
	creates an {\eclipse} array (size rows) of arrays (size cols) of doubles
	from the given C++ array.  The data is copied.

\item[EC_word handle(const t_ext_type *cl, const t_ext_ptr data)]\ \\
	Construct an {\eclipse} handle for external data, attaching the
	given method table.

\item[EC_word newvar()]\ \\
	Construct a fresh {\eclipse} variable.

\item[EC_word nil()]\ \\
	Construct the empty list \nil.
\end{description}

\section{Decomposing {\eclipse} terms in C++}
The following methods type-check an {\eclipse} term and
retrieve its contents if it is of the correct type.
The return code is EC_succeed for successful conversion,
an error code otherwise.
\begin{description}
\item[int EC_word::is_atom(EC_atom *)]\ \\
	checks whether the {\eclipse} pword is an atom, and if so,
	return its atom identifier.

\item[int EC_word::is_string(char **)]\ \\
	checks whether the EC_word is a string (or atom) and converts
	it to a C++ string. This string is volatile, ie.\ it should be copied
	when it is required to survive resuming of {\eclipse}.

\item[int EC_word::is_string(char **, long *)]\ \\
	checks whether the EC_word is a string (or atom) and converts
	it to a C++ string. This string is volatile, ie.\ it should be copied
	when it is required to survive resuming of {\eclipse}.
	The string's length is returned in the second argument.

\item[int EC_word::is_long(long *)]\ \\
	checks whether the EC_word is an integer that fits into
	a C++ long, and if so, stores it via the pointer provided.

\item[int EC_word::is_long_long(long long *)]\ \\
	checks whether the EC_word is an integer that fits into
	a C++ long long, and if so, stores it via the pointer provided.

\item[int EC_word::is_double(double *)]\ \\
	checks whether the EC_word is a floating point number,
	and if so, returns it as an C++ double.

\item[int EC_word::is_list(EC_word\&,EC_word\&)]\ \\
	checks whether the EC_word is a list and if so, returns its head and tail.

\item[int EC_word::is_nil()]\ \\
	checks whether the EC_word is nil, the empty list.

\item[int EC_word::functor(EC_functor *)]\ \\
	checks whether the EC_word is a compound term and if so,
	returns its functor.

\item[int EC_word::arg(const int,EC_word\&)]\ \\
	checks whether the EC_word is a compound term and if so,
	returns its nth argument.

\item[int EC_word::arity()]\ \\
	returns the arity of an EC_word if it is a compound term,
	zero otherwise.

\item[int EC_word::is_handle(const t_ext_type *, t_ext_ptr *)]\ \\
	checks whether the EC_word is a handle whose method table
	matches the given one, and if so, the data pointer is returned.

\item[int EC_word::is_var()]\ \\
	checks whether the EC_word is a variable.
	Returns EC_succeed if so, EC_fail otherwise.
\end{description}


\section{Referring to {\eclipse} terms from C++}
The data types EC_refs and EC_ref provide a means to have non-volatile
references to {\eclipse} data from within C++ data structures. However,
it must be kept in mind that {\eclipse} data structures are nevertheless
subject to backtracking, which means they may be reset to an earlier
status when the search engine requires it. Creating a reference to
a data structure does not change that in any way. In particular, 
when the search engine fails beyond the state where the reference was
set up, the reference disappears and is also reset to its earlier value.
\begin{description}
\item[EC_refs(int n)]\ \\
	create a data structure capable of holding n non-volatile
	references to {\eclipse} data items. They are each initialised
	with a freshly created {\eclipse} variable.

\item[EC_refs(int n,EC_word pw)]\ \\
	create a data structure capable of holding n non-volatile
	references to {\eclipse} data items. They are all initialised
	with the value pw, which must be of a simple type.

\item[\tld EC_refs()]\ \\
	destroy the {\eclipse} references. It is important that this
	is done, otherwise the {\eclipse} garbage collector will not
	be able to free the references data structures, which may
	eventually lead to memory overflow.

\item[EC_word EC_refs::operator\nil(int i)]\ \\
	return the {\eclipse} term referred to by the i'th reference.

\item[void EC_refs::set(int i, EC_word new)]\ \\
	assign the term new to the i'th reference.
	This is a backtrackable operation very similar to
	\bipref{setarg/3}{../bips/kernel/termmanip/setarg-3.html}.

\item[EC_word list(EC_refs\&)]\ \\
	creates an {\eclipse} list containing all the elements of the EC_refs.

\item[EC_ref()]\ \\
\item[EC_ref(EC_word pw)]\ \\
\item[\tld EC_ref()]\ \\
	analogous to the corresponding EC_refs constructors/destructor.

\item[EC_ref\& operator=(const EC_word)]\ \\
	assign a value to the EC_ref.

\item[EC_word(const EC_ref\&)]\ \\
	retrieves the {\eclipse} term referenced by the EC_ref.
\end{description}

\section{Passing Data to and from External Predicates in C++}
These two functions are only meaningful inside C++ functions that have been
called from {\eclipse} as external predicates.
\begin{description}
\item[EC_word		EC_arg(int i)]\ \\
	If inside a C++ function called from {\eclipse}, this returns the
	i'th argument of the call.

\item[int		unify(EC_word,EC_word)]\ \\
	Unify the two given pwords. Note that, if attributed variables
	are involved in the unification, the associated unification handlers
	as well as subsequent waking will only happen once control is
	returned to {\eclipse}.

\item[int		EC_word::unify(EC_word)]\ \\
	Similar, but a method of EC_word.
\end{description}

\section{Operations on {\eclipse} Data}
Interfaces to some basic operations on {\eclipse} Data.
\begin{description}
\item[int		compare(const EC_word\& pw1, const EC_word\& pw2)]\ \\
	Similar to the compare/3 built-in predicate:
	returns 0 if the arguments are identical,
	a negative number if pw1 is smaller than pw2,
	and a positive number if pw1 is greater than pw2
	in the standard term ordering.

\item[int		EC_word::schedule_suspensions(int)]\ \\
	Similar to the
	\bipref{schedule_suspensions/2}{../bips/kernel/suspensions/schedule_suspensions-2.html}
	built-in predicate.
	Waking will only happen once control is returned to {\eclipse}
	and the wake/0 predicate is invoked.
	Return code is EC_succeed or an error code.

\item[pword		EC_word::free_handle(const t_ext_type*)]\ \\
	checks whether the EC_word is an {\eclipse} external data handle of the
	expected type, and calls its free-method. After doing that, the
	handle is stale and cannot be used any longer. Calling this method
	on an already stale handle silently succeeds.
	Return code is EC_succeed or an error code.

\end{description}



\section{Initialising and Shutting Down the {\eclipse} Subsystem}
These are the functions needed to embed {\eclipse} into a C++ main program.
\begin{description}
\item[int		ec_init()]\ \\
	Initialise the {\eclipse} engine. This is required before any other
	functions of this interface can be used.

\item[int		ec_cleanup()]\ \\
	Shutdown the {\eclipse} engine.
\end{description}


\section{Passing Control and Data to {\eclipse} from C++}
These are the functions needed to embed {\eclipse} into C++ code.
\begin{description}
\item[void post_goal(const EC_word)]\ \\
\item[void post_goal(const char *)]\ \\
	post a goal (constraint) that will be executed when \eclipse
	is resumed.

\item[int EC_resume(EC_word FromC, EC_ref\& ToC)]\ \\
\item[int EC_resume(EC_word FromC)]\ \\
\item[int EC_resume()]\ \\
	resume execution of the {\eclipse} engine: All posted goals will
	be executed. The return value will be EC_succeed if the goals succeed
	(in this case the ToC argument returns a cut value that can be
	used to discard alternative solutions). EC_fail is returned if the
	goals fail, and EC_yield if control was yielded because of a
	yield/2 predicate call in the {\eclipse} code (in this case, ToC
	contains the data passed by the first argument of yield/2).
	If a writable queue stream with yield-option
	(see \bipref{open/4}{../bips/kernel/iostream/open-4.html})
	was flushed, the return value is PFLUSHIO and ToC contains the
	associated stream number.
	If there was an attempt to read from an empty queue stream with
	yield-option, the return value is PWAITIO and ToC contains the
	associated stream number.
	Moreover, if the previous EC_resume yielded
	due to a yield/2 predicate call, The term FromC is passed
	as input into the second argument of yield/2 on resuming.

\item[void EC_ref::cut_to()]\ \\
	Should be applied to the ToC cut return value of an EC_resume().
	Cut all choicepoints created by the batch of goals whose execution
	succeeded.

\item[int post_event(EC_word Name)]\ \\
	Post an event to the {\eclipse} engine. This will lead to the
	execution of the corresponding event handler once the \eclipse
	execution is resumed. See also \bipref{event/1}{../bips/kernel/event/event-1.html} and the User Manual
	chapter on event handling for more information. Name should be 
	an {\eclipse} atom.
\end{description}

%HEVEA\cutend
