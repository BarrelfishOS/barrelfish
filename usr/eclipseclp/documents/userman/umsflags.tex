% BEGIN LICENSE BLOCK
% Version: CMPL 1.1
%
% The contents of this file are subject to the Cisco-style Mozilla Public
% License Version 1.1 (the "License"); you may not use this file except
% in compliance with the License.  You may obtain a copy of the License
% at www.eclipse-clp.org/license.
%
% Software distributed under the License is distributed on an "AS IS"
% basis, WITHOUT WARRANTY OF ANY KIND, either express or implied.  See
% the License for the specific language governing rights and limitations
% under the License.
%
% The Original Code is  The ECLiPSe Constraint Logic Programming System.
% The Initial Developer of the Original Code is  Cisco Systems, Inc.
% Portions created by the Initial Developer are
% Copyright (C) 1994 - 2006 Cisco Systems, Inc.  All Rights Reserved.
%
% Contributor(s):
%
% END LICENSE BLOCK
%
% @(#)umsflags.tex	1.7 94/12/06
%

%----------------------------------------------------------------------
\chapter{Global Flags}
%----------------------------------------------------------------------

{\eclipse} has a lot of options and different modes of execution,
some of which are controlled by the global flags described below.
The current value of every flag can be retrieved using
\index{set_flag/2}
\index{get_flag/2}
get_flag(+FlagName, ?Value), writeable flags can be set using
set_flag(+FlagName, +NewValue).
The built-in \bipref{env/0}{../bips/kernel/env/env-0.html} prints a list of all the flags and their current
setting.

%----------------------------------------------------------------------
% Copy here the text from the get_flag/2 bip book page, remove
% all \linebreak commands and replace \newitem by \idxnewitem
%----------------------------------------------------------------------

\newcommand{\idxnewitem}[1]{\index{#1}\item[#1]}

\begin{description}

\idxnewitem{break_level}
\begin{description}
\item[Access mode : ] read/write
\item[Type : ] Integer
\item[Description : ] Specifies current nesting level of recursive
top-level loops.
\end{description}

\idxnewitem{breal_exceptions}
\begin{description}
\item[Access mode : ] read/write
\item[Type : ] Atomic constant {\tt on} or {\tt off}
\item[Default : ] {\tt off}
\item[Description : ]
If {\tt on},  unifying or testing bounded reals for identity raises
an 'undecidable comparison of bounded reals' exception in case the
comparison is undecidable (i.e. the bounds overlap).
If {\tt off}, bounded reals compare equal when their bounds
are equal. This flag does not affect arithmetic comparisons.
\end{description}

\idxnewitem{coroutine}
\begin{description}
\item[Access mode : ] read/write
\item[Type : ] Atomic constant {\tt on} or {\tt off}
\item[Default : ] configuration dependent
\item[Description : ] Specifies whether built-in predicates delay or whether
they raise instantiation faults.
\end{description}

\idxnewitem{cwd}
\begin{description}
\item[Access mode : ] read/write
\item[Type : ] String
\item[Description : ] Specifies the name of the current working directory.
May also be set using \bipref{cd/1}{../bips/kernel/opsys/cd-1.html} or read using \bipref{getcwd/1}{../bips/kernel/opsys/getcwd-1.html}.
\end{description}

\pagebreak[3]
\idxnewitem{debug_compile}
\begin{description}
\item[Access mode : ] read/write
\item[Type : ] Atomic constant {\tt on} or {\tt off}
\item[Default : ] {\tt on}
\item[Description : ] Specifies whether clauses are compiled for debugging or
not. May be set on by \bipref{dbgcomp/0}{../bips/kernel/obsolete/dbgcomp-0.html} or off by \bipref{nodbgcomp/0}{../bips/kernel/obsolete/nodbgcomp-0.html}.
\end{description}

\idxnewitem{debugging}
\begin{description}
\item[Access mode :] read-only
\item[Type :] Atomic constant {\tt nodebug}, {\tt creep}, {\tt leap}, {\tt skip}, {\tt jump_invoc}, {\tt jump_level}, {\tt var_skip}, {\tt woke_skip}, {\tt backtrack_skip} or {\tt same_port_skip}
\item[Default : ] {\tt nodebug}
\item[Description :] Specifies whether debugging is disabled
({\tt nodebug} value)
or enabled. \bipref{trace/0}{../bips/kernel/debug/trace-0.html} sets the {\tt creep} mode, \bipref{debug/0}{../bips/kernel/debug/debug-0.html} sets the
{\tt leap} mode, the other modes are set with the debugger options.
\end{description}

\idxnewitem{debugger_model}
\begin{description}
\item[Access mode :] read/write
\item[Type :] Atomic constant {\tt sepia}, {\tt byrd} or {\tt kcm}
\item[Default : ] {\tt sepia}
\item[Description :] select one of the available debugger models.
They differ in memory consumption, retained information about exited subgoals
and instantiations shown at FAIL ports. Refer to the debugger description
for details.
\end{description}

% \idxnewitem{dfid_compile}
% \begin{description}
% \item[Access mode :] read/write
% \item[Type :] Atomic constant {\tt on} or {\tt off}
% \item[Default : ] {\tt off}
% \item[Description :] When on, the compiler will generate code that keeps track
% of the number of ancestors of each goal. This is used by {\bf library(dfid)}
% to perform bounded depth-first search and iterative deepening.
% \end{description}

\idxnewitem{enable_interrupts}
\begin{description}
\item[Access mode :] read/write
\item[Type :] Atomic constant {\tt on} or {\tt off}
\item[Default : ] {\tt on}
\item[Description :] If {\tt on}, interrupts are recognised and processed
as they occur. If {\tt off}, interrupts are entered into a delay queue and
processed only when the flag is switched back to {\tt on}.
Interrupts should be disabled only for short periods of time in order to keep
the system's interrupt response time short.
\end{description}

\pagebreak[3]
\idxnewitem{extension}
\begin{description}
\item[Access mode :] read-only
\item[Type :] Atomic constant.
\item[Default : ] configuration dependent
\item[Description :] Specifies which extensions are available in the system.
This flag may contain multiple values and will return them on backtracking.
\end{description}

\idxnewitem{float_precision}
\begin{description}
\item[Access mode :] read-only
\item[Type :] Atomic constant.
\item[Default : ] {\tt double}
\item[Description :] This flag is obsolete. Its value is always {\tt double}.
\end{description}

\idxnewitem{gc}
\begin{description}
\item[Access mode : ] read/write
\item[Type : ] Atomic constant {\tt on}, {\tt verbose} or {\tt off}
\item[Default : ] {\tt on}
\item[Description : ] Specifies whether garbage collection is enabled
({\tt on}), disabled ({\tt off}) or enabled and reports every collection
on {\tt toplevel_output} ({\tt verbose}).
\end{description}

\idxnewitem{gc_interval}
\begin{description}
\item[Access mode : ] read/write
\item[Type : ] Integer
\item[Default : ] 1/8 of global stack size
\item[Description : ]
Specify how often the collector is invoked. Roughly, the argument
specifies how many bytes a program can use up before
a garbage collection is triggered.
If the garbage collector runs frequently while reclaiming little space
it may make sense to increase {\tt gc_interval}, thus reducing the number
of garbage collections.
\end{description}

\idxnewitem{gc_interval_dict}
\begin{description}
\item[Access mode : ] read/write
\item[Type : ] Integer
\item[Default : ] 960
\item[Description : ]
Specify after how many new dictionary entries the dictionary gar\-bage collector is invoked.
\end{description}

\idxnewitem{goal_expansion}
\begin{description}
\item[Access mode : ] read/write
\item[Type : ] Atomic constant {\tt on} or {\tt off}
\item[Default : ] {\tt on}
\item[Description : ] Specifies whether goal expansion is done by the compiler.
This includes goal macros and other compiler inlining.
Can be disabled for debugging purposes.
\end{description}

\pagebreak[3]
\idxnewitem{hostarch}
\begin{description}
\item[Access mode :] read-only
\item[Type :] String
\item[Description :] String identifying the host processor and operating
system. It is also the name of the machine-dependent subdirectories in the
{\eclipse} installation.
\end{description}

\idxnewitem{hostid}
\begin{description}
\item[Access mode :] read-only
\item[Type :] Integer or string
\item[Description :] The unique identification of the host hardware
the system is running on.
\end{description}

\idxnewitem{hostname}
\begin{description}
\item[Access mode :] read-only
\item[Type :] String
\item[Description :] The name of the current host machine.
\end{description}

\idxnewitem{ignore_eof}
\begin{description}
\item[Access mode : ] read/write
\item[Type : ] Atomic constant {\tt on} or {\tt off}
\item[Default : ] Depends on host architecture
\item[Description : ] Specifies whether {\eclipse} is exited
when end-of-file is typed to the toplevel prompt or whether this is ignored.
If ignored, {\eclipse} can be exited by calling
\bipref{halt/0}{../bips/kernel/opsys/halt-0.html}.
\end{description}

\idxnewitem{installation_directory}
\begin{description}
\item[Access mode :] read-only
\item[Type :] String
\item[Description :] The name of the toplevel directory of the
running {\eclipse} installation. All {\eclipse} library, documentation and
other directories are below this one.
\end{description}

%\idxnewitem{integer_arithmetic}
%\begin{description}
%\item[Access mode :] read/write
%\item[Type :] Atomic constant {\tt overflow} or {\tt modulo_32}.
%\item[Description :] The behaviour of integer arithmetic operation.
%They can either raise events on overflows or return a result modulo 32.
%\end{description}

\idxnewitem{last_errno}
\begin{description}
\item[Access mode :] read-only
\item[Type :] Integer
\item[Description :] The error code that the most recent failed
operating system call returned.
\end{description}

\idxnewitem{library_path}
\begin{description}
\item[Access mode :] read/write
\item[Type :] List of strings
\item[Default : ] the contents of the user's ECLIPSELIBRARYPATH environment
\index{ECLIPSELIBRARYPATH}
variable, followed by the system library directories
\item[Description :] Specifies the list of pathnames used by the system to
search for library files.
The library path is used by \bipref{lib/1,2}{../bips/kernel/compiler/lib-1.html}, for autoloading, and to expand
library/1 structures in pathnames.
\end{description}

\idxnewitem{loaded_library}
\begin{description}
\item[Access mode : ] read/write
\item[Type : ] Atom
\item[Description : ] Returns the names of the currently loaded libraries.
This flag may contain multiple values and will return them on backtracking.
\end{description}

\idxnewitem{macro_expansion}
\begin{description}
\item[Access mode : ] read/write
\item[Type : ] Atomic constant {\tt on} or {\tt off}
\item[Default : ] {\tt on}
\item[Description : ] Specifies whether macro expansion used by the source
transformation facility is enabled or disabled.
Should be disabled only for debugging purposes.
\end{description}

\idxnewitem{max_global_trail}
\begin{description}
\item[Access mode :] read-only
\item[Type :] Integer
\item[Description :] The total memory available to the global and the trail
stack in bytes.
\end{description}

\idxnewitem{max_local_control}
\begin{description}
\item[Access mode :] read-only
\item[Type :] Integer
\item[Description :] The total memory available to the local and control stack
in bytes.
\end{description}

\idxnewitem{max_predicate_arity}
\begin{description}
\item[Access mode : ] read-only
\item[Type : ] Integer
\item[Default : ] 255
\item[Description : ] Returns the maximum number of arguments allowed for
an {\eclipse} predicate.
\end{description}

\item[object_suffix]
\begin{description}
\item[Access mode : ] read-only
\item[Type : ] String
\item[Description : ] Returns the suffix of the external
object files that can be loaded using \bipref{load/1}{../bips/kernel/externals/load-1.html}.
It is usually \notation{so} for systems that support shared libraries
and \notation{o} for the others.
\end{description}

\idxnewitem{occur_check}
\begin{description}
\item[Access mode :] read/write
\item[Type : ] Atomic constant {\tt on} or {\tt off}
\item[Default : ] {\tt off}
\item[Description :] If the flag is on, occur check is performed
on uncompiled unifications and the compiler generates code for unifications
that will perform the occur check when necessary.
\end{description}

\idxnewitem{output_mode}
\begin{description}
\item[Access mode :] read/write
\item[Type :] String
\item[Default : ] "QPm"
\item[Description :] The value is a control string that is recognised
by the {\bf \%w} format of \bipref{printf/3}{../bips/kernel/ioterm/printf-3.html}.
This format is used to output results on the toplevel loop and to print
debugger trace lines.
\end{description}

\idxnewitem{pid}
\begin{description}
\item[Access mode : ] read-only
\item[Type : ] Integer
\item[Description : ] Returns the process identifier of the current {\eclipse}.
\end{description}

\pagebreak[3]
\idxnewitem{ppid}
\begin{description}
\item[Access mode : ] read-only
\item[Type : ] Integer
\item[Description : ] Returns the process identifier of the current {\eclipse}'s
parent process.
\end{description}

\idxnewitem{prefer_rationals}
\begin{description}
\item[Access mode : ] read/write
\item[Type : ] Atomic constant {\tt on} or {\tt off}
\item[Default : ] {\tt off}
\item[Description : ] Specifies if the result of a /-division of two integers
gives a rational or a floating point result.
Similar for the result of raising an integer to a negative integral power.
\end{description}

\idxnewitem{print_depth}
\begin{description}
\item[Access mode : ] read/write
\item[Type : ] Integer
\item[Default : ] 20
\item[Description : ] Specifies the print depth bound for printing
compound terms. It can also be set by the '$<$' option of the debugger.
It is not taken into account by \bipref{writeq/1, 2}{../bips/kernel/ioterm/writeq-1.html},
other I/O built-ins obey this flag.
\end{description}

\idxnewitem{prolog_suffix}
\begin{description}
\item[Access mode : ] read/write
\item[Type : ] List of strings
\item[Default : ] {\tt ["", ".sd", ".pl"]}
\item[Description : ] Specifies the Prolog source file suffix(es) used when
compiling files or loading libraries.
The system tries to find the file by appending
the given suffixes in the order provided.
\end{description}

\idxnewitem{statistics}
\begin{description}
\item[Access mode : ] read/write
\item[Type : ] Atomic constant {\tt off}, {\tt some}, {\tt all} or {\tt mode}
\item[Default : ] {\tt off}
\item[Description : ] Used to control the operation of the profiling facility.
{\tt off} disables profiling, {\tt all} enables counting of all traceable
procedure ports, {\tt some} restricts this to the ones that have been
selected with {\bf set_flag/3} or \bipref{statistics/2}{../bips/kernel/env/statistics-2.html}. {\tt mode} is like
{\tt all} but it also records the modes of procedure calls.
\end{description}

\pagebreak[3]
\idxnewitem{syntax_option}
\begin{description}
\item[Access mode : ] read/write
\item[Type : ] Atom
\item[Description : ] Returns the names of the currently enabled syntax options.
This flag may contain multiple values and will return them on backtracking.
A syntax option is reset using set_flag(syntax_option, not(Flag)).
Most compatibility packages affect these flags as well.
The following options are available:
\begin{itemize}
\item {\tt based_bignums} - Allow base notation even for integers
longer than the wordsize (i.e.\ they are always positive).
\item {\tt blanks_in_nil} - allow blanks between the brackets in {\tt []}.
\item {\tt limit_arg_precedence} - do not allow terms with a precedence higher
than 999 as structure arguments, unless parenthesised.
\item {\tt nested_comments} - allow bracketed comments to be nested.
\item {\tt nl_in_quotes} - allow newlines to occur inside quotes.
\item {\tt no_blanks} - do not allow blanks between functor an opening
parenthesis
\item {\tt no_other_quotes} - do not allow string quotes inside atom quotes
and vice versa.
% obsolete:
%\item {\tt quintus_fmt} - use $\sim$ rather than \% in format strings
%for printf/2,3.
\item {\tt \$VAR} - terms of the form '\$VAR'(N) are printed in a special way by all the predicates that obey operator declarations (i.e.\ write, writeq, print and partly printf). '\$VAR'(0) is printed as A, '\$VAR'(25) as Z, '\$VAR'(26) as A1 and so on.
When the argument is an atom or a string, just this argument is printed.
\end{itemize}
\end{description}

\idxnewitem{toplevel_module}
\begin{description}
\item[Access mode : ] read/write
\item[Type : ] Atom
\item[Description : ] The name of the current top-level module. This is the
caller module for all queries entered in the top-level loop.
By default, this is also shown in the top-level prompt.
\end{description}

\idxnewitem{unix_time}
\begin{description}
\item[Access mode : ] read-only
\item[Type : ] Integer
\item[Description : ] Return the time as given by the UNIX time(3) function,
ie. seconds since 00:00:00 GMT, Jan 1 1970.
\end{description}

\idxnewitem{tmp_dir}
\begin{description}
\item[Access mode : ] read/write
\item[Type : ] String
\item[Description : ] Specifies the temporary directory that ECLiPSe may use
for storing temporary data files. Value should be an
existing directory (in ECLiPSe's file name syntax) that the user
can write to (set_flag/2 will fail otherwise). It is also
recommended that the directory should reside on a local disk where
the I/O operations are as fast as possible.   By default, this
directory is taken from the environment variable ECLIPSETMP if it
exist. Otherwise, it is {\tt "/tmp"} for Unix systems; {\tt "//C/Temp"} for
Windows. If none of these exist, it is set to the current
working directory at start up.
\end{description}

%\idxnewitem{user_options}
%\begin{description}
%\item[Access mode : ] read/write
%\item[Type : ] String
%\item[Default : ] {\tt ""}
%\item[Description : ] Specifies which non-standard command line options
%\index{command line options!nonstandard}
%(beginning with a {\bf -}) may
%be given when invoking {\eclipse} without causing an error message. These
%arguments can be processed using \bipref{argc/1}{../bips/kernel/opsys/argc-1.html} and \bipref{argv/2}{../bips/kernel/opsys/argv-2.html}. This
%flag is used to create a {\eclipse} saved state that accepts non-standard
%command line options and/or does not interpret the standard command
%line options.  \end{description}

\idxnewitem{variable_names}
\begin{description}
\item[Access mode : ] read/write
\item[Type : ] Atomic constant {\tt on}, {\tt off} or {\tt check_singletons}
\item[Default : ] {\bf check_singletons}
\item[Description : ] Controls the ability to retain the source name of
variables. This flag affects the reading process only, i.e.\ when a variable
is read (or compiled) with the flag set to {\bf on}, it will keep its name
even when the flag is switched off later.
{\bf check_singletons} is like {\bf on} but additionally the compiler emits
warnings about source variables which occur only once in a clause
and whose name does not start with an underscore.
The source variable names are being created
during the term input if this flag is not {\bf off},
and then they are kept independently of the value of this flag.
\end{description}

\idxnewitem{version}
\begin{description}
\item[Access mode : ] read-only
\item[Type : ] Atom
\item[Description : ] Returns the current version number of {\eclipse}.
\end{description}

\item[worker]
\begin{description}
\item[Access mode : ] read-only
\item[Type : ] Integer
\item[Description : ] In a parallel session it returns a positive number,
identifying the worker on which the flag inquiry was executed.
In a sequential session 0 is returned.
\end{description}

\end{description}
