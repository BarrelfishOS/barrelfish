% BEGIN LICENSE BLOCK
% Version: CMPL 1.1
%
% The contents of this file are subject to the Cisco-style Mozilla Public
% License Version 1.1 (the "License"); you may not use this file except
% in compliance with the License.  You may obtain a copy of the License
% at www.eclipse-clp.org/license.
%
% Software distributed under the License is distributed on an "AS IS"
% basis, WITHOUT WARRANTY OF ANY KIND, either express or implied.  See
% the License for the specific language governing rights and limitations
% under the License.
%
% The Original Code is  The ECLiPSe Constraint Logic Programming System.
% The Initial Developer of the Original Code is  Cisco Systems, Inc.
% Portions created by the Initial Developer are
% Copyright (C) 1994 - 2006 Cisco Systems, Inc.  All Rights Reserved.
%
% Contributor(s):
%
% END LICENSE BLOCK
%
% @(#)umsporting.tex	1.1 94/10/07
%

\chapter{Language Dialects, ISO Prolog and Porting Prolog Applications}
%HEVEA\cutdef[1]{section}
\label{chapporting}

The {\eclipse} system has evolved from the Edinburgh family of
Prolog systems, and thus shares many properties with other systems
in the same tradition.  It also supports the
\index{ISO Prolog}
ISO Prolog Standard from 1995 and its 2005 and 2012 corrigenda.

However, the default programming language dialect used with {\eclipse}
(known as \notationidx{eclipse_language}) is a separate and unique dialect,
which is the result of design decisions taken for conceptual, practical
and occasionally historical reasons.

To run an application written in another Prolog dialect on {\eclipse},
one has basically two choices:
Using a compatibility package, or modifying the program.


\section{Using compatibility language dialects}
The {\eclipse} compatibility language dialects are the fastest way to get a
program running that was originally written for a different system.
The dialects are implemented as libraries.
The module system makes it possible for different application
modules to use different language dialects.

To use a particular language dialect, prefix your program with a
\bipref{module/3}{../bips/kernel/modules/module-3.html}
directive that specifies the desired language dialect, for example
\begin{verbatim}
    :- module(mymodule, [], iso).
\end{verbatim}
Here, the last argument of the module/3 directive indicates the language
(the default being eclipse_language).
It is not advisable to use \verb.:-lib(iso). or \verb.:-ensure_loaded(library(iso)).
within an eclipse_language module, because this would lead to import
conflicts between the different versions of built-in predicates.

Examples of supported language dialects are
\begin{itemize}
\item ISO Standard Prolog
    (\bipref{iso_strict}{../bips/lib/iso_strict/index.html},
    \bipref{iso}{../bips/lib/iso/index.html} and
    \bipref{iso_light}{../bips/lib/iso_light/index.html}).
\item C-Prolog (\bipref{cprolog}{../bips/lib/cprolog/index.html}), one
    of the oldest and most influential Prolog implementations.
\item Quintus Prolog (\bipref{quintus}{../bips/lib/quintus/index.html}),
    an early influential commercial system.
\item SICStus Prolog (\bipref{sicstus}{../bips/lib/sicstus/index.html}),
    an academic and commercial system based on Quintus.
\item SWI Prolog (\bipref{swi}{../bips/lib/swi/index.html}),
    a popular Prolog with large user base. 
\end{itemize}
See the Reference Manual for details on the compatibility provided by the
language dialects.
The language dialects are just modules which provide the necessary code
and exports  to emulate a particular Prolog dialect. This module is imported
instead of the default \notation{eclipse_language} dialect which provides the
{\eclipse} language.
The source code of the language dialect module is provided in the
{\eclipse} library directory.
Using this as a guideline, it should be easy to write similar packages for
other systems, as long as their syntax does not deviate too much
from the Edinburgh tradition.

For quick experiments with a language dialect, {\eclipse} can be started
with a different \notationidx{default_language} option
(see section \ref{cmdlineopts}), e.g.
\begin{verbatim}
    % eclipse -L <dialect>
\end{verbatim}
This will give you a toplevel prompt in the given language dialect.
The same effect can be achieved by setting the ECLIPSEDEFAULTLANGUAGE
\index{ECLIPSEDEFAULTLANGUAGE}
environment variable to the name of the chosen dialect.


\subsection{ISO Prolog}
\index{ISO Prolog}
The ISO Prolog standard \cite{isoprolog95} is supported in three variants:
\begin{itemize}
\item The \bipref{iso_strict}{../bips/lib/iso_strict/index.html} dialect
    provides an implementation of ISO Standard Prolog and complies
    strictly with ISO/IEC 13211-1 (Information Technology, Programming
    Languages, Prolog, Part 1, General Core, 1995) and the technical
    corrigenda ISO/IEC 13211-1 TC1 (2007) and TC2 (2012).
\item The \bipref{iso}{../bips/lib/iso/index.html} dialect provides
    a blend of ISO and ECLiPSe functionality.  All ISO features
    are available, plus such ECLiPSe features that do not significantly
    conflict with ISO.  As some of these extensions go beyond what the
    letter of the standard allows, and because error checking may be
    less strict than required by ISO, this is not a fully compliant mode.
\item The \bipref{iso_light}{../bips/lib/iso_light/index.html} dialect
    is the same as 'iso', with the exception of error handling.
    This is sufficient for code that does not rely on a particular
    form of error terms being thrown by built-in predicates.
\end{itemize}
The specification of implementation-defined features stipulated
by the standard can be found in the reference manual for
\bipref{iso_strict}{../bips/lib/iso_strict/index.html}.


\subsection{Compiler versus interpreter}
The following problem can occur despite the use of compatibility packages:
If your program was written for an interpreter, e.g., C-Prolog,
you have to be aware that {\eclipse} is a compiling system.
There is a distinction between \emph{static} and \emph{dynamic} predicates.
By default, a predicate is static. This means that its clauses have to be
be compiled as a whole (they must not be spread over multiple files),
its source code is not stored in the system,
and it can not be modified (only recompiled as a whole).
In contrast, a dynamic predicate may be modified by compiling or
asserting new clauses and by retracting clauses.
Its source code can be accessed using
\biprefni{clause/1,2}{../bips/kernel/dynamic/clause-1.html}%
\indextt{clause/1}\indextt{clause/2}
or
\biprefni{listing/0,1}{../bips/kernel/dynamic/listing-0.html}.%
\indextt{listing/0}\indextt{listing/1}
A predicate is dynamic when it is explicitly declared as such or when
it was created using \bipref{assert/1}{../bips/kernel/dynamic/assert-1.html}.
Porting programs from an interpreter usually requires the addition of
some \notation{dynamic} declarations.
In the worst case, when (almost) all procedures have to be dynamic,
the flag \notation{all_dynamic} can be set instead.


\section{Porting programs to plain {\eclipse}}
If you want to use {\eclipse} to do further development of your application,
it is probably advantageous to modify it such that it runs under plain
{\eclipse}.
In the following we summarise the main aspects that have to be considered
when doing so.

\begin{itemize}
\item
In general, it is almost always possible to add to your program
a small routine that fixes the problem, rather than to modify
the source of the application in many places.
For example, name clashes are fixed more easily
by using the \bipref{local/1}{../bips/kernel/modules/local-1.html} declaration
rather than by renaming
the clashing predicate in the whole application program.

\item
Due to lack of standardisation, some subtle differences in the
syntax exist between Prolog systems. See \ref{syntaxdiff}
for details. {\eclipse} has a number of options that make it possible
to configure its behaviour as desired.

\item
{\eclipse} has the \notation{string} data type which is not present in Prolog
of the Edinburgh family.
Double-quoted items are parsed as strings in {\eclipse}, while they are
lists of integers in other systems and when the compatibility
packages are used (cf.\ chapter \ref{chapstring}).

\item
I/O predicates of the \predspec{see} and \predspec{tell} group are not built-ins
in {\eclipse}, but they are provided in the \libspec{cio} library.
Call \notation{lib(cio)} in order to have them available (cf.\ appendix A).
Similarly for \bipref{numbervars/3}{../bips/lib/numbervars/index.html}.

\item
In {\eclipse}, some built-ins raise events in cases where they just fail
in other systems, e.g., \notation{arg(1,~2,~X)} fails in C-Prolog, but
raises a type error in {\eclipse}.
If some code relies on such behaviour, it is best to modify it by
adding an explicit check like
\begin{verbatim}
        ..., compound(T), arg(N, T, X), ...
\end{verbatim}

Another alternative is to redefine the \predspec{arg/3} built-in, using
\predspec{:/2} to access the original version:
\begin{verbatim}
:- local arg/3.
arg(N, T, X) :-
        compound(X),
	eclipse_language:arg(N, T, X).
\end{verbatim}

A third alternative
is to define an error handler which will fail the predicate
whenever the event is raised. In this case:
\begin{verbatim}
my_type_error(_, arg(_, _, _)) :- !, fail.
my_type_error(E, Goal) :- error(default(E), Goal).
:- set_error_handler(5, my_type_error/2).
\end{verbatim}

\item As the {\eclipse} compiler does not accept procedures whose clauses
are not consecutive in a file, it may be necessary to add
\bipref{discontiguous/1}{../bips/kernel/compiler/discontiguous-1.html}
directives if you want to compile such procedures.

\end{itemize}


\section{Exploiting the features of {\eclipse}}
When rewriting existing applications as well as when writing new programs,
it is useful to bear in mind important {\eclipse} features which can make
programs easier to write and/or faster:
\begin{itemize}
\item Compiler features relevant for performance can be found in section
\ref{secefficientcode}.

\item Use {\eclipse}'s nonlogical
\bipref{storage}{../bips/kernel/storage/index.html} facilities
(section \ref{secarrays}),
which are usually more suitable to store permanent data than
\bipref{assert/1}{../bips/kernel/dynamic/assert-1.html} is, and are usually
faster.

\item {\eclipse} has a number of language extensions which make programming
easier, see chapter \ref{chaplanguage}.

\item The predicates \predspec{get_flag/2}, \predspec{get_flag/3},
\predspec{get_file_info/3}, \predspec{get_stream_info/3} and\\
\predspec{get_var_info/3}
give a lot of useful information about the system and the data.

\item The {\eclipse} macros often help to solve syntactic problems
(see chapter \ref{chapmacros}).

\item The {\tkeclipse} GUI provides many features that should make
developing programs easier than with the traditional tty interface.

\item It is worth familiarising oneself with the debugger's features,
see chapter \ref{chapdebug}.

\item {\eclipse} is highly customizable, even problems which seemingly
require modification of the {\eclipse} sources
can very often be solved at the Prolog level.
\end{itemize}

%HEVEA\cutend
