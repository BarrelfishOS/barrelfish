% BEGIN LICENSE BLOCK
% Version: CMPL 1.1
%
% The contents of this file are subject to the Cisco-style Mozilla Public
% License Version 1.1 (the "License"); you may not use this file except
% in compliance with the License.  You may obtain a copy of the License
% at www.eclipse-clp.org/license.
%
% Software distributed under the License is distributed on an "AS IS"
% basis, WITHOUT WARRANTY OF ANY KIND, either express or implied.  See
% the License for the specific language governing rights and limitations
% under the License.
%
% The Original Code is  The ECLiPSe Constraint Logic Programming System.
% The Initial Developer of the Original Code is  Cisco Systems, Inc.
% Portions created by the Initial Developer are
% Copyright (C) 2006 Cisco Systems, Inc.  All Rights Reserved.
%
% Contributor(s):
%
% END LICENSE BLOCK
%------------------------------------------------------------------------
\chapter{Command Line and Startup Options}
\label{cmdlineopts}
%------------------------------------------------------------------------
\section{Command Line Options}
\index{command line options}

The {\eclipse} system has several parameters which may be specified on the
command line at invocation time. All the parameters are available with
the command line version of \notation{eclipse}; with \notation{tkeclipse}, only the \notation{-g}
and \notation{-l}
parameters are available.
The parameters are as follows:

\begin{description}
\item[$-$b file]\cmdlineoptionidx{b}
The same as $-$f file.

\item[$-$f file]\cmdlineoptionidx{f}
Compile the file \notation{file} before starting the session.
Multiple \notation{-f} options are allowed.
The file name is expected to be in the operating system's syntax.
The file is processed by
\bipref{ensure_loaded/1}{../bips/kernel/compiler/ensure_loaded-1.html},
i.e., it can be a precompiled file or a source file, and file extensions
are added as specified there.
%In this case the {\tt .eclipserc} file will not be compiled.

\item[$-$e goal]\cmdlineoptionidx{e}
Instead of starting an interactive toplevel, the system will execute the
goal \notation{goal}. \notation{goal} is given in normal Prolog syntax, and has
to be
quoted if it contains any characters that would normally be interpreted by the
shell. The \notation{-e} option can be used together with the \notation{-f}
option and is executed
afterwards. Only one \notation{-e} option is allowed.
%If the -e option is used together with the saved state option -s, then the
%goal will not be executed when making the saved state, but when restoring it.

The exit status of the {\eclipse} process reflects success or failure of the
\index{exit status}
executed Prolog goal (0 for success, 1 for failure, 2 for abort).

When you only have a runtime installation of eclipse, the -e option
is compulsory because a runtime system does not have an interactive
toplevel.


\item[$-$g size]\cmdlineoptionidx{g}
This option specifies the limit to which the memory consumption of the
{\eclipse} global/trail stack can grow.
The size is specified in kilobytes (followed by an optional \notation{K}), in
megabytes
(followed by \notation{M}) or in gigabytes (followed by \notation{G}).
The default is \notation{512M} on 64-bit, or \notation{256M} on 32-bit machines.
The amount required for this stack depends on the program's data
structures and may have to be increased for very large applications.

\item[$-$l size]\cmdlineoptionidx{l}
This option specifies the limit to which the memory consumption of the
{\eclipse} local/control stack can grow.
The size is specified in kilobytes (followed by an optional \notation{K}), in
megabytes
(followed by \notation{M}) or in gigabytes (followed by \notation{G}).
The default is \notation{128M} on 64-bit, or \notation{64M} on 32-bit machines.
The local/control stack is unlikely to require more than this default.
If it does, it is probably caused by a programming error.

\item[$-$D directory]\cmdlineoptionidx{D}
This options allows one to explicitly specify the {\eclipse} installation
directory, i.e., the directory in which the system tries to find
the {\eclipse} runtime system and libraries.  This option overrides
(and renders unnecessary) any setting of the ECLIPSEDIR environment
variable (Unix) or, respectively, an ECLIPSEDIR registry entry
(Windows) that may be in effect.

\item[$-$L language]\cmdlineoptionidx{L}
The name of the language dialect used in the ``top level module''.
The default is \notation{eclipse_language}, other possible values
include \notation{iso}, \notation{iso_strict}, \notation{quintus} etc.
This property can also be set via an \notationidx{ECLIPSEDEFAULTLANGUAGE}
environment variable.

\item[$-$t module]\cmdlineoptionidx{t}
The name of the ``top level module''.  This is an initially empty module,
created by the system, which serves as the context for \notation{-f} and
\notation{-e} options, and in which interactive toplevel queries are executed.
This can be an arbitrary name, as long as it does not conflict with
important library names.  The default is \notation{eclipse}.

\item[$-$ $-$]\cmdlineoptionidx{-}
The {\eclipse} system will ignore this argument and everything that follows on
the commmand line. The Prolog program will only see the part of the
command line that follows this argument.
\end{description}


\section{{\tkeclipse} Startup Settings}

\notation{tkeclipse} accepts the same $-$g and $-$l options as
\notation{eclipse} for setting stack sizes.

In addition, \notation{tkeclipse} on UNIX reads settings from 
the files \notationidx{.tkeclipserc} (for toplevel parameters) and
\notationidx{.tkeclipsetoolsrc} (for development tools parameters).
The system will first search for these files in the current directory,
and then in the user's home directory.
Each parameter is specified on a separate line in the appropriate
file, in the format
\begin{verbatim}
    parameter value
\end{verbatim}
On Windows, the settings are instead from the registry keys
\begin{verbatim}
    HKEY_CURRENT_USER\\Software\\IC-Parc\\ECLiPSe\\tkeclipserc
    HKEY_CURRENT_USER\\Software\\IC-Parc\\ECLiPSe\\tkeclipsetoolsrc
\end{verbatim}
where each parameter is specified as a string value under the appropriate
parameter key.
In either case, the parameters can be modified and saved using the
{\tkeclipse} Preference Editor.

The parameters corresponding to the {\eclipse} command line settings are
\begin{description}
\item[\notationidx{globalsize} (positive integer)]
    The maximum size to which the
    global/trail stack area can grow
    to. The unit is megabytes. 
    Overridden by the -g option.
\item[\notationidx{localsize} (positive integer)]
    The maximum size to which the
    local/control stack area can grow
    to. The unit is megabytes.
    Overridden by the -l option.
\item[\notationidx{initquery} (string)]
    An {\eclipse} query that {\tkeclipse}
    will execute immediately after
    startup. It can be used to perform
    user defined initialisations.
\item[\notationidx{default_language} (string)]
    The default language name (default: eclipse_language).
    Can be overridden by ECLIPSEDEFAULTLANGUAGE environment variable.
\item[\notationidx{default_module} (string)]
    The default toplevel module name (default: eclipse).
\end{description}

