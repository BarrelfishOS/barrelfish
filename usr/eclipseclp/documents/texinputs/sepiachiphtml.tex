%
% $Id: sepiachiphtml.tex,v 1.9 2015/10/17 03:01:33 kish_shen Exp $
%
% BEGIN LICENSE BLOCK
% Version: CMPL 1.1
%
% The contents of this file are subject to the Cisco-style Mozilla Public
% License Version 1.1 (the "License"); you may not use this file except
% in compliance with the License.  You may obtain a copy of the License
% at www.eclipse-clp.org/license.
%
% Software distributed under the License is distributed on an "AS IS"
% basis, WITHOUT WARRANTY OF ANY KIND, either express or implied.  See
% the License for the specific language governing rights and limitations
% under the License.
%
% The Original Code is  The ECLiPSe Constraint Logic Programming System.
% The Initial Developer of the Original Code is  Cisco Systems, Inc.
% Portions created by the Initial Developer are
% Copyright (C) 2006 Cisco Systems, Inc.  All Rights Reserved.
%
% Contributor(s):
%
% END LICENSE BLOCK

% This is not the original sepiachip.sty,
% but a drastically simplified one.
%

\newcommand{\eclipseversion}{6.2}

% characters for indexing ? needed for a HeVeA bug
\newcommand*{\query}{?}
\newcommand*{\atsym}{@}
\newcommand*{\cutsym}{!}

% Like index, but in tt font
\newcommand*{\indextt}[1]{\index{#1@\texttt{#1}}}

\newcommand*{\newitem}[1]{\item[#1]}

\newcommand*{\bipnoidx}[1]{\textbf{#1}}
\newcommand*{\bip}[1]{\bipnoidx{#1}\indextt{#1}}

%\newcommand{\biprefnoidx}[2]{\latex{{\bf #1}}\html{\htmladdnormallink{#1}{#2}}}
\newcommand*{\biprefnoidx}[2]{\ahref{#2}{\textbf{#1}}}
\newcommand*{\biprefni}[2]{\biprefnoidx{#1}{#2}}
\newcommand*{\bipref}[2]{\biprefnoidx{#1}{#2}\indextt{#1}}

\newcommand*{\biptxt}[2]{\bipnoidx{#1}\indextt{#2}}
\newcommand*{\txtbip}[2]{\bipnoidx{#1}\indextt{#1}}

\newcommand*{\biptxtrefni}[3]{\biprefnoidx{#1}{#3}}
\newcommand*{\biptxtref}[3]{\biprefnoidx{#1}{#3}\indextt{#2}}

\newcommand*{\txtbiprefni}[3]{\biprefnoidx{#1}{#3}}
\newcommand*{\txtbipref}[3]{\txtbiprefni{#1}{}{#3}\indextt{#1}}

% Put this word in the text, but also index it:
\newcommand*{\Index}[1]{#1\index{#1}}
% Ditto, but index in textt:
\newcommand*{\Indextt}[1]{#1\indextt{#1}}

% A word/phrase that we are talking about, not just a part of the sentence:
\newcommand*{\about}[1]{\emph{#1}}
% Ditto, with an index entry:
\newcommand*{\aboutidx}[1]{\about{#1}\index{#1}}

% Chapter name
\newcommand*{\chapname}[1]{\emph{#1}}

% Example name
\newcommand*{\examplename}[1]{\emph{#1}}

% Tool name
\newcommand*{\toolname}[1]{\emph{#1}}

% The first, defining occurrence of the name of a concept etc.:
\newcommand*{\defnotion}[1]{\textbf{#1}\index{#1}}
% Ditto with a different index entry:
\newcommand*{\defnotioni}[2]{\textbf{#1}\index{#2}}
% Ditto without an index entry:
\newcommand*{\defnotionni}[1]{\textbf{#1}}

% Concrete notation, also a very short fragment of a program embedded in the
% text, a concrete file name etc.
\newcommand*{\notation}[1]{\texttt{#1}}
% Ditto, with an index entry:
\newcommand*{\notationidx}[1]{\notation{#1}\indextt{#1}}

% A pattern, i.e., notation that is not concrete:
\newcommand*{\pattern}[1]{\textsl{#1}}
% Ditto, with an index entry:
\newcommand*{\patternidx}[1]{\pattern{#1}\index{#1@\textsl{#1}}}

% Predicate specification, i.e., name/arity
\newcommand*{\predspec}[1]{\textbf{#1}}
% Ditto with an index entry:
\newcommand*{\predspecidx}[1]{\predspec{#1}\indextt{#1}}

% Predicate ``definition'', e.g., showing it with arguments patterns.
% NOTE: have to add indextt explicitly.
\newcommand*{\preddef}[1]{\textbf{#1}}

% Index entry for a library:
\newcommand*{\libidx}[1]%
{\index{#1@\textbf{#1} (library)}\index{library!\textbf{#1}}}
% Library specification, with index:
\newcommand*{\libspec}[1]{\textbf{#1}\libidx{#1}}


% Index entry for a command line option:
\newcommand*{\cmdlineoptionidx}[1]%
{\index{command line options!\notation{-#1}}%
\index{#1 (command line option)@\notation{-#1} (command line option)}}

% Index entry for a handler:
\newcommand*{\handleridx}[1]%
{\index{#1 handler@\notation{#1} handler}\index{handler!\notation{#1}}}

% Bold table column heading etc.
\newcommand*{\heading}[1]{\textbf{#1}}



\newcommand*{\vbar}{$\mid$}
\newcommand*{\uparr}{$\wedge$}
\newcommand*{\bsl}{$\backslash$}
\newcommand*{\andsy}{$/\backslash$}
\newcommand*{\orsy}{$\backslash/$}
\newcommand*{\tld}{$\sim$}
\newcommand*{\lbr}{$[$}
\newcommand*{\rbr}{$]$}
\newcommand*{\nil}{$[~]$}
\newcommand*{\lt}{$<$}
\newcommand*{\gt}{$>$}
\newcommand*{\chr}{{\sf CHR}}
\newcommand*{\chrs}{{\sf CHR}s}
\newcommand*{\eclipse}{ECL$^i$PS$^e$}
\newcommand*{\tkeclipse}{TkECL$^i$PS$^e$}
\newcommand*{\sepia}{SEPIA}



%-------------------------------
%
% @(#)umsdebuggercomms.tex	1.3 93/03/29
%
\newenvironment{descr}[1]
{\begin{list}{}{\setlength{\leftmargin}{#1}}}{\end{list}}

% Index entry for a debugger command:
\newcommand*{\dbgcmdidx}[2]{\dbgcmdidxPlus{#1}{#1}{#2}}
% Ditto when the indexing entry is different from the one shown,
% e.g., \dbgcmdidxPlus{$<$}{<}{print depth}:
\newcommand*{\dbgcmdidxPlus}[3]%
{\index{#1---#3 (debugger cmd)@\notation{#2}---#3 (debugger cmd)}%
\index{debugger command!\notation{#2}}}

\newcommand*{\cmd}[2]%
{\item[\textbf{#1}\hfill]\textbf{#2}\dbgcmdidx{#1}{#2}}

\newcommand*{\ncmd}[2]{\item[\textit{n} \textbf{#1}\hfill]\textbf{#2}%
\dbgcmdidx{#1}{#2}}

\newcommand*{\mcmd}[2]{\mcmdPlus{#1}{#1}{#2}}

\newcommand*{\mcmdPlus}[3]{\item[\textbf{#1} \textit{par}\hfill]\textbf{#3}%
\dbgcmdidxPlus{#1}{#2}{#3}}

\newcommand*{\nmcmd}[2]{\item[{\it n\bf #1 {\it par}} \hfill] {\bf #2}%
\dbgcmdidx{#1}{#2}}
%-------------------------------


\let\ifonline=\iffalse
