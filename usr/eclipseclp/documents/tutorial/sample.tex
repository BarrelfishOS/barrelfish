% BEGIN LICENSE BLOCK
% Version: CMPL 1.1
%
% The contents of this file are subject to the Cisco-style Mozilla Public
% License Version 1.1 (the "License"); you may not use this file except
% in compliance with the License.  You may obtain a copy of the License
% at www.eclipse-clp.org/license.
% 
% Software distributed under the License is distributed on an "AS IS"
% basis, WITHOUT WARRANTY OF ANY KIND, either express or implied.  See
% the License for the specific language governing rights and limitations
% under the License. 
% 
% The Original Code is  The ECLiPSe Constraint Logic Programming System. 
% The Initial Developer of the Original Code is  Cisco Systems, Inc. 
% Portions created by the Initial Developer are
% Copyright (C) 2006 Cisco Systems, Inc.  All Rights Reserved.
% 
% Contributor(s): 
% 
% END LICENSE BLOCK

\chapter{Sample Chapter}

\section{Guidelines}

\begin{enumerate}
\item Prefer task-oriented structure over feature-oriented one.
	E.g.\ {\em Reading Data From A File} is better than
	{\em The read/3 Built-In}.
\item Make short paragraphs with meaningful headings.
\item Don't use subsubsections.
\item Provide frequent summaries.
\item Go for 50\% examples or pictures.
\item Use tables and overviews where possible.
\item Refer to where more detailed documentation can be found.
\end{enumerate}

\section{Layout}

\begin{itemize}
\item
	For summaries and overviews use
	$\backslash$quickref$[$width$]$\{caption\}\{text\}
	which puts the text in a grey box in a figure.
\item For pieces of code use
$\backslash$begin\{code\}
text
$\backslash$end\{code\}
\item For sample queries and toplevel interaction etc use
$\backslash$begin\{quote\}$\backslash$begin\{verbatim\} text
$\backslash$end\{verbatim\}$\backslash$end\{quote\}
\item For see-also references use $\backslash$See\{text\}.
\item For notes that the reader can safely skip on first reading,
use $\backslash$Note\{text\}.
\end{itemize}
\Note{For notes that the reader can safely skip on first reading,
use the Note command.}
\See{Use the See command to refer to where more details can be found.}

Here is a sample query
\begin{quote}\begin{verbatim}
?- X is 3+4.
X = 7
Yes.
\end{verbatim}\end{quote}

Here is sample code
\begin{code}
append([], C, C).
append([A|B], C, ABC) :- append(B, C, BC).
\end{code}
\quickref{A Summary}{
Every section should have a summary
\begin{itemize}
\item For summaries and overviews use the quickref command.
\end{itemize}
}

\section{References}

