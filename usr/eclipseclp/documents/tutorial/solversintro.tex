% BEGIN LICENSE BLOCK
% Version: CMPL 1.1
%
% The contents of this file are subject to the Cisco-style Mozilla Public
% License Version 1.1 (the "License"); you may not use this file except
% in compliance with the License.  You may obtain a copy of the License
% at www.eclipse-clp.org/license.
% 
% Software distributed under the License is distributed on an "AS IS"
% basis, WITHOUT WARRANTY OF ANY KIND, either express or implied.  See
% the License for the specific language governing rights and limitations
% under the License. 
% 
% The Original Code is  The ECLiPSe Constraint Logic Programming System. 
% The Initial Developer of the Original Code is  Cisco Systems, Inc. 
% Portions created by the Initial Developer are
% Copyright (C) 2006 Cisco Systems, Inc.  All Rights Reserved.
% 
% Contributor(s): 
% 
% END LICENSE BLOCK

%----------------------------------------------------------------------
\chapter{An Overview of the Constraint Libraries}
%HEVEA\cutdef[1]{section}
%----------------------------------------------------------------------

%----------------------------------------------------------------------
\section{Introduction}
%----------------------------------------------------------------------
In this section we shall briefly summarize the constraint solving
libraries of \eclipse which will be discussed in the rest of this tutorial.
%No examples are given here - each solver has its own documentation
%with examples for the interested reader.


%----------------------------------------------------------------------
\section{Implementations of Domains and Constraints}
%----------------------------------------------------------------------

\subsection{Suspended Goals: {\em suspend}}
\index{suspend}
\label{shortsecsuspend}
The constraint solvers of { \eclipse } are all implemented using suspended
goals.
The simplest implementation of any constraint is to suspend it
until all its variables are sufficiently instantiated, and then test it.

The {\em suspend} solver implements this behaviour for all
the mathematical constraints of { \eclipse },
$>=$, $>$, $=:=$, =\bsl=, $=<$ and $<$.


\subsection{Interval Solver: {\em ic}}
\index{ic}
\label{shortsecic}
The standard constraint solver offered by most constraint programming
systems is the {\em finite domain} solver, which applies constraint
propagation techniques developed in the AI community \cite{VanHentenryck}.  
{ \eclipse } supports finite domain constraints via the {\em ic}
library\footnote{and the {\em fd} library which will not be addressed in this tutorial}.
The library implements finite domains of integers, together with a basic
set of constraints.

In addition, {\em ic} also allows {\em continuous domains}
(in the form of numeric intervals), and constraints
(equations and inequations) between expressions involving
variables with continuous domains.
These expressions can contain non-linear functions such as $sin$ and built-in
constants such as $pi$.
%The user can also specify any piecewise linear unary function and {\em ic}
%will apply interval reasoning on that.
Integrality is treated as a constraint, and it is possible to mix
continuous and integral variables in the same constraint.
Specialised search techniques 
({\em splitting} \cite{VanHentenryck:95} and {\em squashing} 
\cite{lhomme96boosting}) support
the solving of problems with continuous variables.

Most constraints are also available in reified form, providing
a convenient way of combining several primitive constraints.

Note that the {\em ic} library itself implements only a standard,
basic set of arithmetic constraints. 
Many more finite domain
constraints can be defined, which have uses in specific applications.
The behaviour of these constraints is to prune the finite domains of
their variables, in just the same way as the standard constraints.
\eclipse{} offers several further libraries which implement such
constraints using the underlying domain of the {\em ic} library. 


\subsection{Global Constraints: {\em ic\_global}}
\index{ic_global}
\label{shortsecglobal}
\label{secglobalcstr}
One such library is {\em ic\_global}.
It supports a variety of constraints, each of which takes as an argument
a list of finite domain variables, of unspecified length.
Such constraints are called ``global'' constraints  \cite{beldiceanu}.
Examples of such constraints, available from the {\em ic\_global} library
are
\verb0alldifferent/10, \verb0maxlist/20, \verb0occurrences/30 and
\verb0sorted/20.
For more details see section \ref{secglobal} in chapter \ref{chapicintro}.


\subsection{Scheduling Constraints: {\em ic_cumulative, ic_edge_finder}}
\index{cumulative}
\index{edge_finder}
\label{shortsecsched}
There are several { \eclipse } libraries implementing global constraints for
scheduling applications.
The constraints take a list
of tasks (start times, durations and resource needs), and a maximum
resource level. They reduce the finite domains of the task start times
by reasoning on resource bottlenecks \cite{lepape}.  Three { \eclipse } libraries
implementing scheduling constraints are
{\em ic_cumulative}, {\em ic_edge\_finder} and {\em ic_edge\_finder3}.
They implement the same constraints declaratively, but with
different time complexity and strength of propagation.
For more details see the library documentation in the Reference Manual.



\subsection{Finite Integer Sets: {\em ic_sets}}
\index{ic_sets}
\label{shortsecsets}
The {\em ic_sets} library
implements constraints over the domain of finite
sets of integers\footnote{
the other set solvers lib(conjunto) and lib(fd_sets) are similar but not
addressed in this tutorial}.
The constraints are the usual relations over sets,
e.g.\ membership, inclusion, intersection, union, disjointness.
In addition, there are constraints between sets and integers, e.g.\
cardinality and weight constraints. For those, the {\em ic_sets} library
cooperates with the {\em ic} library.
For more details see chapter \ref{icsets}.




\subsection{Linear Constraints: {\em eplex}}
\index{eplex}
\label{shortseceplex}
{\em eplex} supports a tight integration \cite{Bockmayr} between an
external linear programming (LP) / mixed integer programming (MIP)
solver (XPRESS \cite{Dash} or CPLEX \cite{ILOG}) and { \eclipse }. 
Constraints as well as variables can be handled by the external LP/MIP
solver, by a propagation solver like {\em ic}, or by both.
Optimal solutions and other solution porperties can be returned to
\eclipse{} as required.
Search can be carried out either in \eclipse{} or in the external solver.
For more details see chapter \ref{chapeplex}.



\subsection{Constraints over symbols: {\em ic\_symbolic}}
\index{ic_symbolic}
\label{shortsecsymbolic}
The {\em ic\_symbolic} library supports variables ranging over ordered
symbolic domains (e.g. the names of products, the names of the weekdays),
and constraints over such variables. It is implemented by mapping such
variables and constraints to variables over integers and {\em ic}-constraints.


%----------------------------------------------------------------------
\section{User-Defined Constraints}
%----------------------------------------------------------------------
\subsection{Generalised Propagation: {\em propia}}
\index{propia}
\label{shortsecpropia}
The predicate {\em infers} takes as one argument
any user-defined predicate, and as a second argument a form of
propagation to be applied to that predicate.

This functionality enables the user to turn any predicate into a
constraint \cite{LeProvost93b}. The forms of propagation include finite
domains and intervals.
For more details see chapter \ref{chappropiachr}.

\subsection{Constraint Handling Rules: {\em ech}}
\index{ech}
\index{chr}
\label{shortsecech}
The user can also specify predicates using rules with guards
\cite{Fruehwirth}.  
They delay until the guard is entailed or disentailed, and then
execute or terminate accordingly. 

This functionality enables the user to implement constraints in a way
that is clearer than directly using the underlying {\em suspend}
library.
For more details see chapter \ref{chappropiachr}.


%----------------------------------------------------------------------
\section{Search and Optimisation Support}
%----------------------------------------------------------------------

\subsection{Tree Search Methods: {\em ic_search}}
\index{ic_search}
\label{shortsecsearch}
\eclipse{} has built-in backtracking and is therefore well suited for
performing depth-first tree search.
With combinatorial problems, naive depth-first search is usually not
good enough, even in the presence of constraint propagation.
It is usually necessary to apply heuristics, and if the problems are
large, one may even need to resort to incomplete search.
The {\em ic_search} contains a collection of predefined, easy-to-use
search heuristics as well as incomplete tree search strategies, applicable
to problems involving {\em ic} variables.
For more details see chapter \ref{chapsearch}.


\subsection{Optimisation: {\em branch_and_bound}}
\index{branch_and_bound}
\label{shortsecbb}
Solvers that are based on constraint propagation are typically only
concerned with satisfiability, i.e.\ with finding some or all solutions
to a problems.
The branch-and-bound method is a general technique to build optimisation
on top of a satisfiability solver.
The \eclipse{} {\em branch_and_bound} library is a solver-independent
implementation of the branch-and-bound method, and provides a number
of options and variants of the basic technique.


%----------------------------------------------------------------------
\section{Hybridisation Support}
%----------------------------------------------------------------------
\subsection{Repair and Local Search: {\em repair}}
\index{repair}
\label{shortsecrepair}
The {\em repair} library allows a {\em tentative} value to be
associated with any variable \cite{cp99wkshoptalk}.
This tentative value may violate constraints on the variable, in which
case the constraint is recorded in a list of violated constraints.
The repair library also supports propagation {\em invariants}
\cite{Localizer}.
Using invariants,  if a variable's tentative
value is changed, the consequences of this change can be propagated to
any variables whose tentative values depend on the changed one.
The use of tentative values in search is illustrated in chapter \ref{chaprepair}.
 

\subsection{Hybrid: {\em ic\_probing\_for\_scheduling}}
\index{probing_for_scheduling}
\label{shortsecprobing}
For scheduling applications where the cost is dependent on each start
time, a combination of solvers can be very powerful.
For example, we can use finite domain
propagation to reason on 
resources and linear constraint solving to reason on cost \cite{HaniProbe}.
The {\em probing\_for\_scheduling} library supports such a combination,
via a similar user interface to the {\em cumulative} constraint mentioned
above in section \ref{secglobalcstr}.
For more details see chapter \ref{chaphybrid}.


%----------------------------------------------------------------------
\section{Other Libraries}
%----------------------------------------------------------------------
The solvers described above are just a few of the many libraries
available in ECLiPSe and listed in the \eclipse{} library directory.
Any \eclipse{} user who has implemented a constraint solver is
encouraged to make it available to the user community and publicise
it via the {\tt eclipse-users@icparc.ic.ac.uk} mailing list!
Comments and suggestions on the existing libraries are also welcome!


%\bibliographystyle{alpha}
%\bibliography{solver_intro}

%\end{document}

%HEVEA\cutend
