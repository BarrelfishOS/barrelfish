%%%%%%%%%%%%%%%%%%%%%%%%%%%%%%%%%%%%%%%%%%%%%%%%%%%%%%%%%%%%%%%%%%%%%%%%%%
% Copyright (c) 2015, ETH Zurich.
% All rights reserved.
%
% This file is distributed under the terms in the attached LICENSE file.
% If you do not find this file, copies can be found by writing to:
% ETH Zurich D-INFK, Universitaetstr 6, CH-8092 Zurich. Attn: Systems Group.
%%%%%%%%%%%%%%%%%%%%%%%%%%%%%%%%%%%%%%%%%%%%%%%%%%%%%%%%%%%%%%%%%%%%%%%%%%

\documentclass[a4paper,11pt,twoside]{report}
\usepackage{bftn}
\usepackage{calc}
\usepackage{verbatim}
\usepackage{xspace}
\usepackage{pifont}
\usepackage{pxfonts}
\usepackage{textcomp}
\usepackage{amsmath}
\usepackage{multirow}
\usepackage{listings}
\usepackage{todonotes}
\usepackage{hyperref}

\title{Skate in Barrelfish}
\author{Barrelfish project}
% \date{\today}   % Uncomment (if needed) - date is automatic
\tnnumber{020}
\tnkey{Skate}


\lstdefinelanguage{skate}{
    morekeywords={schema,typedef,fact,enum},
    sensitive=true,
    morecomment=[l]{//},
    morecomment=[s]{/*}{*/},
    morestring=[b]",
}

\presetkeys{todonotes}{inline}{}

\begin{document}
\maketitle      % Uncomment for final draft

\begin{versionhistory}
\vhEntry{0.1}{16.11.2015}{MH}{Initial Version}
\vhEntry{0.2}{20.04.2017}{RA}{Renaming ot Skate and expanding.}
\end{versionhistory}

% \intro{Abstract}    % Insert abstract here
% \intro{Acknowledgements}  % Uncomment (if needed) for acknowledgements
\tableofcontents    % Uncomment (if needed) for final draft
% \listoffigures    % Uncomment (if needed) for final draft
% \listoftables     % Uncomment (if needed) for final draft
\cleardoublepage
\setcounter{secnumdepth}{2}

\newcommand{\fnname}[1]{\textit{\texttt{#1}}}%
\newcommand{\datatype}[1]{\textit{\texttt{#1}}}%
\newcommand{\varname}[1]{\texttt{#1}}%
\newcommand{\keywname}[1]{\textbf{\texttt{#1}}}%
\newcommand{\pathname}[1]{\texttt{#1}}%
\newcommand{\tabindent}{\hspace*{3ex}}%
\newcommand{\Skate}{\lstinline[language=skate]}
\newcommand{\ccode}{\lstinline[language=C]}

\lstset{
  language=C,
  basicstyle=\ttfamily \small,
  keywordstyle=\bfseries,
  flexiblecolumns=false,
  basewidth={0.5em,0.45em},
  boxpos=t,
  captionpos=b
}


%%%%%%%%%%%%%%%%%%%%%%%%%%%%%%%%%%%%%%%%%%%%%%%%%%%%%%%%%%%%%%%%%%%%%%%%%%%%%%
\chapter{Introduction and usage}
\label{chap:introduction}
%%%%%%%%%%%%%%%%%%%%%%%%%%%%%%%%%%%%%%%%%%%%%%%%%%%%%%%%%%%%%%%%%%%%%%%%%%%%%%

\emph{Skate}\footnote{Skates are cartilaginous fish belonging to the family 
Rajidae in the superorder Batoidea of rays. More than 200 species have been 
described, in 32 genera. The two subfamilies are Rajinae (hardnose skates) and 
Arhynchobatinae (softnose skates). 
Source: \href{https://en.wikipedia.org/wiki/Skate_(fish)}{Wikipedia}}
is a domain specific language to describe the schema of 
Barrelfish's System Knowledge Base (SKB)~\cite{skb}. The SKB stores all 
statically or dynamically discovered facts about the system. Static facts are 
known and exist already at compile time of the SKB ramdisk or are added through
an initialization script or program. 

Examples for static facts include the device database, that associates known 
drivers with devices or the devices of a wellknown SoC. Dynamic facts, on the 
otherhand, are added to the SKB during and based on hardware discovery. 
Examples for dynamic facts include the number of processors or PCI Express 
devices. 

Inside the SKB, a prolog based constraint solver takes the added facts and 
computes a solution for hardware configuration such as PCI bridge programming,
NUMA information for memory allocation or device driver lookup. Programs can 
query the SKB using Prolog statements and obtain device configuration and PCI 
bridge programming, interrupt routing and constructing routing trees for IPC. 
Applications can use information to determine hardware characteristics such as 
cores, nodes, caches and memory as well as their affinity.


The Skate language is used to define format of those facts. The DSL is then 
compiled into a set of fact definitions and functions that are wrappers arround
the SKB client functions, in particular \texttt{skb\_add\_fact()}, to ensure
the correct format of the added facts.  

The intention when designing Skate is that the contents of system descriptor
tables such as ACPI, hardware information obtained by CPUID or PCI discovery
can be extracted from the respective manuals and easily specified in a Skate 
file. 

Skate complements the SKB by defining a \emph{schema} of the data stored in
the SKB. A schema defines facts and their structure, which is similar to Prolog
facts and their arity. A code-generation tool generates a C-API to populate the
SKB according to a specific schema instance.

The Skate compiler is written in Haskell using the Parsec parsing library. It
generates C header files from the Skate files. In addition it supports the 
generation of Schema documentation.

The source code for Skate can be found in \texttt{SOURCE/tools/skate}.


\section{Command line options}
\label{sec:cmdline}

\begin{verbatim}
$ skate <options> INFILE.skt
\end{verbatim}


Where options is one of
\begin{description}
  \item[-o] \textit{filename} The output file name
  \item[-D] generate documentation
  \item[-H] generate headerfile
\end{description}



%%%%%%%%%%%%%%%%%%%%%%%%%%%%%%%%%%%%%%%%%%%%%%%%%%%%%%%%%%%%%%%%%%%%%%%%%%%%%%
\chapter{Lexical Conventions}
\label{chap:lexer}
%%%%%%%%%%%%%%%%%%%%%%%%%%%%%%%%%%%%%%%%%%%%%%%%%%%%%%%%%%%%%%%%%%%%%%%%%%%%%%

We initialize a Java-style-like parser for Skate and thus adopt the following
convention. We follow a similar convention as opted by modern day programming 
languages like C and Java.

\begin{description}
\item[Whitespace:]  As in C and Java, Skate considers sequences of
  space, newline, tab, and carriage return characters to be
  whitespace.  Whitespace is generally not significant. 

\item[Comments:] Skate supports C-style comments.  Single line comments
  start with \texttt{//} and continue until the end of the line.
  Multiline comments are enclosed between \texttt{/*} and \texttt{*/};
  anything inbetween is ignored and treated as white space.

\item[Identifiers:] Valid Skate identifiers are sequences of numbers
  (0-9), letters (a-z, A-Z) and the underscore character ``\texttt{\_}''.  They
  must start with a letter or ``\texttt{\_}''.  
  \begin{align*}
    identifier & \rightarrow ( letter \mid \_ ) (letter \mid digit \mid \_)^{\textrm{*}} \\
    letter & \rightarrow (\textsf{A \ldots Z} \mid  \textsf{a \ldots z})\\
    digit & \rightarrow (\textsf{0 \ldots 9})
\end{align*}

  Note that a single underscore ``\texttt{\_}'' by itself is a special,
  ``don't care'' or anonymous identifier which is treated differently
  inside the language. 
  
\item[Integer Literals:] A Skate integer literal is a sequence of
  digits, optionally preceded by a radix specifier.  As in C, decimal (base 10)
  literals have no specifier and hexadecimal literals start with
  \texttt{0x}.  Binary literals start with \texttt{0b}. 

  In addition, as a special case the string \texttt{1s} can be used to
  indicate an integer which is composed entirely of binary 1's. 

\begin{align*}
digit & \rightarrow (\textsf{0 \ldots 9})^{\textrm{1}}\\
hexadecimal & \rightarrow (\textsf{0x})(\textsf{0 \ldots 9} \mid \textsf{A \ldots F} \mid \textsf{a \ldots f})^{\textrm{1}}\\
binary & \rightarrow (\textsf{0b})(\textsf{0, 1})^{\textrm{1}}\\
\end{align*}


\item[Reserved words:] The following are reserved words in Skate:
\begin{alltt}
  \begin{tabbing}
xxxxxxxxx \= xxxxxxxxx \= xxxxxxxxx \= xxxxxxxxx \= xxxxxxxxx \= xxxxxxxxx \kill
fact \> query \> flags \> constants \>  \>  \\
  \end{tabbing}
\end{alltt}

\item[Special characters:] The following characters are used as operators,
  separators, terminators or other special purposes in Skate:
\begin{alltt}

  \{ \} [ ] ( ) + - * / ; , . = 

\end{alltt}

\end{description}



%%%%%%%%%%%%%%%%%%%%%%%%%%%%%%%%%%%%%%%%%%%%%%%%%%%%%%%%%%%%%%%%%%%%%%%%%%%%%%
\chapter{Declarations}
\label{chap:declaration}
%%%%%%%%%%%%%%%%%%%%%%%%%%%%%%%%%%%%%%%%%%%%%%%%%%%%%%%%%%%%%%%%%%%%%%%%%%%%%%

In this chapter we define the layout of a skate file, which declarations it
must contain and what other declarations it can have.

\section{The Skate File}
A Skate file must consist of zero or more \emph{import} declarations (see
~\ref{sec:decl:schema}) followed by a single \emph{schema} declaration 
(see~\ref{sec:decl:schema}) which contains the actual fact definitions, etc.

\begin{syntax}
(Import)*
Schema
\end{syntax}

%\begin{align*}
%    skatefile & \rightarrow ( Import )^{\textrm{*}} (Schema)
%\end{align*}

\section{Imports}\label{sec:decl:import}
An import declaration makes the definitions in a different device file
available in the current device definition, as described below.  The
syntax of an import declaration is as follows:

\begin{syntax}
import \synit{schema};
\end{syntax}

\begin{description}
\item[schema] is the name of the schema to import definitions from.  
\end{description}

The Skate compiler will search for a file with the appropriate name and
parse this at the same time as the main file, along with this file's
imports, and so on.  Cyclic dependencies between device files will not
cause errors, but at present are unlikely to result in C header files
which will successfully compile. 

\section{Schemas}\label{sec:decl:schema}

\begin{syntax}
schema \synit{name} "\synit{description}"
\verb+{+
  \synit{declaration};
  \ldots
\verb+}+;
\end{syntax}

\begin{description}
\item[name] is an identifier for the Schema type, and will be used to
  generate identifiers in the target language (typically C).  
  The name of the name of the schema \emph{must} correspond to the
  filename of the file, including case sensitivity: for example, 
  the file \texttt{cpuid.schema} will define a schema type
  of name \texttt{cpuid}. 

\item [description] is a string literal in double quotes, which
  describes the schema type being specified, for example \texttt{"CPUID 
  Information Schema"}. 

\item [declaration] One of the declarations described in~\ref{sec:decl:decls}.

\end{description}

\section{Declarations}\label{sec:decl:decls}

\subsection{Facts}

\subsection{Flags}

\subsection{Constants}

\subsection{Queries}




\chapter{High-level overview: Skate schema for the SKB}
\label{chap:overview}




Throughout this document a example describing certain CPU characteristics is
being used. It is an abstraction of the CPUID information found on Intel and AMD
CPUs. The source code for the Skate tool can be found in
\pathname{tools/Skate}.

\section{Schema language}

The schema language is a C-like language defining facts and their attributes. An
attribute is typed and can be one of a basic type, a fact or an enumeration
type. Basic types include signed and unsigned integers as well as memory
addresses with a machine-specific word size and strings. As facts can be
attributes of other facts we can achieve a simple nesting of facts.

In Listing \ref{lst:sample_schema} an excerpt of the CPUID schema is shown. In
Barrelfish, a CPU core is identified using an eight bit unsigned number, hence
we use a \Skate{typedef} to declare a \varname{core\_id} type.

\begin{lstlisting}[caption={Sample Skate schema definition},
label={lst:sample_schema},language=Skate]

schema cpuid "CPUID abstractions" {
    typedef uint8 core_ID;
    fact vendor "Literal vendor" {
        core_id Core_id "Core";
        string vendor "Core vendor";
    };
};
\end{lstlisting}

Each schema starts with the keyword \Skate{schema}. It is followed by an
identifier and a documentation string for that schema. The identifier also needs
to be equal to the file name, i.e.~the schema \varname{cpuid} needs to reside in
\varname{cpuid.Skate}. After the schema declaration we find a list of facts,
typdefs and enums enclosed in curly braces (\{\}).

Each fact has an identifying name and a documentation string. Its attributes
follow the documentation string and are enclosed in curly braces. The attributes
are a tuple consisting of a type, a name and an optional documentation. The type
can be a basic type, a fact or an enumeration type. Note that types need to have
unique names, there is no mechanism to derive new types from existing types.

Typedefs allow to use a different identifier for a built-in type. This is useful
for abstracting common built-in types.

\todo{Explain enumerations}

\section{Code generation}

Skate can translate schema definitions into documentation and C code (header
and implementation). Here, we explain the mechanisms that are used to generate a
C implementation of the schema.

The example presented in Listing \ref{lst:sample_schema} translates into C
structures and type definitions for both \varname{core\_id} and
\varname{vendor}. Functions are provided to populate the SKB with instanced of
facts.

\section{Translating facts}

Each fact corresponds to a structure in C. Skate generates both a structure
and a type definition for each fact. Using the example, it creates the
structure shown in Listing \ref{lst:c_vendor}.

\begin{lstlisting}[caption={C header for fact
\varname{vendor}},label={lst:c_vendor},language=C]
struct cpuid__vendor {
    /* CPU ID */
    cpuid__core_id_t Core_ID;
    /* Vendor string */
    char *vendor;
};
typedef struct cpuid__vendor cpuid__vendor_t;
\end{lstlisting}

Each fact attribute is represented by a \ccode!struct!~field. Nested structures are
allowed, but there cannot be a recursive dependency.

On top of this, it generates a function
\ccode!cpuid__vendor__add(struct cpuid__vendor *)!.
This function can be used to add facts of the specified type to the SKB. For
example, the code presented in Listing \ref{lst:c_add_vendor} adds an Intel core
to the SKB.

\begin{lstlisting}[caption={C example to add a vendor fact.},
label={lst:c_add_vendor}, language=C]
cpuid__vendor_t vendor;
vendor.Core_ID = core_id;
vendor.vendor = vendor_string;

errval_t err = cpuid__vendor__add(&vendor);
\end{lstlisting}

\section{Mapping facts to Prolog facts}

Each fact added to the SKB using Skate is represented by a single Prolog
functor.  The functor name in Prolog consist of the schema and fact name.  The
fact defined in Listing \ref{lst:sample_schema} is represented by the functor
\lstinline!cpuid_vendor!~and has an arity of three.

\section{Integration with the build process}

Skate is a tool that is integrated with Hake. Add the attribute
\lstinline!SkateSchema!~to a Hakefile to invoke Skate as shown in Listing
\ref{lst:Skate_hake}.

\begin{lstlisting}[caption={Including Skate schemata in Hake},
label={lst:Skate_hake}, language=Haskell]
[ build application {
    SkateSchema = [ "cpu" ]
    ... 
} ]
\end{lstlisting}

Adding an entry for \varname{SkateSchema} to a Hakefile will generate both
header and implementation and adds it to the list of compiled resources. A
Skate schema is referred to by its name and Skate will look for a file
ending with \varname{.Skate} containing the schema definition.

The header file is placed in \pathname{include/schema} in the build tree, the C
implementation is stored in the Hakefile application or library directory.

\chapter{Limitations \& Work in Progress}

Skate is not yet in a feature-complete state. It does not allow querying facts
or specifying more advanced rules for types besides the ones being enforced by C
on \ccode!struct!s. Although features are missing users can start writing
schemata as the syntax is not going to change significantly at the moment.


%%%%%%%%%%%%%%%%%%%%%%%%%%%%%%%%%%%%%%%%%%%%%%%%%%%%%%%%%%%%%%%%%%%%%%%%%%%%%%%%
\bibliographystyle{abbrv}
\bibliography{barrelfish}

\end{document}
