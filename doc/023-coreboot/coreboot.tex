\documentclass[a4paper,11pt,twoside]{report}
\usepackage{bftn}
\usepackage{calc}
\usepackage{verbatim}
\usepackage{xspace}
\usepackage{pifont}
\usepackage{pxfonts}
\usepackage{textcomp}
\usepackage{amsmath}
\usepackage{multirow}
\usepackage{listings}
\usepackage[framemethod=default]{mdframed}
\usepackage[shortlabels]{enumitem}
\usepackage{parskip}
\usepackage{xparse}

\newcommand{\todo}[1]{[\textcolor{red}{\emph{#1}}]}

\title{Coreboot in Barrelfish}
\author{Barrelfish project}
\tnnumber{23}
\tnkey{Coreboot}

\begin{document}
\maketitle			% Uncomment for final draft

\begin{versionhistory}
\vhEntry{0.1}{28.02.2017}{GZ}{Initial Version}
\vhEntry{0.1}{02.03.2017}{RA}{Adding basic structure}
\end{versionhistory}

% \intro{Abstract}		% Insert abstract here
% \intro{Acknowledgements}	% Uncomment (if needed) for acknowledgements
\tableofcontents		% Uncomment (if needed) for final draft
% \listoffigures		% Uncomment (if needed) for final draft
% \listoftables			% Uncomment (if needed) for final draft
\cleardoublepage
\setcounter{secnumdepth}{2}

\newcommand{\fnname}[1]{\textit{\texttt{#1}}}%
\newcommand{\datatype}[1]{\textit{\texttt{#1}}}%
\newcommand{\varname}[1]{\texttt{#1}}%
\newcommand{\keywname}[1]{\textbf{\texttt{#1}}}%
\newcommand{\pathname}[1]{\texttt{#1}}%
\newcommand{\tabindent}{\hspace*{3ex}}%
\newcommand{\sockeye}{\lstinline[language=sockeye]}
\newcommand{\ccode}{\lstinline[language=C]}

\lstset{
  language=C,
  basicstyle=\ttfamily \small,
  keywordstyle=\bfseries,
  flexiblecolumns=false,
  basewidth={0.5em,0.45em},
  boxpos=t,
  captionpos=b
}

\chapter{Introduction}
\label{chap:introduction}

This document describes the way Barrelfish boots secondary cores on supported 
architectures.

\chapter{Generic Operations}
\label{chap:generic}

\section{Terminology}
\begin{itemize}
    \item OLD cpudriver
    \item NEW cpudriver
    \item KCB
    \item bootdriver
\end{itemize}

\section{Boot process}
\begin{enumerate}
    \item Instantiation of a boot driver for a \emph{specific} core.
    \item Invoking the operations of the boot driver
    \item opt: Removing the old cpu driver
    \item (re)start core with the new CPU driver
\end{enumerate}

\section{CPU Capability}
\todo{Figure out how we can integrate the right to boot/reset etc. into the 
    cap system. CF. interrupt capabilities}

\section{Components}

\subsection{Bootdriver}
\begin{itemize}
    \item Input args: information about the core it's responsible, capabilities
    \item Exported interface: (cf. cpuboot operations)
\end{itemize}


\subsubsection{Operations (current)}
\begin{itemize}
\item boot: Boot a fresh core with a KCB. "boot <target coreid>"
\item update: Update the kernel on an existing core. "update <target coreid>"
\item stop: Stop execution on an existing core. "stop <target coreid>"
\item give: Give kcb from one core to another. "give <from kcbid> <to kcbid>",
\item rmkcb: Remove a running KCB from a core. "rm <kcbid>",
\item park: Stop execution on an existing core and park KCB on another core.",
    "park <kcbid to stop> <recv kcbid>"
\item unpark: Reestablish parked KCB on its original core."unpark <kcbid to 
unpark>"
\item lscpu: List current status of all cores. "lscpu"
\item lskcb: List current KCBs. "lskcb"
\end{itemize}





\chapter{Booting x86 Cores}
\label{chap:x86}
\todo{explain x86 specific boot arguements / protocols}

\section{Core discovery and identification}

\section{Booting a new core}

\chapter{Booting ARMv7 Cores}
\label{chap:armv7}
\todo{explain ARMv7 specific boot arguements / protocols}

\section{Core discovery and identification}

\section{Booting a new core}

\chapter{Booting ARMv8 Cores}
\label{chap:armv8}

\todo{explain ARMv8 specific boot arguements / protocols}

\section{Core discovery and identification}

\section{Booting a new core}
% 
%http://infocenter.arm.com/help/topic/com.arm.doc.den0044b/DEN0044B_Server_Base_Boot_Requirements.pdf

\subsection{Power State Coordination Interface (PSCI)}

% 
%http://infocenter.arm.com/help/topic/com.arm.doc.den0022c/DEN0022C_Power_State_Coordination_Interface.pdf

\subsection{Parking Protocol}
%https://acpica.org/sites/acpica/files/MP%20Startup%20for%20ARM%20platforms.doc

\subsection{All at once}
Raspberry Pi


%%%%%%%%%%%%%%%%%%%%%%%%%%%%%%%%%%%%%%%%%%%%%%%%%%%%%%%%%%%%%%%%%%%%%%%%%%%%%%%%
\bibliographystyle{abbrv}
\bibliography{barrelfish}

\end{document}
