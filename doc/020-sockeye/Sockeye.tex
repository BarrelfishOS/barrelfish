%%%%%%%%%%%%%%%%%%%%%%%%%%%%%%%%%%%%%%%%%%%%%%%%%%%%%%%%%%%%%%%%%%%%%%%%%%
% Copyright (c) 2015, ETH Zurich.
% All rights reserved.
%
% This file is distributed under the terms in the attached LICENSE file.
% If you do not find this file, copies can be found by writing to:
% ETH Zurich D-INFK, Universitaetstr 6, CH-8092 Zurich. Attn: Systems Group.
%%%%%%%%%%%%%%%%%%%%%%%%%%%%%%%%%%%%%%%%%%%%%%%%%%%%%%%%%%%%%%%%%%%%%%%%%%

\documentclass[a4paper,11pt,twoside]{report}
\usepackage{bftn}
\usepackage{calc}
\usepackage{verbatim}
\usepackage{xspace}
\usepackage{pifont}
\usepackage{pxfonts}
\usepackage{textcomp}
\usepackage{amsmath}
\usepackage{multirow}
\usepackage{listings}
\usepackage{todonotes}

\title{Sockeye in Barrelfish}
\author{Barrelfish project}
% \date{\today}		% Uncomment (if needed) - date is automatic
\tnnumber{20}
\tnkey{Sockeye}

\lstdefinelanguage{sockeye}{
    morekeywords={schema,typedef,fact,enum},
    sensitive=true,
    morecomment=[l]{//},
    morecomment=[s]{/*}{*/},
    morestring=[b]",
}

\presetkeys{todonotes}{inline}{}

\begin{document}
\maketitle			% Uncomment for final draft

\begin{versionhistory}
\vhEntry{0.1}{16.11.2015}{MH}{Initial Version}
\end{versionhistory}

% \intro{Abstract}		% Insert abstract here
% \intro{Acknowledgements}	% Uncomment (if needed) for acknowledgements
\tableofcontents		% Uncomment (if needed) for final draft
% \listoffigures		% Uncomment (if needed) for final draft
% \listoftables			% Uncomment (if needed) for final draft
\cleardoublepage
\setcounter{secnumdepth}{2}

\newcommand{\fnname}[1]{\textit{\texttt{#1}}}%
\newcommand{\datatype}[1]{\textit{\texttt{#1}}}%
\newcommand{\varname}[1]{\texttt{#1}}%
\newcommand{\keywname}[1]{\textbf{\texttt{#1}}}%
\newcommand{\pathname}[1]{\texttt{#1}}%
\newcommand{\tabindent}{\hspace*{3ex}}%
\newcommand{\sockeye}{\lstinline[language=sockeye]}
\newcommand{\ccode}{\lstinline[language=C]}

\lstset{
  language=C,
  basicstyle=\ttfamily \small,
  keywordstyle=\bfseries,
  flexiblecolumns=false,
  basewidth={0.5em,0.45em},
  boxpos=t,
  captionpos=b
}

\chapter{Introduction}
\label{chap:introduction}

This document describes Sockeye, a tool to define schemata for the SKB. It
explains functionality, syntax and code generation plus the API concepts of
generated code. The documentation uses a practical example from Barrelfish to
demonstrate Sockeye.

\chapter{High-level overview: Sockeye schema for the SKB}
\label{chap:overview}

In Barrelfish, the SKB\cite{btn019-devicedrivers} stores configuration and
system data which can be queried by programs using Prolog. It serves device
configuration and PCI bridge programming, interrupt routing and constructing
routing trees for IPC. Applications can use information to determine hardware
characteristics such as cores, nodes, caches and memory as well as their
affinity.

Sockeye complements the SKB by defining a \emph{schema} of the data stored in
the SKB. A schema defines facts and their structure, which is similar to Prolog
facts and their arity. A code-generation tool generates a C-API to populate the
SKB according to a specific schema instance.

Throughout this document a example describing certain CPU characteristics is
being used. It is an abstraction of the CPUID information found on Intel and AMD
CPUs. The source code for the Sockeye tool can be found in
\pathname{tools/sockeye}.

\section{Schema language}

The schema language is a C-like language defining facts and their attributes. An
attribute is typed and can be one of a basic type, a fact or an enumeration
type. Basic types include signed and unsigned integers as well as memory
addresses with a machine-specific word size and strings. As facts can be
attributes of other facts we can achieve a simple nesting of facts.

In Listing \ref{lst:sample_schema} an excerpt of the CPUID schema is shown. In
Barrelfish, a CPU core is identified using an eight bit unsigned number, hence
we use a \sockeye{typedef} to declare a \varname{core\_id} type.

\begin{lstlisting}[caption={Sample Sockeye schema definition},
label={lst:sample_schema},language=sockeye]

schema cpuid "CPUID abstractions" {
    typedef uint8 core_ID;
    fact vendor "Literal vendor" {
        core_id Core_id "Core";
        string vendor "Core vendor";
    };
};
\end{lstlisting}

Each schema starts with the keyword \sockeye{schema}. It is followed by an
identifier and a documentation string for that schema. The identifier also needs
to be equal to the file name, i.e.~the schema \varname{cpuid} needs to reside in
\varname{cpuid.sockeye}. After the schema declaration we find a list of facts,
typdefs and enums enclosed in curly braces (\{\}).

Each fact has an identifying name and a documentation string. Its attributes
follow the documentation string and are enclosed in curly braces. The attributes
are a tuple consisting of a type, a name and an optional documentation. The type
can be a basic type, a fact or an enumeration type. Note that types need to have
unique names, there is no mechanism to derive new types from existing types.

Typedefs allow to use a different identifier for a built-in type. This is useful
for abstracting common built-in types.

\todo{Explain enumerations}

\section{Code generation}

Sockeye can translate schema definitions into documentation and C code (header
and implementation). Here, we explain the mechanisms that are used to generate a
C implementation of the schema.

The example presented in Listing \ref{lst:sample_schema} translates into C
structures and type definitions for both \varname{core\_id} and
\varname{vendor}. Functions are provided to populate the SKB with instanced of
facts.

\section{Translating facts}

Each fact corresponds to a structure in C. Sockeye generates both a structure
and a type definition for each fact. Using the example, it creates the
structure shown in Listing \ref{lst:c_vendor}.

\begin{lstlisting}[caption={C header for fact
\varname{vendor}},label={lst:c_vendor},language=C]
struct cpuid__vendor {
    /* CPU ID */
    cpuid__core_id_t Core_ID;
    /* Vendor string */
    char *vendor;
};
typedef struct cpuid__vendor cpuid__vendor_t;
\end{lstlisting}

Each fact attribute is represented by a \ccode!struct!~field. Nested structures are
allowed, but there cannot be a recursive dependency.

On top of this, it generates a function
\ccode!cpuid__vendor__add(struct cpuid__vendor *)!.
This function can be used to add facts of the specified type to the SKB. For
example, the code presented in Listing \ref{lst:c_add_vendor} adds an Intel core
to the SKB.

\begin{lstlisting}[caption={C example to add a vendor fact.},
label={lst:c_add_vendor}, language=C]
cpuid__vendor_t vendor;
vendor.Core_ID = core_id;
vendor.vendor = vendor_string;

errval_t err = cpuid__vendor__add(&vendor);
\end{lstlisting}

\section{Mapping facts to Prolog facts}

Each fact added to the SKB using Sockeye is represented by a single Prolog
functor.  The functor name in Prolog consist of the schema and fact name.  The
fact defined in Listing \ref{lst:sample_schema} is represented by the functor
\lstinline!cpuid_vendor!~and has an arity of three.

\section{Integration with the build process}

Sockeye is a tool that is integrated with Hake. Add the attribute
\lstinline!sockeyeSchema!~to a Hakefile to invoke Sockeye as shown in Listing
\ref{lst:sockeye_hake}.

\begin{lstlisting}[caption={Including Sockeye schemata in Hake},
label={lst:sockeye_hake}, language=Haskell]
[ build application {
    sockeyeSchema = [ "cpu" ]
    ... 
} ]
\end{lstlisting}

Adding an entry for \varname{sockeyeSchema} to a Hakefile will generate both
header and implementation and adds it to the list of compiled resources. A
Sockeye schema is referred to by its name and Sockeye will look for a file
ending with \varname{.sockeye} containing the schema definition.

The header file is placed in \pathname{include/schema} in the build tree, the C
implementation is stored in the Hakefile application or library directory.

\chapter{Limitations \& Work in Progress}

Sockeye is not yet in a feature-complete state. It does not allow querying facts
or specifying more advanced rules for types besides the ones being enforced by C
on \ccode!struct!s. Although features are missing users can start writing
schemata as the syntax is not going to change significantly at the moment.


%%%%%%%%%%%%%%%%%%%%%%%%%%%%%%%%%%%%%%%%%%%%%%%%%%%%%%%%%%%%%%%%%%%%%%%%%%%%%%%%
\bibliographystyle{abbrv}
\bibliography{barrelfish}

\end{document}
